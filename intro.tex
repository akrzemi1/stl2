
\rSec0[intro]{General}

\rSec1[intro.scope]{Scope}

\pnum
This Technical Specification describes extensions to the C++
Programming Language \ref{intro.refs} that
permit operations on ranges of data. These extensions include
changes and additions to the existing library facilities as well
as the extension of some core language facilities. In particular,
changes and extensions to the Standard Library include:
\begin{itemize}
\item the reformulation of concept requirements using concepts (\ref{xxx}).
\item ..., and
\item new facilities for composing range abstractions (\ref{range}).
\end{itemize}
%
Changes to the core language include:
\begin{itemize}
\item the extension of the range-based \tcode{for} statement to support
the new iterator range requirements (\ref{range.access}), and
\item the definition of new constraints (\ref{temp.constr.constr}) to
support the changes to the library.
\end{itemize}

\pnum
The International Standard, ISO/IEC 14882, provides important context
and specification for this Technical Specification. This document is
written as a set of changes against that specification. Instructions
to modify or add paragraphs are written as explicit instructions.
Modifications made directly to existing text from the International
Standard use \added{underlining} to represent added text and
\removed{strikethrough} to represent deleted text.


\rSec1[intro.refs]{Normative references}

\pnum
The following referenced document is indispensable for the
application of this document. For dated references, only the
edition cited applies. For undated references, the latest edition
of the referenced document (including any amendments) applies.

\begin{itemize}
\item ISO/IEC 14882:2014, \doccite{Programming Languages - \Cpp}
\item ISO/IEC ????, \doccite{Technical Specification - \Cpp extensions for concepts}
\end{itemize}

ISO/IEC 14882:2014 is herein called the \defn{\Cpp Standard}, and
???? is called the \defn{Concepts TS}.

\rSec1[intro.compliance]{Implementation compliance}

\pnum
Conformance requirements for this specification are the same as those
defined in \ref{intro.compliance} in the \Cpp Standard.
\enternote
Conformance is defined in terms of the behavior of programs.
\exitnote

\rSec1[intro.ack]{Acknowledgments}

\pnum
The design of this specification is based, in part, on a concept
specification of the algorithms part of the C++ standard library, known
as ``The Palo Alto'' report (WG21 N3351), which was developed by a large
group of experts as a test of the expressive power of the idea of
concepts.
