%!TEX root = std.tex

\setcounter{chapter}{23}
\rSec0[iterators]{Iterators library}

\rSec1[iterators.general]{General}

\pnum
This Clause describes components that \Cpp programs may use to perform
iterations over containers (Clause \cxxref{containers}),
streams~(\cxxref{iostream.format}),
\removed{and} stream buffers~(\cxxref{stream.buffers})
\added{, and ranges~(\ref{range.iterables})}.

\pnum
The following subclauses describe
iterator requirements, and
components for
iterator primitives,
predefined iterators,
and stream iterators,
as summarized in Table~\ref{tab:iterators.lib.summary}.

\begin{libsumtab}{Iterators library summary}{tab:iterators.lib.summary}
\ref{iterator.requirements} & Requirements        &                           \\ \rowsep
\ref{iterator.primitives} & Iterator primitives   &   \tcode{<iterator>}      \\
\ref{predef.iterators} & Predefined iterators     &                           \\
\ref{stream.iterators} & Stream iterators         &                           \\
\added{\ref{iterables}} & \added{Iterables}       &                           \\
\end{libsumtab}


\rSec1[iterator.requirements]{Iterator requirements}

\rSec2[iterator.requirements.general]{In general}

\pnum
\indextext{requirements!iterator}%
Iterators are a generalization of pointers that allow a \Cpp program to work with different data structures
(containers\added{ and ranges}) in a uniform manner.
To be able to construct template algorithms that work correctly and
efficiently on different types of data structures, the library formalizes not just the interfaces but also the
semantics and complexity assumptions of iterators.
All input iterators
\tcode{i}
support the expression
\tcode{*i},
resulting in a value of some object type
\tcode{T},
called the
\term{value type}
of the iterator.
All output iterators support the expression
\tcode{*i = o}
where
\tcode{o}
is a value of some type that is in the set of types that are
\term{writable}
to the particular iterator type of
\tcode{i}.
\removed{All iterators
\tcode{i}
for which the expression
\tcode{(*i).m}
is well-defined, support the expression
\tcode{i->m}
with the same semantics as
\tcode{(*i).m}.}
For every iterator type
\tcode{X}
for which
equality is defined, there is a corresponding signed integer type called the
\term{difference type}
of the iterator.

\pnum
Since iterators are an abstraction of pointers, their semantics is
a generalization of most of the semantics of pointers in \Cpp.
This ensures that every
function template
that takes iterators
works as well with regular pointers.
This International Standard defines
\changed{five}{seven} categories of iterators, according to the operations
defined on them:
\added{\techterm{weak input iterators}, }
\techterm{input iterators},
\added{\techterm{weak output iterators}, }
\techterm{output iterators},
\techterm{forward iterators},
\techterm{bidirectional iterators}
and
\techterm{random access iterators},
as shown in Table~\ref{tab:iterators.relations}.

\begin{floattable}{Relations among iterator categories}{tab:iterators.relations}
{lllll}
\topline
\textbf{Random Access}          &   $\rightarrow$ \textbf{Bidirectional}    &
$\rightarrow$ \textbf{Forward}  &   $\rightarrow$ \textbf{Input}            & \added{   $\rightarrow$ \textbf{WeakInput}}           \\
                        &   &   &   $\rightarrow$ \textbf{Output}           & \added{   $\rightarrow$ \textbf{WeakOutput}}          \\
\end{floattable}

\pnum
\changed{Forward}{Input} iterators satisfy all the requirements of \added{weak }input
iterators and can be used whenever \changed{an}{a weak} input iterator is specified;
\added{Forward iterators also satisfy all the requirements of
input iterators and can be used whenever an input iterator is specified;}
Bidirectional iterators also satisfy all the requirements of
forward iterators and can be used whenever a forward iterator is specified;
Random access iterators also satisfy all the requirements of bidirectional
iterators and can be used whenever a bidirectional iterator is specified.

\pnum
Iterators that further satisfy the requirements of \added{weak }output iterators are
called \defn{mutable iterator}{s}. Nonmutable iterators are referred to
as \defn{constant iterator}{s}.

\pnum
Just as a regular pointer to an array guarantees that there is a pointer value pointing past the last element
of the array, so for any iterator type there is an iterator value that points past the last element of a
corresponding sequence.
These values are called
\term{past-the-end}
values.
Values of an iterator
\tcode{i}
for which the expression
\tcode{*i}
is defined are called
\term{dereferenceable}.
The library never assumes that past-the-end values are dereferenceable.
Iterators can also have singular values that are not associated with any
sequence.
\enterexample
After the declaration of an uninitialized pointer
\tcode{x}
(as with
\tcode{int* x;}),
\tcode{x}
must always be assumed to have a singular value of a pointer.
\exitexample
Results of most expressions are undefined for singular values;
the only exceptions are destroying an iterator that holds a singular value,
the assignment of a non-singular value to
an iterator that holds a singular value, and\removed{, for iterators that satisfy the
\tcode{DefaultConstructible} requirements,} using a value-initialized iterator
as the source of a copy or move operation. \enternote This guarantee is not
offered for default initialization, although the distinction only matters for types
with trivial default constructors such as pointers or aggregates holding pointers.
\exitnote
In these cases the singular
value is overwritten the same way as any other value.
Dereferenceable
values are always non-singular.

\pnum
An iterator\added{ or sentinel}
\tcode{j}
is called
\term{reachable}
from an iterator
\tcode{i}
if and only if there is a finite sequence of applications of
the expression
\tcode{++i}
that makes
\tcode{i == j}.
If
\tcode{j}
is reachable from
\tcode{i},
they refer to elements of the same sequence.

\pnum
Most of the library's algorithmic templates that operate on data structures have interfaces that use ranges.
A
\term{range}
is a pair of iterators\added{ or an iterator and a sentinel} that designate the beginning and end of the computation.
A range \range{i}{i}
is an empty range;
in general, a range \range{i}{j}
refers to the elements in the data structure starting with the element
pointed to by
\tcode{i}
and up to but not including the element \changed{pointed to}{denoted} by
\tcode{j}.
Range \range{i}{j}
is valid if and only if
\tcode{j}
is reachable from
\tcode{i}.
The result of the application of functions in the library to invalid ranges is
undefined.

\added{
\pnum
A
\term{sentinel}
is an abstraction of a past-the-end iterator. Sentinels are Regular types that can be used to denote
the end of a range. A sentinel and an iterator denoting a range shall be EqualityComparable. A
sentinel denotes an element when an iterator
\tcode{i}
compares equal to the sentinel, and
\tcode{i}
points to that element.}

\pnum
All the categories of iterators require only those functions that are realizable for a given category in
constant time (amortized).
\removed{Therefore, requirement tables for the iterators do not have a complexity column.}

\pnum
Destruction of an iterator may invalidate pointers and references
previously obtained from that iterator.

\pnum
An
\techterm{invalid}
iterator is an iterator that may be singular.\footnote{This definition applies to pointers, since pointers are iterators.
The effect of dereferencing an iterator that has been invalidated
is undefined.
}

\pnum
In the following sections,
\tcode{a}
and
\tcode{b}
denote values of type
\tcode{X} or \tcode{const X},
\tcode{difference_type} and \tcode{reference} refer to the
types \tcode{\removed{iterator_traits<X>::difference_type}}\tcode{\added{DifferenceType<X>}} and
\tcode{\removed{iterator_traits<X>::ref\-erence}}~\tcode{\added{ReferenceType<X>}}, respectively,
\tcode{n}
denotes a value of
\tcode{difference_type},
\tcode{u},
\tcode{tmp},
and
\tcode{m}
denote identifiers,
\tcode{r}
denotes a value of
\tcode{X\&},
\tcode{t}
denotes a value of value type
\tcode{T},
\tcode{o}
denotes a value of some type that is writable to the \added{weak }output iterator.
\enternote For an iterator type \tcode{X} \changed{there must be an instantiation
of \tcode{iterator_traits<X>}~(\cxxref{iterator.traits})}{the type aliases
\tcode{DifferenceType<X>} and \tcode{ReferenceType<X>} must be well-formed}. \exitnote

\begin{addedblock}
\rSec2[readable.iterators]{Readable types}

\pnum
The \tcode{Readable} concept is modeled by types that are readable by applying \tcode{operator*}
including pointers, smart pointers, and iterators.

\indexlibrary{\idxcode{Readable}}%
\begin{codeblock}
  template <class I>
  concept bool Readable =
    Semiregular<I> &&
    requires (I i) {
      typename ValueType<I>;
      { *i } -> const ValueType<I>&; // pre: i is dereferencable
    };
\end{codeblock}

\pnum
A \tcode{Readable} type has an associated value type that can be accessed with the \tcode{ValueType}
alias template.

\indexlibrary{\idxcode{ValueType}}%
\begin{codeblock}
  template <class, class = void> struct value_type { };
  template <class T>
  struct value_type<T*>
    : enable_if<!is_void<T>::value, remove_cv_t<T>> { };
  template <class T>
  struct value_type<T[]> : remove_cv<T> { };
  template <class T, size_t N>
  struct value_type<T[N]> : remove_cv<T> { };
  template <class T>
  struct value_type<T, void_t<typename T::value_type>>
    : enable_if<!is_void<typename T::value_type>::value, typename T::value_type> { };
  template <class T>
  struct value_type<T, void_t<typename T::element_type>>
    : enable_if<!is_void<typename T::element_type>::value, typename T::element_type> { };

  template <class T>
  using ValueType = typename value_type<T>::type;
\end{codeblock}

\pnum
If a type \tcode{I} has an associated value type, then \tcode{value_type<I>::type} shall name the
value type. Otherwise, there shall be no nested type \tcode{type}.

\pnum
The \tcode{value_type} class template may be specialized on user-defined types.

\pnum
When instantiated with a type \tcode{I} that has a nested type \tcode{value_type},
\tcode{value_type<I>::type} names that type, unless it is \tcode{void} in which case
\tcode{value_type<I>} shall have no nested type \tcode{type}. \enternote Some legacy output
iterators define a nested type named \tcode{value_type} that is an alias for \tcode{void}. These
types are not \tcode{Readable} and have no associated value types.\exitnote

\pnum
When instantiated with a type \tcode{I} that has a nested type \tcode{element_type},
\tcode{value_type<I>::type} names that type, unless it is \tcode{void} in which case
\tcode{value_type<I>} shall have no nested type \tcode{type}. \enternote Smart pointers like
\tcode{shared_ptr<int>} are \tcode{Readable} and have an associated value type. But a smart pointer
like \tcode{shared_ptr<void>} is not \tcode{Readable} and has no associated value type.\exitnote

\rSec2[movewritable.iterators]{MoveWritable types}

\pnum
The \tcode{MoveWritable} concept describes a requirements for moving a value into an iterator's
referenced object.

\indexlibrary{\idxcode{MoveWritable}}%
\begin{codeblock}
  template <class Out, class T>
  concept bool MoveWritable =
    Semiregular<Out> &&
    requires (Out o, T value) {
      { *o = move(value) };
    };
\end{codeblock}

\pnum
Let \tcode{value} be an rvalue or a non-const lvalue of type \tcode{T}, and let \tcode{o} be an
object of type \tcode{Out}. Then types \tcode{T} and \tcode{Out} model \tcode{MoveWritable} if and
only if

\begin{itemize}
\item If \tcode{o} is dereferencable, then after the assignment \tcode{*o = move(value)}, the value
referred to by \tcode{*o} is equal to the value of \tcode{value} before the assignment.
\end{itemize}

\pnum
After the expression \tcode{*o = move(value)}, object \tcode{o} is not required to be dereferencable.

\pnum
\tcode{value}'s state is unspecified. \enternote \tcode{value} must still meet the
requirements of the library component that is using it. The operations listed
in those requirements must work as specified whether \tcode{value} has been moved
from or not.\exitnote

\pnum
\enternote
The only valid use of an \tcode{operator*} is on the left side of the assignment statement.
\textit{Assignment through the same value of the writable type happens only once.}
\exitnote

\rSec2[writable.iterators]{Writable types}

\pnum
The \tcode{Writable} concept describes a requirements for copying a value into an iterator's
referenced object.

\indexlibrary{\idxcode{Writable}}%
\begin{codeblock}
  template <class Out, class T>
  concept bool Writable =
    Semiregular<Out> &&
    MoveWritable<Out, T> &&
    requires (Out o, T value) {
      { *o = value };
    };
\end{codeblock}

\pnum
Let \tcode{value} be an lvalue of type (possibly \tcode{const}) \tcode{T} or an rvalue
of type \tcode{const T}, and let \tcode{o} be an object of type \tcode{Out}. Then types
\tcode{T} and \tcode{Out} model \tcode{Writable} if and only if

\begin{itemize}
\item After the assignment \tcode{*o = value}, the value referred to by \tcode{*o} is equal to
the value of \tcode{value} and \tcode{value} is unchanged.
\end{itemize}

\rSec2[indirectlymovable.iterators]{IndirectlyMovable types}

\pnum
The \tcode{IndirectlyMovable} concept describes the move relationship between a \tcode{Readable}
type and a \tcode{MoveWritable} type.

\indexlibrary{\idxcode{IndirectlyMovable}}%
\begin{codeblock}
  template <class I, class Out>
  concept bool IndirectlyMovable =
    Readable<I> &&
    Semiregular<Out> &&
    MoveWritable<Out, ValueType<I>>;
\end{codeblock}

\pnum
Let \tcode{i} be an object of type \tcode{I}, and let \tcode{o} be an object of type \tcode{Out}.
Then types \tcode{I} and \tcode{Out} model \tcode{IndirectlyMovable} if and only if

\begin{itemize}
\item After the assignment \tcode{*o = move(*i)}, the value referred to by \tcode{*o} is equal to
the value of \tcode{*i} before the assignment.
\end{itemize}

\rSec2[indirectlycopyable.iterators]{IndirectlyCopyable types}

\pnum
The \tcode{IndirectlyCopyable} concept describes the copy relationship between a \tcode{Readable}
type and a \tcode{Writable} type.

\indexlibrary{\idxcode{IndirectlyCopyable}}%
\begin{codeblock}
  template <class I, class Out>
  concept bool IndirectlyCopyable =
    Readable<I> &&
    Semiregular<I> &&
    IndirectlyMovable<I, Out> &&
    Writable<Out, ValueType<I>>;
\end{codeblock}

\pnum
Let \tcode{i} be an object of type \tcode{I}, and let \tcode{o} be an object of type \tcode{Out}.
Then types \tcode{I} and \tcode{Out} model \tcode{IndirectlyCopyable} if and only if

\begin{itemize}
\item After the assignment \tcode{*o = *i}, the value referred to by \tcode{*o} is equal to
the value of \tcode{*i} and the value of \tcode{*i} is unchanged.
\end{itemize}

\rSec2[indirectlycomparable.iterators]{IndirectlyComparable types}

\pnum
The \tcode{IndirectlyComparable} concept describes an relation between
two \tcode{Readable} types and (optionally) a relation and two projections.

\indexlibrary{\idxcode{IndirectlyComparable}}%
\begin{codeblock}
  template < class I1, class I2,
    class R = equal_to<>,
    class P1 = identity,
    class P2 = identity>
  concept bool IndirectlyComparable =
    Readable<I1> &&
    Readable<I2> &&
    RegularCallable<P1, ValueType<I1>> &&
    RegularCallable<P2, ValueType<I2>> &&
    CallableRelation<R,
      ResultType<FunctionType<P1>, ValueType<I1>>,
      ResultType<FunctionType<P2>, ValueType<I2>>>;
\end{codeblock}

\pnum
Let \tcode{i1} be an object of type \tcode{I1},
\tcode{i2} be an object of type \tcode{I2},
\tcode{r} be an object of type \tcode{R},
\tcode{p1} be an object of type \tcode{P1},
\tcode{p2} be an object of type \tcode{P2}.
Then types \tcode{I1}, \tcode{I2}, \tcode{R}, \tcode{P1}, and \tcode{P2} model
\tcode{IndirectlyComparable} if and only if

\begin{itemize}
\item \tcode{\textit{INVOKE}(r, \textit{INVOKE}(p1, *i1), \textit{INVOKE}(p2, *i2))}
is a relation.
\end{itemize}

\rSec2[indirectlyswappable.iterators]{IndirectlySwappable types}

\pnum
The \tcode{IndirectlySwappable} concept describes a swappable relationship between the
value types of two \tcode{Readable} types.

\indexlibrary{\idxcode{IndirectlySwappable}}%
\begin{codeblock}
  template <class I1, class I2 = I1>
  concept bool IndirectlySwappable =
    Readable<I1> && 
    Readable<I2> &&
    Swappable<ValueType<I1>, ValueType<I2>> &&
    Swappable<ValueType<I1>> &&
    Swappable<ValueType<I2>>;
\end{codeblock}

\rSec2[weaklyincrementable.iterators]{WeaklyIncrementable types}

The \tcode{WeaklyIncrementable} concept describes types that can be incremented with the pre-
and post-increment operators. The increment operations are not required to be equality-preserving,
nor is the type required to be \tcode{EqualityComparable}.

\indexlibrary{\idxcode{WeaklyIncrementable}}%
\begin{codeblock}
  template <class I>
  concept bool WeaklyIncrementable =
    Semiregular<I> &&
    requires (I i) {
      typename DifferenceType<I>;
      requires SignedIntegral<DifferenceType<I>>;
      { ++i };
      requires Same<I&, decltype(++i)>;
      { i++ };
    };
\end{codeblock}

\pnum
Not all arguments will be incrementable for a given type. For example $NaN$ is not a well-formed
floating point value and hence is not incrementable. Likewise, past-the-end iterators are also not
incrementable. This does not mean that the type does not model \tcode{WeaklyIncrementable}.

\pnum
Let \tcode{i} be an object of type \tcode{I}.
Then the type \tcode{I} models \tcode{WeaklyIncrementable} if and only if

\begin{itemize}
\item If \tcode{i} is incrementable, then \tcode{++i} moves \tcode{i} to the next element.
\item If \tcode{i} is incrementable, then \tcode{(\&++i == \&i) != false}.
\item If \tcode{i} is incrementable, then \tcode{i++} moves \tcode{i} to the next element.
\item \tcode{++i} is valid if and only if \tcode{i++} is valid.
\end{itemize}

\ednote{Copied almost verbatim from the InputIterator description. Remove this wording there.}

\pnum
\enternote For \tcode{WeaklyIncrementable} types, \tcode{a} equals \tcode{b} does imply that \tcode{++a}
equals \tcode{++b}. (Equality does not guarantee the substitution property or referential
transparency.) Algorithms on weakly incrementable types should never attempt to pass
through the same incrementable value twice. They should be single pass algorithms. These algorithms
can be used with istreams as the source of the input data through the \tcode{istream_iterator} class
template.\exitnote

\rSec2[incrementable.iterators]{Incrementable types}

The \tcode{Incrementable} concept describes types that can be incremented with the pre-
and post-increment operators. The increment operations are required to be equality-preserving,
and the type is required to be \tcode{EqualityComparable}.

\indexlibrary{\idxcode{Incrementable}}%
\begin{codeblock}
  template <class I>
  concept bool Incrementable =
    Regular<I> &&
    WeaklyIncrementable<I> &&
    Same<I, decltype(i++)>;
\end{codeblock}

\pnum
Let \tcode{a} and \tcode{b} be incrementable objects of type \tcode{I}.
Then the type \tcode{I} models \tcode{Incrementable} if and only if

\begin{itemize}
\item If \tcode{(a == b) != false} then \tcode{(++a == ++b) != false}.
\item If \tcode{(a == b) != false} then \tcode{(a++, a) == (b++, b) != false}.
\item If \tcode{(a == b) != false} then \tcode{(a++ == b) != false}.
\item If \tcode{(a == b) != false} then \tcode{((a++, a) == ++b) != false}.
\end{itemize}

\ednote{Copied in part from the ForwardIterator description. Remove this wording there.}

\pnum
\enternote The requirement that \tcode{a} equals \tcode{b} implies \tcode{++a} equals \tcode{++b}
(which is not true for weakly incrementable types) allows the use of multi-pass one-directional
algorithms with types that model \tcode{Incrementable}.\exitnote

\end{addedblock}

\rSec2[weakiterator.iterators]{Weak iterators}

\pnum
The \added{\tcode{Weak}}\tcode{Iterator} \changed{requirements}{concept} form\added{s}
the basis of the iterator concept taxonomy; every iterator satisfies the
\added{\tcode{Weak}}\tcode{Iterator} requirements. This
\changed{set of requirements}{concept} specifies operations for dereferencing and incrementing
an iterator. Most algorithms will require additional operations
\added{to compare iterators~(\ref{iterator.iterators}),} to
read~(\ref{input.iterators}) or write~(\ref{output.iterators}) values, or
to provide a richer set of iterator movements~(\ref{forward.iterators},
\ref{bidirectional.iterators}, \ref{random.access.iterators}).)

\ednote{Remove para 2 and Table 106.}

\begin{addedblock}
\indexlibrary{\idxcode{WeakIterator}}%
\begin{codeblock}
  template <class I>
  concept bool WeakIterator =
    WeaklyIncrementable<I>
    requires(I i) {
      { *i } -> auto&&; // pre: i is dereferenceable
    };
\end{codeblock}

\enternote The requirement that the result of dereferencing the iterator is deducable from
\tcode{auto\&\&} effectively means that it cannot be \tcode{void}.\exitnote
\end{addedblock}

\begin{addedblock}
\rSec2[iterator.iterators]{Iterators}

\pnum
The \tcode{Iterator} concept refines \tcode{WeakIterator}~(\ref{weakiterator.iterators}) and adds
the requirement that the iterator is equality comparable.

\pnum
In the \tcode{Iterator} concept, the set of values over which
\tcode{==} is (required to be) defined can change over time.
Each algorithm places additional requirements on the domain of
\tcode{==} for the iterator values it uses.
These requirements can be inferred from the uses that algorithm
makes of \tcode{==} and \tcode{!=}.
\enterexample
the call \tcode{find(a,b,x)}
is defined only if the value of \tcode{a}
has the property \textit{p}
defined as follows:
\tcode{b} has property \textit{p}
and a value \tcode{i}
has property \textit{p}
if
\tcode{(*i==x)}
or if
\tcode{(*i!=x}
and
\tcode{++i}
has property
\tcode{p}).
\exitexample

\indexlibrary{\idxcode{Iterator}}%
\begin{codeblock}
  template <class I>
  concept bool Iterator =
    WeakIterator<I> &&
    EqualityComparable<I>();
\end{codeblock}

\rSec2[sentinel.iterators]{Sentinels}

The \tcode{Sentinel} concept defines requirements for a type that
is an abstraction of the past-the-end iterator. Its values can be
compared to an iterator for equality.

\indexlibrary{\idxcode{Sentinel}}%
\begin{codeblock}
  template <class T, class I>
  concept bool Sentinel =
    Regular<T> &&
    Iterator<I> &&
    EqualityComparable<T, I>();
\end{codeblock}

\rSec2[weakinput.iterators]{Weak input iterators}

\pnum
The \tcode{WeakInputIterator} concept is a refinement of
\tcode{WeakIterator}~(\ref{weakiterator.iterators}). It
defines requirements for a type whose referred to values can be read (from the requirement for
\tcode{Readable}~(\ref{readable.iterators})) and which can be both pre- and post-incremented. However,
weak input iterators are not required to be comparable for equality.

\indexlibrary{\idxcode{WeakInputIterator}}%
\begin{codeblock}
  template <class I>
  concept bool WeakInputIterator =
    WeakIterator<I> &&
    Readable<I> &&
    requires(I i) {
      typename IteratorCategory<I>;
      { i++ } -> Readable;
      requires Derived<IteratorCategory<I>, weak_input_iterator_tag>;
    };
\end{codeblock}
\end{addedblock}

\rSec2[input.iterators]{Input iterators}

\ednote{Remove para 1, 2 and Table 107}

\begin{addedblock}
\pnum
The \tcode{InputIterator} concept is a refinement of \tcode{Iterator}~(\ref{iterator.iterators}) and
\tcode{WeakInputIterator}~(\ref{weakinput.iterators}).

\indexlibrary{\idxcode{InputIterator}}%
\begin{codeblock}
  template <class I>
  concept bool InputIterator =
    WeakInputIterator<I> &&
    Iterator<I> &&
    Derived<IteratorCategory<I>, input_iterator_tag>;
\end{codeblock}

\end{addedblock}

\pnum
\enternote
For input iterators,
\tcode{a == b}
does not imply
\tcode{++a == ++b}.
(Equality does not guarantee the substitution property or referential transparency.)
Algorithms on input iterators should never attempt to pass through the same iterator twice.
They should be
\term{single pass}
algorithms.
Value type T is not required to be a \tcode{CopyAssignable} type (\removed{Table}~\ref{concepts.lib.copyassignable}).
These algorithms can be used with istreams as the source of the input data through the
\tcode{istream_iterator}
class template.
\exitnote

\ednote{Section Output iterators renamed to Weak output iterators below:}

\rSec2[weakoutput.iterators]{Weak output iterators}

\ednote{Remove para 1 and Table 108}

\begin{addedblock}
\pnum
The \tcode{WeakOutputIterator} concept is a refinement of
\tcode{WeakIterator}~(\ref{weakiterator.iterators}). It defines requirements for a type that
can be used to write values (from the requirement for
\tcode{Writable}~(\ref{writable.iterators})) and which can be both pre- and post-incremented.
However, weak output iterators are not required to be comparable for equality.

\indexlibrary{\idxcode{WeakOutputIterator}}%
\begin{codeblock}
  template <class I, class T>
  concept bool WeakOutputIterator =
    WeakIterator<I> && Writable<I, T>;
\end{codeblock}
\end{addedblock}

\pnum
\enternote
\removed{The only valid use of an
\tcode{operator*}
is on the left side of the assignment statement.}
\textit{\removed{Assignment through the same value of the iterator happens only once.}}
Algorithms on output iterators should never attempt to pass through the same iterator twice.
They should be
\term{single pass}
algorithms.
\removed{Equality and inequality might not be defined.}
Algorithms that take \added{weak }output iterators can be used with ostreams as the destination
for placing data through the
\tcode{ostream_iterator}
class as well as with insert iterators and insert pointers.
\exitnote

\begin{addedblock}
\rSec2[output.iterators]{Output iterators}

\pnum
The \tcode{OutputIterator} concept is a refinement of \tcode{Iterator}~(\ref{iterator.iterators}) and
\tcode{WeakOutputIterator}~(\ref{weakoutput.iterators}).

\indexlibrary{\idxcode{OutputIterator}}%
\begin{codeblock}
  template <class I, class T>
  concept bool OutputIterator =
    WeakOutputIterator<I, T> && Iterator<I>;
\end{codeblock}

\pnum
\enternote Output iterators are used by single-pass
algoritms that write into a bounded range, like \tcode{generate}.
\exitnote

\end{addedblock}

\rSec2[forward.iterators]{Forward iterators}

\begin{removedblock}
\pnum
A class or pointer type
\tcode{X}
satisfies the requirements of a forward iterator if

\begin{itemize}
\item \tcode{X} satisfies the requirements of an input iterator~(\ref{input.iterators}),

\item X satisfies the \tcode{DefaultConstructible}
requirements~(\cxxref{utility.arg.requirements}),

\item if \tcode{X} is a mutable iterator, \tcode{reference} is a reference to \tcode{T};
if \tcode{X} is a const iterator, \tcode{reference} is a reference to \tcode{const T},

\item the expressions in Table~\cxxref{tab:iterator.forward.requirements}
are valid and have the indicated semantics, and

\item objects of type \tcode{X} offer the multi-pass guarantee, described below.
\end{itemize}
\end{removedblock}

\begin{addedblock}
\pnum
The \tcode{ForwardIterator} concept refines \tcode{InputIterator}~(\ref{input.iterators})
and adds the multi-pass guarantee, described below.

\indexlibrary{\idxcode{ForwardIterator}}%
\begin{codeblock}
  template <class I>
  concept bool ForwardIterator =
    InputIterator<I> &&
    Incrementable<I> &&
    Derived<IteratorCategory<I>, forward_iterator_tag>;
\end{codeblock}
\end{addedblock}

\pnum
The domain of \tcode{==} for forward iterators is that of iterators over the same
underlying sequence. However, value-initialized iterators may be compared and
shall compare equal to other value-initialized iterators of the same type.
\enternote value initialized iterators behave as if they refer past the end of
the same empty sequence \exitnote

\pnum
Two dereferenceable iterators \tcode{a} and \tcode{b} of type \tcode{X} offer the
\defn{multi-pass guarantee} if:

\begin{itemize}
\item \tcode{a == b} implies \tcode{++a == ++b} and
\item \tcode{X} is a pointer type or the expression
\tcode{(void)++X(a), *a} is equivalent to the expression \tcode{*a}.
\end{itemize}

\pnum
\enternote
The requirement that
\tcode{a == b}
implies
\tcode{++a == ++b}
(which is not true for input and output iterators)
and the removal of the restrictions on the number of the assignments through
a mutable iterator
(which applies to output iterators)
allows the use of multi-pass one-directional algorithms with forward iterators.
\exitnote

\ednote{Remove Table 109}

\pnum
If \tcode{a} and \tcode{b} are equal, then either \tcode{a} and \tcode{b}
are both dereferenceable
or else neither is dereferenceable.

\pnum
If \tcode{a} and \tcode{b} are both dereferenceable, then \tcode{a == b}
if and only if
\tcode{*a} and \tcode{*b} are bound to the same object.

\rSec2[bidirectional.iterators]{Bidirectional iterators}

\begin{removedblock}
\pnum
A class or pointer type
\tcode{X}
satisfies the requirements of a bidirectional iterator if,
in addition to satisfying the requirements for forward iterators,
the following expressions are valid as shown in Table~\cxxref{tab:iterator.bidirectional.requirements}.
\end{removedblock}

\begin{addedblock}
\pnum
The \tcode{BidirectionalIterator} concept refines \tcode{ForwardIterator}~(\ref{forward.iterators}),
and adds the ability to move an iterator backward as well as forward.

\indexlibrary{\idxcode{BidirectionalIterator}}%
\begin{codeblock}
  template <class I>
  concept bool BidirectionalIterator =
    ForwardIterator<I> &&
    Derived<IteratorCategory<I>, bidirectional_iterator_tag> &&
    requires (I i, I j) {
      { --i };
      requires Same<I&, decltype(--i)>;
      { i-- };
      requires Same<I, decltype(i--)>;
    };
\end{codeblock}
\end{addedblock}

\ednote{Remove table 110}

\begin{addedblock}
\pnum
A bidirectional iterator \tcode{r} is decrementable only if there exists some \tcode{s} such that
\tcode{++s == r}. The expressions \tcode{\dcr{}r} and \tcode{r\dcr{}} are only valid if \tcode{r} is
decrementable.

\pnum
Let \tcode{a} and \tcode{b} be decrementable objects of type \tcode{I}. Then \tcode{I} models
\tcode{BidirectionalIterator} if and only if:

\begin{itemize}
\item \tcode{\&\dcr{}a == \&a}.
\item If \tcode{(a == b) != false}, then \tcode{(a\dcr{} == j) != false}.
\item If \tcode{(a == b) != false}, then \tcode{((a\dcr{}, a) == \dcr{}j) != false}.
\item If \tcode{a} is incrementable and \tcode{(a == b) != false}, then
      \tcode{(\dcr{}(++a) == j) != false}.
\item If \tcode{(a == b) != false}, then \tcode{(++(\dcr{}a) == j) != false}.
\end{itemize}
\end{addedblock}

\begin{removedblock}
\pnum
\enternote
Bidirectional iterators allow algorithms to move iterators backward as well as forward.
\exitnote
\end{removedblock}

\rSec2[random.access.iterators]{Random access iterators}

\begin{removedblock}
\pnum
A class or pointer type
\tcode{X}
satisfies the requirements of a random access iterator if,
in addition to satisfying the requirements for bidirectional iterators,
the following expressions are valid as shown in Table~\cxxref{tab:iterator.random.access.requirements}.
\end{removedblock}

\begin{addedblock}
The \tcode{RandomAccessIterator} concept refines \tcode{BidirectionalIterator}~(\ref{bidirectional.iterators})
and adds support for constant-time advancement with \tcode{+=}, \tcode{+}, and \tcode{-=}, and the
computation of distance in constant time with \tcode{-}. Random access iterators also support array
notation via subscripting.

\indexlibrary{\idxcode{RandomAccessIterator}}%
\begin{codeblock}
  template <class I>
  concept bool RandomAccessIterator =
    BidirectionalIterator<I> &&
    TotallyOrdered<I>() &&
    Derived<IteratorCategory<I>, random_access_iterator_tag> &&
    SizedIteratorRange<I, I> && // see below
    requires (I i, I j, DifferenceType<I> n) {
      { i += n };
      { i + n };
      { n + i };
      { i -= n };
      { i - n };
      requires Same<decltype(i += n), I&>;
      requires Same<decltype(i + n), I>;
      requires Same<decltype(n + i), I>;
      requires Same<decltype(i -= n), I&>;
      requires Same<decltype(i - n), I>;
      { i[n] } -> const ValueType<I>&;
    };
\end{codeblock}
\end{addedblock}

\ednote{Remove Table 111}

\begin{addedblock}
\pnum
Let \tcode{a} and \tcode{b} be valid iterators of type \tcode{I} such that \tcode{b} is reachable
from \tcode{a}. Let \tcode{n} be an object of type \tcode{DifferenceType<I>} such that after
\tcode{n} applications of \tcode{++a}, \tcode{(a == b) != false}. Then \tcode{I} models
\tcode{RandomAccessIterator} if and only if:

\begin{itemize}
\item \tcode{(a += n) == b}.
\item \tcode{\&(a += n) == \&a}.
\item \tcode{(a + n) == (a += n)}.
\item For any two positive integers \tcode{x} and \tcode{y}, if \tcode{a + (x + y)} is valid, then
\tcode{a + (x + y) == (a + x) + y}.
\item \tcode{a + 0 == a}.
\item If \tcode{(a + (n - 1))} is valid, then \tcode{a + n == ++(a + (n - 1))}.
\item \tcode{(b += -n) == a}.
\item \tcode{(b -= n) == a}.
\item \tcode{\&(b -= n) == \&b}.
\item \tcode{(b - n) == (b -= n)}.
\item If \tcode{b} is dereferenceable, then \tcode{a[n]} is valid and is equal to \tcode{*b}.
\end{itemize}

\rSec1[indirectfunctions]{Indirect function requirements}

\rSec2[indirectfunctions.general]{In general}

\pnum
There are several concepts that group requirements of the higher-order algorithms; that is,
algorithms that take functions as arguments.

\ednote{Specifying the algorithms in terms of these indirect callable concepts would ease
the transition should we ever decide to support proxy iterators in the future. See the
Future Work appendix~(\ref{future}).}

\rSec2[projected.indirectfunctions]{Projected iterator}

\pnum
The \tcode{Projected} class template is intended for use when specifying the constraints of
higher-order algorithms that accept optional projections. It bundles a \tcode{Readable} type
\tcode{I} and a projection function \tcode{Proj} into a new \tcode{Readable} type whose
\tcode{reference} type is the result of applying \tcode{Proj} to the \tcode{reference} type
of \tcode{I}.

\indexlibrary{\idxcode{Projected}}%
\begin{codeblock}
  template <Readable I, RegularCallable<ValueType<I>> Proj>
  struct Projected {
    using value_type = decay_t<ResultType<FunctionType<Proj>, ValueType<I>>>;
    using reference = ResultType<FunctionType<Proj>, decltype(*std::declval<I>())>;
    using pointer = add_pointer_t<reference>;
    reference operator*() const;
  };
\end{codeblock}

\pnum
\enternote \tcode{Projected} is only used to ease constraints specification. Its
member functions need not be defined.\exitnote

\rSec2[indirectfunc.indirectfuncs]{Indirect functions}

\pnum
The indirect callable concepts are used to constrain those algorithms that accept
functions as arguments.

\indexlibrary{\idxcode{IndirectCallable}}%
\indexlibrary{\idxcode{IndirectRegularCallable}}%
\indexlibrary{\idxcode{IndirectCallablePredicate}}%
\indexlibrary{\idxcode{IndirectCallableRelation}}%
\begin{codeblock}
  template <class F, class...Is>
  concept bool IndirectCallable =
    (Readable<Is> && ...) &&
    Callable<F, ValueType<Is>...>;

  template <class F, class...Is>
  concept bool IndirectRegularCallable =
    (Readable<Is> && ...) &&
    RegularCallable<F, ValueType<Is>...>;

  template <class F, class...Is>
  concept bool IndirectCallablePredicate =
    (Readable<Is> && ...) &&
    CallablePredicate<F, ValueType<Is>...>;

  template <class F, class I1, class I2 = I1>
  concept bool IndirectCallableRelation =
    Readable<I1> &&
    Readable<I2> &&
    CallableRelation<F, ValueType<I1>, ValueType<I2>>();
\end{codeblock}

\pnum
\ednote{TODO description here}

\rSec1[rearrangements]{Rearrangement requirements}

\rSec2[rearrangement.general]{In general}

\pnum
There are several additional iterator concepts that are commonly applied to families of algorithms.
These are the so-called \techterm{rearrangement} concepts. They group together iterator requirements
of algorithm families. There are 3 relational concepts for rearrangements: \tcode{Permutable},
\tcode{Mergeable}, and \tcode{Sortable}.

\rSec2[permutable.rearrangements]{Permutable iterators}

\pnum
The \tcode{Permutable} concept specifies the common requirements of algorithms that reorder
elements in place by moving or swapping them.

\indexlibrary{\idxcode{Permutable}}%
\begin{codeblock}
  template <class I>
  concept bool Permutable =
    ForwardIterator<I> &&
    Semiregular<ValueType<I>> &&
    IndirectlyMovable<I>;
\end{codeblock}

\ednote{\tcode{Semiregular} here overconstrains by adding a default-constructibility requirement.
See Appendix D of the ``The Palo Alto'' report for an alternate design.}

\rSec2[mergeable.rearrangements]{Mergeable iterators}

\pnum
The \tcode{Mergeable} concept describes the requirements of algorithms that merge sorted sequences
into an output sequence.

\indexlibrary{\idxcode{Permutable}}%
\begin{codeblock}
  template <class I1, class I2, class Out,
      class R = less<>, class P1 = identity, class P2 = identity>
  concept bool Mergeable =
    InputIterator<I1> &&
    InputIterator<I2> &&
    WeaklyIncrementable<I2> &&
    IndirectlyCopyable<I1, Out> &&
    IndirectlyCopyable<I2, Out> &&
    IndirectlyComparable<I1, I2, R, P1, P2>;
\end{codeblock}

\pnum
\enternote When \tcode{less<>} is used as the
relation, the value type must model \tcode{TotallyOrdered}.\exitnote

\rSec2[sortable.rearrangements]{Sortable iterators}

\pnum
The \tcode{Sortable} concept describes the common requirements of algorithms that permute sequences
of iterators into an ordered sequence (e.g., \tcode{sort}).

\indexlibrary{\idxcode{Permutable}}%
\begin{codeblock}
  template <class I, class R = less<>, class P = identity>
  concept bool Sortable =
    ForwardIterator<I> &&
    Permutable<I> &&
    IndirectCallableRelation<R, Projected<I, P>>;
\end{codeblock}

\pnum
\enternote When \tcode{less<>} is used as the
relation, the value type must model \tcode{TotallyOrdered}.\exitnote

\rSec1[iteratorranges]{Iterator range requirements}

\rSec2[iteratorrange.iteratorranges]{Iterator range}

The \tcode{IteratorRange} concept defines a pair of types (an
\tcode{Iterator}~(\ref{iterator.iterators}) and a \tcode{Sentinel}), that can be compared for
equality. This concept is the key that allows iterator ranges to be defined by pairs of types
that are not the same.

\indexlibrary{\idxcode{IteratorRange}}%
\begin{codeblock}
  template <class I, class S>
  concept bool IteratorRange =
    Iterator<I> &&
    Regular<S> &&
    EqualityComparable<I, S>();
\end{codeblock}

\pnum
Let {a} be a valid iterator of type \tcode{I} and let \tcode{b} be a valid sentinel of type
\tcode{S}. Then \tcode{I} and \tcode{S} model concept \tcode{IteratorRange} if and only if
\tcode{b} is reachable from \tcode{a}.

\rSec2[sizediteratorrange.iteratorranges]{Sized Iterator range}

The \tcode{SizedIteratorRange} concept refines \tcode{IteratorRange}~(\ref{iteratorrange.iteratorranges})
and allows the use of the \tcode{-} operator to compute the distance
between an \tcode{Iterator}~(\ref{iterator.iterators}) and a \tcode{Sentinel} in constant time.

\indexlibrary{\idxcode{SizedIteratorRange}}%
\begin{codeblock}
  template <class I, class S>
  concept bool SizedIteratorRange =
    Iterator<I> &&
    Regular<S> &&
    IteratorRange<I, S> &&
    requires (I i, S j) {
      { i - i } -> DifferenceType<I>;
      { j - j } -> DifferenceType<I>;
      { i - j } -> DifferenceType<I>;
      { j - i } -> DifferenceType<I>;
    };
\end{codeblock}

\pnum
Let \tcode{a} be a valid iterator of type \tcode{I} and \tcode{b} be a valid sentinel of type
\tcode{S}. Let \tcode{n} be an object of type \tcode{DifferenceType<I>} such that after
\tcode{n} applications of \tcode{++a}, \tcode{(a == b) != false}. Then types \tcode{I} and
\tcode{S} model \tcode{SizedIteratorRange} if and only if:

\begin{itemize}
\item \tcode{(b - a) == n}.
\item \tcode{(a - b) == -n}.
\end{itemize}

\enternote The \tcode{SizedIteratorRange} concept is modeled by pairs of
\tcode{RandomAccessIterator}s(~\ref{random.access.iterators}) and by counted iterators and their
sentinels~(\ref{counted.iterator}).\exitnote

\end{addedblock}

\rSec1[iterator.synopsis]{Header \tcode{<iterator>}\ synopsis}

\indexlibrary{\idxhdr{iterator}}%
\begin{codeblock}
namespace std {
  // \ref{iterator.primitives}, primitives:
  template<class Iterator> @\changed{struct}{using}@ iterator_traits@\added{ = \seebelow}@;
  @\removed{template<class T> struct iterator_traits<T*>;}@

  template<class Category, class T, class Distance = ptrdiff_t,
       class Pointer = T*, class Reference = T&> struct iterator;
\end{codeblock}

\begin{addedblock}
\begin{codeblock}
  template <class, class = void> struct difference_type;
  template <class, class = void> struct value_type;
  template <class, class = void> struct iterator_category;
  template <class WeaklyIncrementable> using DifferenceType
    = typename difference_type<WeaklyIncrementable>::type;
  template <class Readable> using ValueType
    = typename value_type<Readable>::type;
  template <class WeakInputIterator> using IteratorCategory
    = typename iterator_category<WeakInputIterator>::type;

\end{codeblock}
\end{addedblock}
\begin{codeblock}
  @\added{struct weak_input_iterator_tag \{ \};}@
  struct input_iterator_tag @\added{: public weak_input_iterator_tag }@{ };
  struct output_iterator_tag { };
  struct forward_iterator_tag: public input_iterator_tag { };
  struct bidirectional_iterator_tag: public forward_iterator_tag { };
  struct random_access_iterator_tag: public bidirectional_iterator_tag { };

  // \ref{iterator.operations}, iterator operations:
\end{codeblock}
\begin{removedblock}
\begin{codeblock}
  template <class InputIterator, class Distance>
    void advance(InputIterator& i, Distance n);
  template <class InputIterator>
    typename iterator_traits<InputIterator>::difference_type
    distance(InputIterator first, InputIterator last);
  template <class ForwardIterator>
    ForwardIterator next(ForwardIterator x,
      typename std::iterator_traits<ForwardIterator>::difference_type n = 1);
  template <class BidirectionalIterator>
    BidirectionalIterator prev(BidirectionalIterator x,
      typename std::iterator_traits<BidirectionalIterator>::difference_type n = 1);
\end{codeblock}
\end{removedblock}
\begin{addedblock}
\begin{codeblock}
  template <WeakIterator I>
    void advance(I& i, DifferenceType<I> n);
  template <Iterator I, Sentinel<I> S>
    void advance(I& i, S bound);
  template <Iterator I, Sentinel<I> S>
    DifferenceType<I> advance(I& i, DifferenceType<I> n, S bound);
  template <Iterator I, Sentinel<I> S>
    DifferenceType<I> distance(I first, S last);
  template <WeakIterator I>
    I next(I x, DifferenceType<I> n = 1);
  template <Iterator I, Sentinel<I> S>
    I next(I x, S bound);
  template <Iterator I, Sentinel<I> S>
    I next(I x, DifferenceType<I> n, S bound);
  template <BidirectionalIterator I>
    I prev(I x, DifferenceType<I> n = 1);
  template <BidirectionalIterator I>
    I prev(I x, DifferenceType<I> n, I bound);
\end{codeblock}
\end{addedblock}

\begin{codeblock}
  // \ref{predef.iterators}, predefined iterators\added{ and sentinels}:

  @\added{\itshape{\rmfamily{// \ref{reverse.iterators} Reverse iterators}}}@
  template <@\changed{class Iterator}{BidirectionalIterator I}@> class reverse_iterator;

  template <@\changed{class Iterator1}{BidirectionalIterator I1}@, @\changed{class Iterator2}{BidirectionalIterator I2}@>
      @\added{requires EqualityComparable<I1, I2>()}@
    bool operator==(
      const reverse_iterator<@\changed{Iterator1}{I1}@>& x,
      const reverse_iterator<@\changed{Iterator2}{I2}@>& y);
  template <@\changed{class Iterator1}{RandomAccessIterator I1}@, @\changed{class Iterator2}{RandomAccessIterator I2}@>
      @\added{requires TotallyOrdered<I1, I2>()}@
    bool operator<(
      const reverse_iterator<@\changed{Iterator1}{I1}@>& x,
      const reverse_iterator<@\changed{Iterator2}{I2}@>& y);
  template <@\changed{class Iterator1}{BidirectionalIterator I1}@, @\changed{class Iterator2}{BidirectionalIterator I2}@>
      @\added{requires EqualityComparable<I1, I2>()}@
    bool operator!=(
      const reverse_iterator<@\changed{Iterator1}{I1}@>& x,
      const reverse_iterator<@\changed{Iterator2}{I2}@>& y);
  template <@\changed{class Iterator1}{RandomAccessIterator I1}@, @\changed{class Iterator2}{RandomAccessIterator I2}@>
      @\added{requires TotallyOrdered<I1, I2>()}@
    bool operator>(
      const reverse_iterator<@\changed{Iterator1}{I1}@>& x,
      const reverse_iterator<@\changed{Iterator2}{I2}@>& y);
  template <@\changed{class Iterator1}{RandomAccessIterator I1}@, @\changed{class Iterator2}{RandomAccessIterator I2}@>
      @\added{requires TotallyOrdered<I1, I2>()}@
    bool operator>=(
      const reverse_iterator<@\changed{Iterator1}{I1}@>& x,
      const reverse_iterator<@\changed{Iterator2}{I2}@>& y);
  template <@\changed{class Iterator1}{RandomAccessIterator I1}@, @\changed{class Iterator2}{RandomAccessIterator I2}@>
      @\added{requires TotallyOrdered<I1, I2>()}@
    bool operator<=(
      const reverse_iterator<@\changed{Iterator1}{I1}@>& x,
      const reverse_iterator<@\changed{Iterator2}{I2}@>& y);

  template <@\changed{class Iterator1}{BidirectionalIterator I1}@, @\changed{class Iterator2}{BidirectionalIterator I2}@>
      @\added{requires SizedIteratorRange<I2, I1>}@
    @\changed{auto}{DifferenceType<I2>}@ operator-(
      const reverse_iterator<@\changed{Iterator1}{I1}@>& x,
      const reverse_iterator<@\changed{Iterator2}{I2}@>& y)@\removed{ ->decltype(y.base() - x.base())}@;
  template <@\changed{class Iterator}{RandomAccessIterator I}@>
    reverse_iterator<@\changed{Iterator}{I}@>
      operator+(
    @\changed{typename reverse_iterator<Iterator>::difference_type}{DifferenceType<I>}@ n,
    const reverse_iterator<@\changed{Iterator}{I}@>& x);

  template <@\changed{class Iterator}{BidirectionalIterator I}@>
    reverse_iterator<@\changed{Iterator}{I}@> make_reverse_iterator(@\changed{Iterator}{I}@ i);

  @\added{\itshape{\rmfamily{// \ref{insert.iterators} Insert iterators}}}@
  template <class Container> class back_insert_iterator;
  template <class Container>
    back_insert_iterator<Container> back_inserter(Container& x);

  template <class Container> class front_insert_iterator;
  template <class Container>
    front_insert_iterator<Container> front_inserter(Container& x);

  template <class Container> class insert_iterator;
  template <class Container>
    insert_iterator<Container> inserter(Container& x, @\removed{typename Container::iterator}@
      @\added{IteratorType<Container>}@ i);

  @\added{\itshape{\rmfamily{// \ref{move.iterators} Move iterators}}}@
  template <@\changed{class Iterator}{WeakInputIterator I}@> class move_iterator;
  template <@\changed{class Iterator1}{InputIterator I1}@, @\changed{class Iterator2}{InputIterator I2}@>
      @\added{requires EqualityComparable<I1, I2>()}@
    bool operator==(
      const move_iterator<@\changed{Iterator1}{I1}@>& x, const move_iterator<@\changed{Iterator2}{I2}@>& y);
  template <@\changed{class Iterator1}{InputIterator I1}@, @\changed{class Iterator2}{InputIterator I2}@>
      @\added{requires EqualityComparable<I1, I2>()}@
    bool operator!=(
      const move_iterator<@\changed{Iterator1}{I1}@>& x, const move_iterator<@\changed{Iterator2}{I2}@>& y);
  template <@\changed{class Iterator1}{RandomAccessIterator I1}@, @\changed{class Iterator2}{RandomAccessIterator I2}@>
      @\added{requires TotallyOrdered<I1, I2>()}@
    bool operator<(
      const move_iterator<@\changed{Iterator1}{I1}@>& x, const move_iterator<@\changed{Iterator2}{I2}@>& y);
  template <@\changed{class Iterator1}{RandomAccessIterator I1}@, @\changed{class Iterator2}{RandomAccessIterator I2}@>
      @\added{requires TotallyOrdered<I1, I2>()}@
    bool operator<=(
      const move_iterator<@\changed{Iterator1}{I1}@>& x, const move_iterator<@\changed{Iterator2}{I2}@>& y);
  template <@\changed{class Iterator1}{RandomAccessIterator I1}@, @\changed{class Iterator2}{RandomAccessIterator I2}@>
      @\added{requires TotallyOrdered<I1, I2>()}@
    bool operator>(
      const move_iterator<@\changed{Iterator1}{I1}@>& x, const move_iterator<@\changed{Iterator2}{I2}@>& y);
  template <@\changed{class Iterator1}{RandomAccessIterator I1}@, @\changed{class Iterator2}{RandomAccessIterator I2}@>
      @\added{requires TotallyOrdered<I1, I2>()}@
    bool operator>=(
      const move_iterator<@\changed{Iterator1}{I1}@>& x, const move_iterator<@\changed{Iterator2}{I2}@>& y);

  template <@\changed{class Iterator1}{WeakInputIterator I1}@, @\changed{class Iterator2}{WeakInputIterator I2}@>
      @\added{requires SizedIteratorRange<I2, I1>}@
    @\changed{auto}{DifferenceType<I2>}@ operator-(
      const move_iterator<@\changed{Iterator1}{I1}@>& x,
      const move_iterator<@\changed{Iterator2}{I2}@>& y)@\removed{ ->decltype(y.base() - x.base())}@;
  template <@\changed{class Iterator}{RandomAccessIterator I}@>
    move_iterator<@\changed{Iterator}{I}@>
      operator+(
    @\changed{typename move_iterator<Iterator>::difference_type}{DifferenceType<I>}@ n,
    const move_iterator<@\changed{Iterator}{I}@>& x);
  template <@\changed{class Iterator}{WeakInputIterator I}@>
    move_iterator<@\changed{Iterator}{I}@> make_move_iterator(@\changed{Iterator}{I}@ i);
\end{codeblock}

\begin{addedblock}
\begin{codeblock}
  // \ref{common.iterators} Common iterators
  template<typename A, typename B>
  concept bool WeaklyEqualityComparable;  // \expos
  template<Iterator I, Regular S>
  concept bool WeakSentinel;              // \expos

  template <InputIterator I, WeakSentinel<I> S> class common_iterator;
  template <InputIterator I1, WeakSentinel<I1> S1,
            InputIterator I2, WeakSentinel<I2> S2>
    requires EqualityComparable<I1, I2>() && WeaklyEqualityComparable<I1, S2> &&
      WeaklyEqualityComparable<I2, S1>
  bool operator==(
    const common_iterator<I1, S1>& x, const common_iterator<I2, S2>& y);
  template <InputIterator I1, WeakSentinel<I1> S1,
            InputIterator I2, WeakSentinel<I2> S2>
    requires EqualityComparable<I1, I2>() && WeaklyEqualityComparable<I1, S2> &&
      WeaklyEqualityComparable<I2, S1>
  bool operator!=(
    const common_iterator<I1, S1>& x, const common_iterator<I2, S2>& y);

  template <InputIterator I1, WeakSentinel<I1> S1,
            InputIterator I2, WeakSentinel<I2> S2>
    requires SizedIteratorRange<I1, I1> && SizedIteratorRange<I2, I2> &&
      requires (I1 a, I2 b) { {a-b}->DifferenceType<I2>; {b-a}->DifferenceType<I2>; }
      requires (I1 i, S2 s) { {i-s}->DifferenceType<I2>; {s-i}->DifferenceType<I2>; }
      requires (I2 i, S1 s) { {i-s}->DifferenceType<I2>; {s-i}->DifferenceType<I2>; }
  DifferenceType<I2> operator-(
    const common_iterator<I1, S1>& x, const common_iterator<I2, S2>& y);

  // \ref{counted.iterators} Counted iterators and sentinels
  @\ednote{Why not WeakIterator?}@
  template <WeakInputIterator I> class counted_iterator;
  class counted_sentinel;

  template <WeakInputIterator I1, WeakInputIterator I2>
    bool operator==(
      const counted_iterator<I1>& x, const counted_iterator<I2>& y);
  template <WeakInputIterator I>
    bool operator==(
      const counted_iterator<I>& x, counted_sentinel y);
  template <WeakInputIterator I>
    bool operator==(
      counted_sentinel x, const counted_iterator<I>& y);
  bool operator==(counted_sentinel x, counted_sentinel y);
  template <WeakInputIterator I1, WeakInputIterator I2>
    bool operator!=(
      const counted_iterator<I1>& x, const counted_iterator<I2>& y);
  template <WeakInputIterator I>
    bool operator!=(
      const counted_iterator<I>& x, counted_sentinel y);
  template <WeakInputIterator I>
    bool operator!=(
      counted_sentinel x, const counted_iterator<I>& y);
  bool operator!=(counted_sentinel x, counted_sentinel y);

  template <RandomAccessIterator I1, RandomAccessIterator I2>
      requires TotallyOrdered<I1, I2>()
    bool operator<(
      const counted_iterator<I1>& x, const counted_iterator<I2>& y);
  template <RandomAccessIterator I>
    bool operator<(
      const counted_iterator<I>& x, counted_sentinel y);
  template <RandomAccessIterator I>
    bool operator<(
      counted_sentinel x, const counted_iterator<I>& y);
  bool operator<(counted_sentinel x, counted_sentinel y);
  template <RandomAccessIterator I1, RandomAccessIterator I2>
      requires TotallyOrdered<I1, I2>()
    bool operator<=(
      const counted_iterator<I1>& x, const counted_iterator<I2>& y);
  template <RandomAccessIterator I>
    bool operator<=(
      const counted_iterator<I>& x, counted_sentinel y);
  template <RandomAccessIterator I>
    bool operator<=(
      counted_sentinel x, const counted_iterator<I>& y);
  bool operator<=(counted_sentinel x, counted_sentinel y);
  template <RandomAccessIterator I1, RandomAccessIterator I2>
      requires TotallyOrdered<I1, I2>()
    bool operator>(
      const counted_iterator<I1>& x, const counted_iterator<I2>& y);
  template <RandomAccessIterator I>
    bool operator>(
      const counted_iterator<I>& x, counted_sentinel y);
  template <RandomAccessIterator I>
    bool operator>(
      counted_sentinel x, const counted_iterator<I>& y);
  bool operator>(counted_sentinel x, counted_sentinel y);
  template <RandomAccessIterator I1, RandomAccessIterator I2>
      requires TotallyOrdered<I1, I2>()
    bool operator>=(
      const counted_iterator<I1>& x, const counted_iterator<I2>& y);
  template <RandomAccessIterator I>
    bool operator>=(
      const counted_iterator<I>& x, counted_sentinel y);
  template <RandomAccessIterator I>
    bool operator>=(
      counted_sentinel x, const counted_iterator<I>& y);
  bool operator>=(counted_sentinel x, counted_sentinel y);

  template <WeakInputIterator I1, WeakInputIterator I2>
    DifferenceType<I2> operator-(
      const counted_iterator<I1>& x, const counted_iterator<I2>& y);
  template <WeakInputIterator I>
    DifferenceType<I> operator-(
      const counted_iterator<I>& x, counted_sentinel y);
  template <WeakInputIterator I>
    DifferenceType<I> operator-(
      counted_sentinel x, const counted_iterator<I>& y);
  ptrdiff_t operator-(counted_sentinel x, counted_sentinel y);
  template <RandomAccessIterator I>
    counted_iterator<I>
      operator+(DifferenceType<I> n, const counted_iterator<I>& x);
  template <WeakInputIterator I>
    counted_iterator<I> make_counted_iterator(I i, DifferenceType<I> n);

  template <WeakInputIterator I>
    void advance(counted_iterator<I>& i, DifferenceType<I> n);

  // \ref{counted.traits.specializations} \tcode{common_type} specializations
  template<WeakInputIterator I>
    struct common_type<counted_iterator<I>, counted_sentinel>;
  template<WeakInputIterator I>
    struct common_type<counted_sentinel, counted_iterator<I>>;

  // \ref{unreachable.sentinels} Unreachable sentinels
  struct unreachable { };
  template <Iterator I>
    constexpr bool operator==(I const &, unreachable) noexcept;
  template <Iterator I>
    constexpr bool operator==(unreachable, I const &) noexcept;
  constexpr bool operator==(unreachable, unreachable) noexcept;
  template <Iterator I>
    constexpr bool operator!=(I const &, unreachable) noexcept;
  template <Iterator I>
    constexpr bool operator!=(unreachable, I const &) noexcept;
  constexpr bool operator!=(unreachable, unreachable) noexcept;

  // \ref{unreachable.traits.specializations} \tcode{common_type} specializations
  template<Iterator I>
    struct common_type<I, unreachable>;
  template<Iterator I>
    struct common_type<unreachable, I>;
\end{codeblock}
\end{addedblock}
\begin{codeblock}

  // \ref{stream.iterators}, stream iterators:
  template <class T, class charT = char, class traits = char_traits<charT>,
      class Distance = ptrdiff_t>
  class istream_iterator;
  template <class T, class charT, class traits, class Distance>
    bool operator==(const istream_iterator<T,charT,traits,Distance>& x,
            const istream_iterator<T,charT,traits,Distance>& y);
  template <class T, class charT, class traits, class Distance>
    bool operator!=(const istream_iterator<T,charT,traits,Distance>& x,
            const istream_iterator<T,charT,traits,Distance>& y);

  template <class T, class charT = char, class traits = char_traits<charT> >
      class ostream_iterator;

  template<class charT, class traits = char_traits<charT> >
    class istreambuf_iterator;
  template <class charT, class traits>
    bool operator==(const istreambuf_iterator<charT,traits>& a,
            const istreambuf_iterator<charT,traits>& b);
  template <class charT, class traits>
    bool operator!=(const istreambuf_iterator<charT,traits>& a,
            const istreambuf_iterator<charT,traits>& b);

  template <class charT, class traits = char_traits<charT> >
    class ostreambuf_iterator;

  // \ref{iterator.range}, range access:
  template <class C> auto begin(C& c) -> decltype(c.begin());
  template <class C> auto begin(const C& c) -> decltype(c.begin());
  template <class C> auto end(C& c) -> decltype(c.end());
  template <class C> auto end(const C& c) -> decltype(c.end());
  template <class T, size_t N> constexpr T* begin(T (&array)[N]) noexcept;
  template <class T, size_t N> constexpr T* end(T (&array)[N]) noexcept;
  template <class C> constexpr auto cbegin(const C& c) noexcept(noexcept(std::begin(c)))
    -> decltype(std::begin(c));
  template <class C> constexpr auto cend(const C& c) noexcept(noexcept(std::end(c)))
    -> decltype(std::end(c));
  template <class C> auto rbegin(C& c) -> decltype(c.rbegin());
  template <class C> auto rbegin(const C& c) -> decltype(c.rbegin());
  template <class C> auto rend(C& c) -> decltype(c.rend());
  template <class C> auto rend(const C& c) -> decltype(c.rend());
  template <class T, size_t N> reverse_iterator<T*> rbegin(T (&array)[N]);
  template <class T, size_t N> reverse_iterator<T*> rend(T (&array)[N]);
  template <class E> reverse_iterator<const E*> rbegin(initializer_list<E> il);
  template <class E> reverse_iterator<const E*> rend(initializer_list<E> il);
  template <class C> auto crbegin(const C& c) -> decltype(std::rbegin(c));
  template <class C> auto crend(const C& c) -> decltype(std::rend(c));
\end{codeblock}
\begin{addedblock}
\begin{codeblock}
  template <class C> auto size(const C& c) -> decltype(c.size());
  template <class T, size_t N> constexpr size_t begin(T (&array)[N]) noexcept;
  template <class E> size_t size(initializer_list<E> il) noexcept;
\end{codeblock}
\end{addedblock}
\begin{codeblock}
}
\end{codeblock}

\rSec1[iterator.primitives]{Iterator primitives}

\pnum
To simplify the task of defining iterators, the library provides
several classes and functions:

\rSec2[iterator.assoc]{Iterator \textcolor{remclr}{traits}\textcolor{addclr}{associated types}}

\pnum
To implement algorithms only in terms of iterators, it is often necessary to
determine the value and
difference types that correspond to a particular iterator type.
Accordingly, it is required that if
\removed{\tcode{Iterator}
is the type of an iterator}\added{\tcode{WeaklyIncrementable} is the name of a type that models the
WeaklyIncrementable concept(~\ref{weaklyincrementable.iterators}), \tcode{Readable} is the name of a type that
models the Readable concept(~\ref{readable.iterators}), and \tcode{WeakInputIterator} is the name of a
type that models the WeakInputIterator(~\ref{weakinput.iterators}) concept}, the types

\begin{removedblock}
\begin{codeblock}
iterator_traits<Iterator>::difference_type
iterator_traits<Iterator>::value_type
iterator_traits<Iterator>::iterator_category
\end{codeblock}
\end{removedblock}
\begin{addedblock}
\begin{codeblock}
DifferenceType<WeaklyIncrementable>
ValueType<Readable>
IteratorCategory<WeakInputIterator>
\end{codeblock}
\end{addedblock}

be defined as the iterator's difference type, value type and iterator category, respectively.
In addition, the type\removed{s}

\begin{addedblock}
\begin{codeblock}
ReferenceType<Readable>
\end{codeblock}

shall be an alias for \tcode{decltype(*declval<Readable>())}.
\end{addedblock}

\begin{removedblock}
\begin{codeblock}
iterator_traits<Iterator>::reference
iterator_traits<Iterator>::pointer
\end{codeblock}

shall be defined as the iterator's reference and pointer types, that is, for an
iterator object \tcode{a}, the same type as the type of \tcode{*a} and \tcode{a->},
respectively. In the case of an output iterator, the types

\begin{codeblock}
iterator_traits<Iterator>::difference_type
iterator_traits<Iterator>::value_type
iterator_traits<Iterator>::reference
iterator_traits<Iterator>::pointer
\end{codeblock}

may be defined as \tcode{void}.

\pnum
The template
\tcode{iterator_traits<Iterator>}
is defined as

\begin{codeblock}
namespace std {
  template<class Iterator> struct iterator_traits {
    typedef typename Iterator::difference_type difference_type;
    typedef typename Iterator::value_type value_type;
    typedef typename Iterator::pointer pointer;
    typedef typename Iterator::reference reference;
    typedef typename Iterator::iterator_category iterator_category;
  };
}
\end{codeblock}

\pnum
It is specialized for pointers as

\begin{codeblock}
namespace std {
  template<class T> struct iterator_traits<T*> {
    typedef ptrdiff_t difference_type;
    typedef T value_type;
    typedef T* pointer;
    typedef T& reference;
    typedef random_access_iterator_tag iterator_category;
  };
}
\end{codeblock}

and for pointers to const as

\begin{codeblock}
namespace std {
  template<class T> struct iterator_traits<const T*> {
    typedef ptrdiff_t difference_type;
    typedef T value_type;
    typedef const T* pointer;
    typedef const T& reference;
    typedef random_access_iterator_tag iterator_category;
  };
}
\end{codeblock}
\end{removedblock}

\begin{addedblock}
\pnum
\indexlibrary{\idxcode{DifferenceType}}%
\tcode{DifferenceType<T>} is implemented as if:

\indexlibrary{\idxcode{difference_type}}%
\begin{codeblock}
  template <class, class = void> struct difference_type { };
  template <class T> struct difference_type<T*> {
    using type = ptrdiff_t;
  };
  template <> struct difference_type<nullptr_t> {
    using type = ptrdiff_t;
  };
  template <class T> struct difference_type<T[]> {
    using type = ptrdiff_t;
  };
  template <class T, size_t N> struct difference_type<T[N]> {
    using type = ptrdiff_t;
  };
  template <class T>
  struct difference_type<T, void_t<typename T::difference_type>> {
    using type = typename T::difference_type;
  }
  template <class T>
  struct difference_type<T, enable_if_t<is_integral<T>::value>> {
    using type = decltype(declval<T>() - declval<T>());
  };
  template <class T>
    using DifferenceType = typename difference_type<T>::type;
\end{codeblock}

\pnum
Users may specialize \tcode{difference_type} on user-defined types.

\pnum
\indexlibrary{\idxcode{IteratorCategory}}%
\tcode{IteratorCategory<T>} is implemented as if:

\indexlibrary{\idxcode{iterator_category}}%
\begin{codeblock}
  template <class, class = void> struct iterator_category { };
  template <class T> struct iterator_category<T*> {
    using type = random_access_iterator_tag;
  };
  template <class T>
  struct iterator_category<T, void_t<typename T::iterator_category>> {
    using type = typename T::iterator_category;
  };
  template <class T>
    using IteratorCategory = typename iterator_category<T>::type;
\end{codeblock}

\pnum
Users may specialize \tcode{iterator_category} on user-defined types.
\end{addedblock}

\pnum
\enternote
If there is an additional pointer type
\tcode{\,\xname{far}}
such that the difference of two
\tcode{\,\xname{far}}
is of type
\tcode{long},
an implementation may define

\begin{removedblock}
\begin{codeblock}
  template<class T> struct iterator_traits<T @\xname{far}@*> {
    typedef long difference_type;
    typedef T value_type;
    typedef T @\xname{far}@* pointer;
    typedef T @\xname{far}@& reference;
    typedef random_access_iterator_tag iterator_category;
  };
\end{codeblock}
\end{removedblock}
\begin{addedblock}
\begin{codeblock}
  template<class T> struct difference_type<T @\xname{far}@*> {
    using type = long;
  };
  template<class T> struct value_type<T @\xname{far}@*> : remove_cv<T> { };
  template<class T> struct iterator_category<T @\xname{far}@*> {
    using type = random_access_iterator_tag;
  };
\end{codeblock}
\end{addedblock}
\exitnote

\begin{addedblock}
\pnum
For the sake of backwards compatibility, this standard specifies the existence of an \tcode{iterator_traits}
alias that collects an iterator's associated types. It is defined as if:

\indexlibrary{\idxcode{iterator_traits}}%
\begin{codeblock}
  template <WeakInputIterator I, class = void> struct @\xname{pointer_type}@ {
    using type = add_pointer_t<ReferenceType<I>>;
  };
  template <WeakInputIterator I>
  struct @\xname{pointer_type}@<I, void_t<decltype(declval<I>().operator->())>> {
    using type = decltype(declval<I>().operator->());
  };
  struct @\xname{iterator_traits_none}@ { };
  template <WeakIterator I> struct @\xname{iterator_traits_output}@ {
    using difference_type = DifferenceType<I>;
    using value_type = void;
    using reference = void;
    using pointer = void;
    using iterator_category = output_iterator_tag;
  };
  template <WeakInputIterator I> struct @\xname{iterator_traits_input}@ {
    using difference_type = DifferenceType<I>;
    using value_type = ValueType<I>;
    using reference = ReferenceType<I>;
    using pointer = typename @\xname{pointer_type}@<I>::type;
    using iterator_category = IteratorCategory<I>;
  };
  template<class> @\xname{iterator_traits_none}@ @iter_traits@();
  template<WeakIterator O> @\xname{iterator_traits_output}@<O> @iter_traits@();
  template<WeakInputIterator I> @\xname{iterator_traits_input}@<I> @iter_traits@();
  template <class I>
    using iterator_traits = decltype(@\xname{iter_traits}@<I>());
\end{codeblock}

\pnum
\enternote
\tcode{iterator_traits} is a template alias to intentionally break code that tries to specialize
it.
\exitnote

\end{addedblock}

\pnum
\enterexample
To implement a generic
\tcode{reverse}
function, a \Cpp program can do the following:

\begin{codeblock}
template <@\removed{class }@BidirectionalIterator@\added{ I}@>
void reverse(@\changed{BidirectionalIterator}{I}@ first, @\changed{BidirectionalIterator}{I}@ last) {
  @\changed{typename iterator_traits<BidirectionalIterator>::difference_type}{DifferenceType<I>}@ n =
    distance(first, last);
  --n;
  while(n > 0) {
    @\changed{typename iterator_traits<BidirectionalIterator>::value_type}{ValueType<I>}@
      tmp = *first;
    *first++ = *--last;
    *last = tmp;
    n -= 2;
  }
}
\end{codeblock}
\exitexample

\rSec2[iterator.basic]{Basic iterator}

\pnum
The
\tcode{iterator}
template may be used as a base class to ease the definition of required types
for new iterators.

\indexlibrary{\idxcode{iterator}}%
\begin{codeblock}
namespace std {
  template<class Category, class T, class Distance = ptrdiff_t,
    class Pointer = T*, class Reference = T&>
  struct iterator {
    typedef T         value_type;
    typedef Distance  difference_type;
    typedef Pointer   pointer;
    typedef Reference reference;
    typedef Category  iterator_category;
  };
}
\end{codeblock}

\begin{addedblock}
\pnum
\enternote The \tcode{Pointer} and \tcode{Reference} template parameters, and the nested \tcode{pointer}
and \tcode{reference} type aliases are for backward compatibility only; they are never used by any
other part of this standard.\exitnote
\end{addedblock}

\rSec2[std.iterator.tags]{Standard iterator tags}

\pnum
\indexlibrary{\idxcode{weak_input_iterator_tag}}%
\indexlibrary{\idxcode{input_iterator_tag}}%
\indexlibrary{\idxcode{output_iterator_tag}}%
\indexlibrary{\idxcode{forward_iterator_tag}}%
\indexlibrary{\idxcode{bidirectional_iterator_tag}}%
\indexlibrary{\idxcode{random_access_iterator_tag}}%
It is often desirable for a
function template specialization
to find out what is the most specific category of its iterator
argument, so that the function can select the most efficient algorithm at compile time.
To facilitate this, the
library introduces
\techterm{category tag}
classes which \changed{are}{can be} used as compile time tags for algorithm selection.
\added{\enternote The preferred way to dispatch to more specialized algorithm implementations is
with concept-based overloading.\exitnote}
\changed{They}{The category tags} are:
\tcode{\added{weak_input_iterator_tag}},
\tcode{input_iterator_tag},
\tcode{output_iterator_tag},
\tcode{forward_iterator_tag},
\tcode{bidirectional_iterator_tag}
and
\tcode{random_access_iterator_tag}.
For every \added{weak input }iterator of type
\tcode{Iterator},
\tcode{\changed{iterator_traits<Iterator>::it\-er\-a\-tor_ca\-te\-go\-ry}{It\-er\-a\-tor\-Ca\-te\-go\-ry<Iterator>}}
shall be defined to be the most specific category tag that describes the
iterator's behavior.

\begin{codeblock}
namespace std {
  @\added{struct weak_input_iterator_tag \{ \};}@
  struct input_iterator_tag@\added{: public weak_input_iterator_tag}@ { };
  struct output_iterator_tag { };
  struct forward_iterator_tag: public input_iterator_tag { };
  struct bidirectional_iterator_tag: public forward_iterator_tag { };
  struct random_access_iterator_tag: public bidirectional_iterator_tag { };
}
\end{codeblock}

\begin{addedblock}
\pnum
\enternote
The \tcode{output_iterator_tag} is provided for the sake of backward compatibility.
\exitnote
\end{addedblock}

\pnum
\indexlibrary{\idxcode{empty}}%
\indexlibrary{\idxcode{weak_input_iterator_tag}}%
\indexlibrary{\idxcode{input_iterator_tag}}%
\indexlibrary{\idxcode{output_iterator_tag}}%
\indexlibrary{\idxcode{forward_iterator_tag}}%
\indexlibrary{\idxcode{bidirectional_iterator_tag}}%
\indexlibrary{\idxcode{random_access_iterator_tag}}%
\enterexample
For a program-defined iterator
\tcode{BinaryTreeIterator},
it could be included
into the bidirectional iterator category by specializing the
\tcode{\removed{iterator_traits}}\tcode{\added{difference_type}}\added{, }\tcode{\added{value_type}}\added{, and }
\tcode{\added{iterator_category}} template\added{s}:

\begin{removedblock}
\begin{codeblock}
template<class T> struct iterator_traits<BinaryTreeIterator<T> > {
  typedef std::ptrdiff_t difference_type;
  typedef T value_type;
  typedef T* pointer;
  typedef T& reference;
  typedef bidirectional_iterator_tag iterator_category;
};
\end{codeblock}
\end{removedblock}
\begin{addedblock}
\begin{codeblock}
template<class T> struct difference_type<BinaryTreeIterator<T> > {
  using type = std::ptrdiff_t;
};
template<class T> struct value_type<BinaryTreeIterator<T> > {
  using type = T;
};
template<class T> struct iterator_category<BinaryTreeIterator<T> > {
  using type = bidirectional_iterator_tag;
};
\end{codeblock}
\end{addedblock}

Typically, however, it would be easier to derive
\tcode{BinaryTreeIterator<T>}
from
\tcode{iterator<bidirectional_iterator_tag,T,ptrdiff_t\removed{,T*,T\&}>}.
\exitexample

\pnum
\enterexample
If
\tcode{evolve()}
is well defined for bidirectional iterators, but can be implemented more
efficiently for random access iterators, then \changed{the}{one possible} implementation is as
follows:

\begin{codeblock}
template <class BidirectionalIterator>
inline void
evolve(BidirectionalIterator first, BidirectionalIterator last) {
  evolve(first, last,
    @\removed{typename iterator_traits<BidirectionalIterator>::iterator_category()}@
    @\added{IteratorCategory<BidirectionalIterator>\{\}}@);
}

template <class BidirectionalIterator>
void evolve(BidirectionalIterator first, BidirectionalIterator last,
  bidirectional_iterator_tag) {
  // more generic, but less efficient algorithm
}

template <class RandomAccessIterator>
void evolve(RandomAccessIterator first, RandomAccessIterator last,
  random_access_iterator_tag) {
  // more efficient, but less generic algorithm
}
\end{codeblock}
\exitexample

\pnum
\enterexample
If a \Cpp program wants to define a bidirectional iterator for some data structure containing
\tcode{double}
and such that it
works on a large memory model of the implementation, it can do so with:

\begin{codeblock}
class MyIterator :
  public iterator<bidirectional_iterator_tag, double, long@\removed{, T*, T\&}@> {
  // code implementing \tcode{++}, etc.
};
\end{codeblock}

\pnum
Then there is no need to specialize the
\tcode{\removed{iterator_traits}}\tcode{\added{difference_type}}\added{, }
\tcode{\added{value_type}}\added{, or } \tcode{\added{iterator_category}} template\added{s}.
\exitexample

\rSec2[iterator.operations]{Iterator operations}

\pnum
Since only \changed{random access iterators}{models of \tcode{RandomAccessIterator}} provide
\added{the }\tcode{+} \changed{and}{operator, and models of \tcode{SizedIteratorRange} provide the}
\tcode{-}
operator\removed{s}, the library provides two
function templates
\tcode{advance}
and
\tcode{distance}.
These
function templates
use
\tcode{+}
and
\tcode{-}
for random access iterators\added{ and sized iterator ranges, respectively} (and are, therefore, constant
time for them); for input, forward and bidirectional iterators they use
\tcode{++}
to provide linear time
implementations.

\indexlibrary{\idxcode{advance}}%
\begin{removedblock}
\begin{itemdecl}
template <class InputIterator, class Distance>
  void advance(InputIterator& i, Distance n);
\end{itemdecl}
\end{removedblock}
\begin{addedblock}
\begin{itemdecl}
template <WeakIterator I>
  void advance(I& i, DifferenceType<I> n);
\end{itemdecl}
\end{addedblock}

\begin{itemdescr}
\pnum
\requires
\tcode{n}
shall be negative only for bidirectional and random access iterators.

\pnum
\effects
Increments (or decrements for negative
\tcode{n})
iterator reference
\tcode{i}
by
\tcode{n}.
\end{itemdescr}

\begin{addedblock}
\begin{itemdecl}
template <Iterator I, Sentinel<I> S>
  void advance(I& i, S bound);
\end{itemdecl}

\begin{itemdescr}
\pnum
\requires
\tcode{bound} shall be reachable from \tcode{i}.

\pnum
\effects
Increments iterator reference \tcode{i} until
\tcode{i == bound}.

\pnum
If \tcode{I} and \tcode{S} are the same type, this function
is constant time.

\pnum
If \tcode{I} and \tcode{S} model the concept \tcode{SizedIteratorRange}, this
function shall dispatch to \tcode{advance(i, bound - i)}.
\end{itemdescr}

\begin{itemdecl}
template <Iterator I, Sentinel<I> S>
  DifferenceType<I> advance(I& i, DifferenceType<I> n, S bound);
\end{itemdecl}

\begin{itemdescr}
\pnum
\requires
\tcode{n}
shall be negative only for bidirectional and random access iterators. If \tcode{n} is
negative, \tcode{i} shall be reachable from \tcode{bound}; otherwise, \tcode{bound}
shall be reachable from \tcode{i}.

\pnum
\effects
Increments (or decrements for negative \tcode{n}) iterator reference \tcode{i} either
\tcode{n} times or until \tcode{i == bound}, whichever comes first.

\pnum
If \tcode{I} and \tcode{S} model \tcode{SizedIteratorRange}:

\begin{itemize}
\item If \tcode{(0 <= n ? n >= $D$ : n <= $D$)} is true, where $D$ is \tcode{bound - i},
this function dispatches to \tcode{advance(i, bound)},
\item Otherwise, this function dispatches to \tcode{advance(i, n)}.
\end{itemize}

\pnum
\returns
\tcode{n - $M$}, where $M$ is the distance from the starting position of \tcode{i} to
the ending position.
\end{itemdescr}
\end{addedblock}

\indexlibrary{\idxcode{distance}}%
\begin{removedblock}
\begin{itemdecl}
  template<class InputIterator>
      typename iterator_traits<InputIterator>::difference_type
         distance(InputIterator first, InputIterator last);
\end{itemdecl}
\end{removedblock}
\begin{addedblock}
\begin{itemdecl}
template <Iterator I, Sentinel<I> S>
  DifferenceType<I> distance(I first, S last);
\end{itemdecl}
\end{addedblock}

\begin{itemdescr}
\pnum
\effects
If \tcode{\changed{InputIterator}{I}}\added{ and \tcode{S}} \changed{meets the
requirements of random access iterator}{model \tcode{SizedIteratorRange}},
returns \tcode{(last - first)}; otherwise, returns
the number of increments needed to get from
\tcode{first}
to
\tcode{last}.

\pnum
\requires
If \tcode{\changed{InputIterator}{I}}\added{ and \tcode{S}} \changed{meets the
requirements of random access iterator}{model \tcode{SizedIteratorRange}},
\tcode{last} shall be reachable from \tcode{first} or \tcode{first} shall be
reachable from \tcode{last}; otherwise,
\tcode{last}
shall be reachable from
\tcode{first}.
\end{itemdescr}

\indexlibrary{\idxcode{next}}%
\begin{removedblock}
\begin{itemdecl}
template <class ForwardIterator>
  ForwardIterator next(ForwardIterator x,
    typename std::iterator_traits<ForwardIterator>::difference_type n = 1);
\end{itemdecl}
\end{removedblock}
\begin{addedblock}
\begin{itemdecl}
template <WeakIterator I>
  I next(I x, DifferenceType<I> n = 1);
\end{itemdecl}
\end{addedblock}

\begin{itemdescr}
\pnum
\effects Equivalent to \tcode{advance(x, n); return x;}
\end{itemdescr}

\begin{addedblock}
\begin{itemdecl}
template <Iterator I, Sentinel<I> S>
  I next(I x, S bound);
\end{itemdecl}

\begin{itemdescr}
\pnum
\effects Equivalent to \tcode{advance(x, bound); return x;}
\end{itemdescr}

\begin{itemdecl}
template <Iterator I, Sentinel<I> S>
  I next(I x, DifferenceType<I> n, S bound);
\end{itemdecl}

\begin{itemdescr}
\pnum
\effects Equivalent to \tcode{advance(x, n, bound); return x;}
\end{itemdescr}
\end{addedblock}

\indexlibrary{\idxcode{prev}}%
\begin{removedblock}
\begin{itemdecl}
template <class BidirectionalIterator>
  BidirectionalIterator prev(BidirectionalIterator x,
    typename std::iterator_traits<BidirectionalIterator>::difference_type n = 1);
\end{itemdecl}
\end{removedblock}
\begin{addedblock}
\begin{itemdecl}
template <BidirectionalIterator I>
  I prev(I x, DifferenceType<I> n = 1);
\end{itemdecl}
\end{addedblock}

\begin{itemdescr}
\pnum
\effects Equivalent to \tcode{advance(x, -n); return x;}
\end{itemdescr}

\begin{addedblock}
\begin{itemdecl}
template <BidirectionalIterator I>
  I prev(I x, DifferenceType<I> n, I bound);
\end{itemdecl}

\begin{itemdescr}
\pnum
\effects Equivalent to \tcode{advance(x, -n, bound); return x;}
\end{itemdescr}
\end{addedblock}

\rSec1[predef.iterators]{Iterator adaptors}

\rSec2[reverse.iterators]{Reverse iterators}

\pnum
Class template \tcode{reverse_iterator} is an iterator adaptor that iterates from the end of the sequence defined by its underlying iterator to the beginning of that sequence.
The fundamental relation between a reverse iterator and its corresponding iterator
\tcode{i}
is established by the identity:
\tcode{\&*(reverse_iterator(i)) == \&*(i - 1)}.

\rSec3[reverse.iterator]{Class template \tcode{reverse_iterator}}

\indexlibrary{\idxcode{reverse_iterator}}%
\begin{codeblock}
namespace std {
  template <@\changed{class Iterator}{BidirectionalIterator I}@>
  class reverse_iterator @\changed{: public}{\{}@
\end{codeblock}\begin{removedblock}\begin{codeblock}
        iterator<typename iterator_traits<Iterator>::iterator_category,
        typename iterator_traits<Iterator>::value_type,
        typename iterator_traits<Iterator>::difference_type,
        typename iterator_traits<Iterator>::pointer,
        typename iterator_traits<Iterator>::reference> {
\end{codeblock}\end{removedblock}\begin{codeblock}
  public:
\end{codeblock}\begin{removedblock}\begin{codeblock}
    typedef Iterator                                            iterator_type;
    typedef typename iterator_traits<Iterator>::difference_type difference_type;
    typedef typename iterator_traits<Iterator>::reference       reference;
    typedef typename iterator_traits<Iterator>::pointer         pointer;
\end{codeblock}\end{removedblock}\begin{addedblock}\begin{codeblock}
    using iterator_type = I;
    using difference_type = DifferenceType<I>;
    using value_type = ValueType<I>;
    using iterator_category = IteratorCategory<I>;
    using reference = ReferenceType<I>;
\end{codeblock}\end{addedblock}\begin{codeblock}
    reverse_iterator()@\added{ = default}@;
    explicit reverse_iterator(@\changed{Iterator}{I}@ x);
    template <@\changed{class}{BidirectionalIterator}@ U>
      @\added{requires Convertible<U, I>}@
    reverse_iterator(const reverse_iterator<U>& u);
    template <@\changed{class}{BidirectionalIterator}@ U>
      @\added{requires Convertible<U, I>}@
    reverse_iterator& operator=(const reverse_iterator<U>& u);

    @\changed{Iterator}{I}@ base() const;      // explicit
    reference operator*() const;
    @\removed{pointer operator->() const;}@

    reverse_iterator& operator++();
    reverse_iterator  operator++(int);
    reverse_iterator& operator--();
    reverse_iterator  operator--(int);

    reverse_iterator  operator+ (difference_type n) const@\removed{;}@
      @\added{requires RandomAccessIterator<I>;}@
    reverse_iterator& operator+=(difference_type n)@\removed{;}@
      @\added{requires RandomAccessIterator<I>;}@
    reverse_iterator  operator- (difference_type n) const@\removed{;}@
      @\added{requires RandomAccessIterator<I>;}@
    reverse_iterator& operator-=(difference_type n)@\removed{;}@
      @\added{requires RandomAccessIterator<I>;}@
    @\unspec@ operator[](difference_type n) const@\removed{;}@
      @\added{requires RandomAccessIterator<I>}@
  protected:
    @\changed{Iterator}{I}@ current;
  };

  template <@\changed{class Iterator1}{BidirectionalIterator I1}@, @\changed{class Iterator2}{BidirectionalIterator I2}@>
      @\added{requires EqualityComparable<I1, I2>()}@
    bool operator==(
      const reverse_iterator<@\changed{Iterator1}{I1}@>& x,
      const reverse_iterator<@\changed{Iterator2}{I2}@>& y);
  template <@\changed{class Iterator1}{RandomAccessIterator I1}@, @\changed{class Iterator2}{RandomAccessIterator I2}@>
      @\added{requires TotallyOrdered<I1, I2>()}@
    bool operator<(
      const reverse_iterator<@\changed{Iterator1}{I1}@>& x,
      const reverse_iterator<@\changed{Iterator2}{I2}@>& y);
  template <@\changed{class Iterator1}{BidirectionalIterator I1}@, @\changed{class Iterator2}{BidirectionalIterator I2}@>
      @\added{requires EqualityComparable<I1, I2>()}@
    bool operator!=(
      const reverse_iterator<@\changed{Iterator1}{I1}@>& x,
      const reverse_iterator<@\changed{Iterator2}{I2}@>& y);
  template <@\changed{class Iterator1}{RandomAccessIterator I1}@, @\changed{class Iterator2}{RandomAccessIterator I2}@>
      @\added{requires TotallyOrdered<I1, I2>()}@
    bool operator>(
      const reverse_iterator<@\changed{Iterator1}{I1}@>& x,
      const reverse_iterator<@\changed{Iterator2}{I2}@>& y);
  template <@\changed{class Iterator1}{RandomAccessIterator I1}@, @\changed{class Iterator2}{RandomAccessIterator I2}@>
      @\added{requires TotallyOrdered<I1, I2>()}@
    bool operator>=(
      const reverse_iterator<@\changed{Iterator1}{I1}@>& x,
      const reverse_iterator<@\changed{Iterator2}{I2}@>& y);
  template <@\changed{class Iterator1}{RandomAccessIterator I1}@, @\changed{class Iterator2}{RandomAccessIterator I2}@>
      @\added{requires TotallyOrdered<I1, I2>()}@
    bool operator<=(
      const reverse_iterator<@\changed{Iterator1}{I1}@>& x,
      const reverse_iterator<@\changed{Iterator2}{I2}@>& y);
  template <@\changed{class Iterator1}{BidirectionalIterator I1}@, @\changed{class Iterator2}{BidirectionalIterator I2}@>
      @\added{requires SizedIteratorRange<I2, I1>}@
    @\changed{auto}{DifferenceType<I2>}@ operator-(
      const reverse_iterator<@\changed{Iterator1}{I1}@>& x,
      const reverse_iterator<@\changed{Iterator2}{I2}@>& y)@\removed{ ->decltype(y.base() - x.base())}@;
  template <@\changed{class Iterator}{RandomAccessIterator I}@>
    reverse_iterator<@\changed{Iterator}{I}@>
      operator+(
    @\changed{typename reverse_iterator<Iterator>::difference_type}{DifferenceType<I>}@ n,
    const reverse_iterator<@\changed{Iterator}{I}@>& x);

  template <@\changed{class Iterator}{BidirectionalIterator I}@>
    reverse_iterator<@\changed{Iterator}{I}@> make_reverse_iterator(@\changed{Iterator}{I}@ i);
}
\end{codeblock}

\begin{removedblock}
\rSec3[reverse.iter.requirements]{\tcode{reverse_iterator} requirements}

\pnum
The template parameter
\tcode{Iterator}
shall meet all the requirements of a Bidirectional Iterator~(\ref{bidirectional.iterators}).

\pnum
Additionally,
\tcode{Iterator}
shall meet the requirements of a Random Access Iterator~(\ref{random.access.iterators})
if any of the members
\tcode{operator+}~(\ref{reverse.iter.op+}),
\tcode{operator-}~(\ref{reverse.iter.op-}),
\tcode{operator+=}~(\ref{reverse.iter.op+=}),
\tcode{operator-=}~(\ref{reverse.iter.op-=}),
\tcode{operator\,[]}~(\ref{reverse.iter.opindex}),
or the global operators
\tcode{operator<}~(\ref{reverse.iter.op<}),
\tcode{operator>}~(\ref{reverse.iter.op>}),\\
\tcode{operator\,<=}~(\ref{reverse.iter.op<=}),
\tcode{operator>=}~(\ref{reverse.iter.op>=}),
\tcode{operator-}~(\ref{reverse.iter.opdiff})
or
\tcode{operator+}~(\ref{reverse.iter.opsum})
are referenced in a way that requires instantiation~(\cxxref{temp.inst}).
\end{removedblock}

\rSec3[reverse.iter.ops]{\tcode{reverse_iterator} operations}

\rSec4[reverse.iter.cons]{\tcode{reverse_iterator} constructor}

\indexlibrary{\idxcode{reverse_iterator}!\tcode{reverse_iterator}}%
\begin{itemdecl}
reverse_iterator()@\added{ = default}@;
\end{itemdecl}

\begin{itemdescr}
\pnum
\effects
\changed{Value}{Default} initializes
\tcode{current}.
Iterator operations applied to the resulting iterator have defined behavior
if and only if the corresponding operations are defined on a
\changed{value}{default}-initialized iterator of type
\tcode{\changed{Iterator}{I}}.\added{If \tcode{I} is a literal type, then this
constructor shall be a trivial constructor.}
\end{itemdescr}

\indexlibrary{\idxcode{reverse_iterator}!constructor}%

\begin{itemdecl}
explicit reverse_iterator(@\changed{Iterator}{I}@ x);
\end{itemdecl}

\begin{itemdescr}
\pnum
\effects
Initializes
\tcode{current}
with \tcode{x}.
\end{itemdescr}

\indexlibrary{\idxcode{reverse_iterator}!constructor}%

\begin{itemdecl}
template <@\changed{class}{BidirectionalIterator}@ U>
  @\added{requires Convertible<U, I>}@
reverse_iterator(const reverse_iterator<U>& u);
\end{itemdecl}

\begin{itemdescr}
\pnum
\effects
Initializes
\tcode{current}
with
\tcode{u.current}.
\end{itemdescr}

\rSec4[reverse.iter.op=]{\tcode{reverse_iterator::operator=}}

\indexlibrary{\idxcode{operator=}!\tcode{reverse_iterator}}%
\begin{itemdecl}
template <@\changed{class}{BidirectionalIterator}@ U>
  @\added{requires Convertible<U, I>}@
reverse_iterator&
  operator=(const reverse_iterator<U>& u);
\end{itemdecl}

\begin{itemdescr}
\pnum
\effects
Assigns \tcode{u.base()} to \tcode{current}.

\pnum
\returns
\tcode{*this}.
\end{itemdescr}

\rSec4[reverse.iter.conv]{Conversion}

\indexlibrary{\idxcode{base}!\idxcode{reverse_iterator}}%
\indexlibrary{\idxcode{reverse_iterator}!\idxcode{base}}%
\begin{itemdecl}
@\changed{Iterator}{I} base() const;          // explicit
\end{itemdecl}

\begin{itemdescr}
\pnum
\returns
\tcode{current}.
\end{itemdescr}

\rSec4[reverse.iter.op.star]{\tcode{operator*}}

\indexlibrary{\idxcode{operator*}!\idxcode{reverse_iterator}}%
\begin{itemdecl}
reference operator*() const;
\end{itemdecl}

\begin{itemdescr}
\pnum
\effects
\begin{codeblock}
Iterator tmp = current;
return *--tmp;
\end{codeblock}

\end{itemdescr}

\begin{removedblock}
\rSec4[reverse.iter.opref]{\tcode{operator->}}

\indexlibrary{\idxcode{operator->}!\idxcode{reverse_iterator}}%
\begin{itemdecl}
pointer operator->() const;
\end{itemdecl}

\begin{itemdescr}
\pnum
\returns \tcode{std::addressof(operator*())}.
\end{itemdescr}
\end{removedblock}

\rSec4[reverse.iter.op++]{\tcode{operator++}}

\indexlibrary{\idxcode{operator++}!\idxcode{reverse_iterator}}%
\begin{itemdecl}
reverse_iterator& operator++();
\end{itemdecl}

\begin{itemdescr}
\pnum
\effects
\tcode{\dcr current;}

\pnum
\returns
\tcode{*this}.
\end{itemdescr}

\indexlibrary{\idxcode{operator++}!\idxcode{reverse_iterator}}%
\indexlibrary{\idxcode{reverse_iterator}!\idxcode{operator++}}%
\begin{itemdecl}
reverse_iterator operator++(int);
\end{itemdecl}

\begin{itemdescr}
\pnum
\effects
\begin{codeblock}
reverse_iterator tmp = *this;
--current;
return tmp;
\end{codeblock}
\end{itemdescr}

\rSec4[reverse.iter.op\dcr]{\tcode{operator\dcr}}

\indexlibrary{\idxcode{operator\dcr}!\idxcode{reverse_iterator}}%
\begin{itemdecl}
reverse_iterator& operator--();
\end{itemdecl}

\begin{itemdescr}
\pnum
\effects
\tcode{++current}

\pnum
\returns
\tcode{*this}.
\end{itemdescr}

\indexlibrary{\idxcode{operator\dcr}!\idxcode{reverse_iterator}}%
\indexlibrary{\idxcode{reverse_iterator}!\idxcode{operator\dcr}}%
\begin{itemdecl}
reverse_iterator operator--(int);
\end{itemdecl}

\begin{itemdescr}
\pnum
\effects
\begin{codeblock}
reverse_iterator tmp = *this;
++current;
return tmp;
\end{codeblock}
\end{itemdescr}

\rSec4[reverse.iter.op+]{\tcode{operator+}}

\indexlibrary{\idxcode{operator+}!\idxcode{reverse_iterator}}%
\begin{itemdecl}
reverse_iterator
operator+(typename reverse_iterator<@\changed{Iterator}{I}@>::difference_type n) const@\removed{;}@
  @\added{requires RandomAccessIterator<I>;}@
\end{itemdecl}

\begin{itemdescr}
\pnum
\returns
\tcode{reverse_iterator(current-n)}.
\end{itemdescr}

\rSec4[reverse.iter.op+=]{\tcode{operator+=}}

\indexlibrary{\idxcode{operator+=}!\idxcode{reverse_iterator}}%
\begin{itemdecl}
reverse_iterator&
operator+=(typename reverse_iterator<@\changed{Iterator}{I}@>::difference_type n)@\removed{;}@
  @\added{requires RandomAccessIterator<I>;}@
\end{itemdecl}

\begin{itemdescr}
\pnum
\effects
\tcode{current -= n;}

\pnum
\returns
\tcode{*this}.
\end{itemdescr}

\rSec4[reverse.iter.op-]{\tcode{operator-}}

\indexlibrary{\idxcode{operator-}!\idxcode{reverse_iterator}}%
\begin{itemdecl}
reverse_iterator
operator-(typename reverse_iterator<@\changed{Iterator}{I}@>::difference_type n) const@\removed{;}@
  @\added{requires RandomAccessIterator<I>;}@
\end{itemdecl}

\begin{itemdescr}
\pnum
\returns
\tcode{reverse_iterator(current+n)}.
\end{itemdescr}

\rSec4[reverse.iter.op-=]{\tcode{operator-=}}

\indexlibrary{\idxcode{operator-=}!\idxcode{reverse_iterator}}%
\begin{itemdecl}
reverse_iterator&
operator-=(typename reverse_iterator<@\changed{Iterator}{I}@>::difference_type n)@\removed{;}@
  @\added{requires RandomAccessIterator<I>;}@
\end{itemdecl}

\begin{itemdescr}
\pnum
\effects
\tcode{current += n;}

\pnum
\returns
\tcode{*this}.
\end{itemdescr}

\rSec4[reverse.iter.opindex]{\tcode{operator[]}}

\indexlibrary{\idxcode{operator[]}!\idxcode{reverse_iterator}}%
\begin{itemdecl}
@\unspec@ operator[](
  typename reverse_iterator<@\changed{Iterator}{I}@>::difference_type n) const@\removed{;}@
    @\added{requires RandomAccessIterator<I>;}@
\end{itemdecl}

\begin{itemdescr}
\pnum
\returns
\tcode{current[-n-1]}.
\end{itemdescr}

\rSec4[reverse.iter.op==]{\tcode{operator==}}

\indexlibrary{\idxcode{operator==}!\idxcode{reverse_iterator}}%
\begin{itemdecl}
template <@\changed{class Iterator1}{BidirectionalIterator I1}@, @\changed{class Iterator2}{BidirectionalIterator I2}@>
    @\added{requires EqualityComparable<I1, I2>()}@
  bool operator==(
    const reverse_iterator<@\changed{Iterator1}{I1}@>& x,
    const reverse_iterator<@\changed{Iterator2}{I2}@>& y);
\end{itemdecl}

\begin{itemdescr}
\pnum
\returns
\tcode{x.current == y.current}.
\end{itemdescr}

\rSec4[reverse.iter.op<]{\tcode{operator<}}

\indexlibrary{\idxcode{operator<}!\idxcode{reverse_iterator}}%
\begin{itemdecl}
template <@\changed{class Iterator1}{RandomAccessIterator I1}@, @\changed{class Iterator2}{RandomAccessIterator I2}@>
    @\added{requires TotallyOrdered<I1, I2>()}@
  bool operator<(
    const reverse_iterator<@\changed{Iterator1}{I1}@>& x,
    const reverse_iterator<@\changed{Iterator2}{I2}@>& y);
\end{itemdecl}

\begin{itemdescr}
\pnum
\returns
\tcode{x.current > y.current}.
\end{itemdescr}

\rSec4[reverse.iter.op!=]{\tcode{operator!=}}

\indexlibrary{\idxcode{operator"!=}!\idxcode{reverse_iterator}}%
\begin{itemdecl}
template <@\changed{class Iterator1}{BidirectionalIterator I1}@, @\changed{class Iterator2}{BidirectionalIterator I2}@>
    @\added{requires EqualityComparable<I1, I2>()}@
  bool operator!=(
    const reverse_iterator<@\changed{Iterator1}{I1}@>& x,
    const reverse_iterator<@\changed{Iterator2}{I2}@>& y);
\end{itemdecl}

\begin{itemdescr}
\pnum
\returns
\tcode{x.current != y.current}.
\end{itemdescr}

\rSec4[reverse.iter.op>]{\tcode{operator>}}

\indexlibrary{\idxcode{operator>}!\idxcode{reverse_iterator}}%
\begin{itemdecl}
template <@\changed{class Iterator1}{RandomAccessIterator I1}@, @\changed{class Iterator2}{RandomAccessIterator I2}@>
    @\added{requires TotallyOrdered<I1, I2>()}@
  bool operator>(
    const reverse_iterator<@\changed{Iterator1}{I1}@>& x,
    const reverse_iterator<@\changed{Iterator2}{I2}@>& y);
\end{itemdecl}

\begin{itemdescr}
\pnum
\returns
\tcode{x.current < y.current}.
\end{itemdescr}

\rSec4[reverse.iter.op>=]{\tcode{operator>=}}

\indexlibrary{\idxcode{operator>=}!\idxcode{reverse_iterator}}%
\begin{itemdecl}
template <@\changed{class Iterator1}{RandomAccessIterator I1}@, @\changed{class Iterator2}{RandomAccessIterator I2}@>
    @\added{requires TotallyOrdered<I1, I2>()}@
  bool operator>=(
    const reverse_iterator<@\changed{Iterator1}{I1}@>& x,
    const reverse_iterator<@\changed{Iterator2}{I2}@>& y);
\end{itemdecl}

\begin{itemdescr}
\pnum
\returns
\tcode{x.current <= y.current}.
\end{itemdescr}

\rSec4[reverse.iter.op<=]{\tcode{operator<=}}

\indexlibrary{\idxcode{operator<=}!\idxcode{reverse_iterator}}%
\begin{itemdecl}
template <@\changed{class Iterator1}{RandomAccessIterator I1}@, @\changed{class Iterator2}{RandomAccessIterator I2}@>
    @\added{requires TotallyOrdered<I1, I2>()}@
  bool operator<=(
    const reverse_iterator<@\changed{Iterator1}{I1}@>& x,
    const reverse_iterator<@\changed{Iterator2}{I2}@>& y);
\end{itemdecl}

\begin{itemdescr}
\pnum
\returns
\tcode{x.current >= y.current}.
\end{itemdescr}

\rSec4[reverse.iter.opdiff]{\tcode{operator-}}

\indexlibrary{\idxcode{operator-}!\idxcode{reverse_iterator}}%
\begin{itemdecl}
template <@\changed{class Iterator1}{BidirectionalIterator I1}@, @\changed{class Iterator2}{BidirectionalIterator I2}@>
    @\added{requires SizedIteratorRange<I2, I1>}@
  @\changed{auto}{DifferenceType<I2>}@ operator-(
    const reverse_iterator<@\changed{Iterator1}{I1}@>& x,
    const reverse_iterator<@\changed{Iterator2}{I2}@>& y)@\removed{ ->decltype(y.base() - x.base())}@;
\end{itemdecl}

\begin{itemdescr}
\pnum
\returns
\tcode{y.current - x.current}.
\end{itemdescr}

\rSec4[reverse.iter.opsum]{\tcode{operator+}}

\indexlibrary{\idxcode{operator+}!\idxcode{reverse_iterator}}%
\begin{itemdecl}
template <@\changed{class Iterator}{RandomAccessIterator I}@>
  reverse_iterator<@\changed{Iterator}{I}@>
    operator+(
  @\changed{typename reverse_iterator<Iterator>::difference_type}{DifferenceType<I>}@ n,
  const reverse_iterator<@\changed{Iterator}{I}@>& x);
\end{itemdecl}

\begin{itemdescr}
\pnum
\returns
\tcode{reverse_iterator<\changed{Iterator}{I}> (x.current - n)}.
\end{itemdescr}

\rSec4[reverse.iter.make]{Non-member function \tcode{make_reverse_iterator()}}

\indexlibrary{\idxcode{reverse_iterator}!\idxcode{make_reverse_iterator}~non-member~function}
\indexlibrary{\idxcode{make_reverse_iterator}}%
\begin{itemdecl}
template <@\changed{class Iterator}{BidirectionalIterator I}@>
  reverse_iterator<@\changed{Iterator}{I}@> make_reverse_iterator(@\changed{Iterator}{I}@ i);
\end{itemdecl}

\begin{itemdescr}
\pnum
\returns
\tcode{reverse_iterator<\changed{Iterator}{I}>(i)}.
\end{itemdescr}

\rSec2[insert.iterators]{Insert iterators}

\pnum
To make it possible to deal with insertion in the same way as writing into an array, a special kind of iterator
adaptors, called
\techterm{insert iterators},
are provided in the library.
With regular iterator classes,

\begin{codeblock}
while (first != last) *result++ = *first++;
\end{codeblock}

causes a range \range{first}{last}
to be copied into a range starting with result.
The same code with
\tcode{result}
being an insert iterator will insert corresponding elements into the container.
This device allows all of the
copying algorithms in the library to work in the
\techterm{insert mode}
instead of the \techterm{regular overwrite} mode.

\pnum
An insert iterator is constructed from a container and possibly one of its iterators pointing to where
insertion takes place if it is neither at the beginning nor at the end of the container.
Insert iterators satisfy the requirements of output iterators.
\tcode{operator*}
returns the insert iterator itself.
The assignment
\tcode{operator=(const T\& x)}
is defined on insert iterators to allow writing into them, it inserts
\tcode{x}
right before where the insert iterator is pointing.
In other words, an insert iterator is like a cursor pointing into the
container where the insertion takes place.
\tcode{back_insert_iterator}
inserts elements at the end of a container,
\tcode{front_insert_iterator}
inserts elements at the beginning of a container, and
\tcode{insert_iterator}
inserts elements where the iterator points to in a container.
\tcode{back_inserter},
\tcode{front_inserter},
and
\tcode{inserter}
are three
functions making the insert iterators out of a container.

\rSec3[back.insert.iterator]{Class template \tcode{back_insert_iterator}}

\indexlibrary{\idxcode{back_insert_iterator}}%
\ednote{Specify this in terms of a Container concept? Or Iterable? Or leave it?}
\begin{codeblock}
namespace std {
  template <class Container>
  class back_insert_iterator @\changed{:}{\{}@
    @\removed{public iterator<output_iterator_tag,void,void,void,void> \}}@
  protected:
    Container* container;

  public:
    @\changed{typedef Container}{using}@ container_type@\added{ = Container}@;
    @\added{using difference_type = ptrdiff_t;}@
    @\added{using iterator_category = output_iterator_tag;}@
    @\added{back_insert_iterator() = default;}@
    explicit back_insert_iterator(Container& x);
    back_insert_iterator<Container>&
      operator=(const typename Container::value_type& value);
    back_insert_iterator<Container>&
      operator=(typename Container::value_type&& value);

    back_insert_iterator<Container>& operator*();
    back_insert_iterator<Container>& operator++();
    back_insert_iterator<Container>  operator++(int);
  };

  template <class Container>
    back_insert_iterator<Container> back_inserter(Container& x);
}
\end{codeblock}

\rSec3[back.insert.iter.ops]{\tcode{back_insert_iterator} operations}

\rSec4[back.insert.iter.cons]{\tcode{back_insert_iterator} constructor}

\indexlibrary{\idxcode{back_insert_iterator}!\idxcode{back_insert_iterator}}%
\begin{addedblock}
\begin{itemdecl}
back_insert_iterator() = default;
\end{itemdecl}

\begin{itemdescr}
\pnum
\effects
Default-initializes
\tcode{container}. This constructor shall be a trivial constructor.
\end{itemdescr}
\end{addedblock}

\indexlibrary{\idxcode{back_insert_iterator}!constructor}%

\begin{itemdecl}
explicit back_insert_iterator(Container& x);
\end{itemdecl}

\begin{itemdescr}
\pnum
\effects
Initializes
\tcode{container}
with \tcode{std::addressof(x)}.
\end{itemdescr}

\rSec4[back.insert.iter.op=]{\tcode{back_insert_iterator::operator=}}

\indexlibrary{\idxcode{operator=}!\idxcode{back_insert_iterator}}%
\begin{itemdecl}
back_insert_iterator<Container>&
  operator=(const typename Container::value_type& value);
\end{itemdecl}

\begin{itemdescr}
\pnum
\effects
\tcode{container->push_back(value);}

\pnum
\returns
\tcode{*this}.
\end{itemdescr}

\indexlibrary{\idxcode{operator=}!\idxcode{back_insert_iterator}}%
\begin{itemdecl}
back_insert_iterator<Container>&
  operator=(typename Container::value_type&& value);
\end{itemdecl}

\begin{itemdescr}
\pnum
\effects
\tcode{container->push_back(std::move(value));}

\pnum
\returns
\tcode{*this}.
\end{itemdescr}

\rSec4[back.insert.iter.op*]{\tcode{back_insert_iterator::operator*}}

\indexlibrary{\idxcode{operator*}!\idxcode{back_insert_iterator}}%
\begin{itemdecl}
back_insert_iterator<Container>& operator*();
\end{itemdecl}

\begin{itemdescr}
\pnum
\returns
\tcode{*this}.
\end{itemdescr}

\rSec4[back.insert.iter.op++]{\tcode{back_insert_iterator::operator++}}

\indexlibrary{\idxcode{operator++}!\idxcode{back_insert_iterator}}%
\begin{itemdecl}
back_insert_iterator<Container>& operator++();
back_insert_iterator<Container>  operator++(int);
\end{itemdecl}

\begin{itemdescr}
\pnum
\returns
\tcode{*this}.
\end{itemdescr}

\rSec4[back.inserter]{ \tcode{back_inserter}}

\indexlibrary{\idxcode{back_inserter}}%
\begin{itemdecl}
template <class Container>
  back_insert_iterator<Container> back_inserter(Container& x);
\end{itemdecl}

\begin{itemdescr}
\pnum
\returns
\tcode{back_insert_iterator<Container>(x)}.
\end{itemdescr}

\rSec3[front.insert.iterator]{Class template \tcode{front_insert_iterator}}

\indexlibrary{\idxcode{front_insert_iterator}}%
\begin{codeblock}
namespace std {
  template <class Container>
  class front_insert_iterator @\changed{:}{\{}@
    @\removed{public iterator<output_iterator_tag,void,void,void,void> \}}@
  protected:
    Container* container;

  public:
    @\changed{typedef Container}{using}@ container_type@\added{ = Container}@;
    @\added{using difference_type = ptrdiff_t;}@
    @\added{using iterator_category = output_iterator_tag;}@
    @\added{front_insert_iterator() = default;}@
    explicit front_insert_iterator(Container& x);
    front_insert_iterator<Container>&
      operator=(const typename Container::value_type& value);
    front_insert_iterator<Container>&
      operator=(typename Container::value_type&& value);

    front_insert_iterator<Container>& operator*();
    front_insert_iterator<Container>& operator++();
    front_insert_iterator<Container>  operator++(int);
  };

  template <class Container>
    front_insert_iterator<Container> front_inserter(Container& x);
}
\end{codeblock}

\rSec3[front.insert.iter.ops]{\tcode{front_insert_iterator} operations}

\rSec4[front.insert.iter.cons]{\tcode{front_insert_iterator} constructor}

\indexlibrary{\idxcode{front_insert_iterator}!\idxcode{front_insert_iterator}}%
\begin{addedblock}
\begin{itemdecl}
front_insert_iterator() = default;
\end{itemdecl}

\begin{itemdescr}
\pnum
\effects
Default-initializes
\tcode{container}. This constructor shall be a trivial constructor.
\end{itemdescr}
\end{addedblock}

\indexlibrary{\idxcode{front_insert_iterator}!constructor}%

\begin{itemdecl}
explicit front_insert_iterator(Container& x);
\end{itemdecl}

\begin{itemdescr}
\pnum
\effects
Initializes
\tcode{container}
with \tcode{std::addressof(x)}.
\end{itemdescr}

\rSec4[front.insert.iter.op=]{\tcode{front_insert_iterator::operator=}}

\indexlibrary{\idxcode{operator=}!\idxcode{front_insert_iterator}}%
\begin{itemdecl}
front_insert_iterator<Container>&
  operator=(const typename Container::value_type& value);
\end{itemdecl}

\begin{itemdescr}
\pnum
\effects
\tcode{container->push_front(value);}

\pnum
\returns
\tcode{*this}.
\end{itemdescr}

\indexlibrary{\idxcode{operator=}!\idxcode{front_insert_iterator}}%
\begin{itemdecl}
front_insert_iterator<Container>&
  operator=(typename Container::value_type&& value);
\end{itemdecl}

\begin{itemdescr}
\pnum
\effects
\tcode{container->push_front(std::move(value));}

\pnum
\returns
\tcode{*this}.
\end{itemdescr}

\rSec4[front.insert.iter.op*]{\tcode{front_insert_iterator::operator*}}

\indexlibrary{\idxcode{operator*}!\idxcode{front_insert_iterator}}%
\begin{itemdecl}
front_insert_iterator<Container>& operator*();
\end{itemdecl}

\begin{itemdescr}
\pnum
\returns
\tcode{*this}.
\end{itemdescr}

\rSec4[front.insert.iter.op++]{\tcode{front_insert_iterator::operator++}}

\indexlibrary{\idxcode{operator++}!\idxcode{front_insert_iterator}}%
\begin{itemdecl}
front_insert_iterator<Container>& operator++();
front_insert_iterator<Container>  operator++(int);
\end{itemdecl}

\begin{itemdescr}
\pnum
\returns
\tcode{*this}.
\end{itemdescr}

\rSec4[front.inserter]{\tcode{front_inserter}}

\indexlibrary{\idxcode{front_inserter}}%
\begin{itemdecl}
template <class Container>
  front_insert_iterator<Container> front_inserter(Container& x);
\end{itemdecl}

\begin{itemdescr}
\pnum
\returns
\tcode{front_insert_iterator<Container>(x)}.
\end{itemdescr}

\rSec3[insert.iterator]{Class template \tcode{insert_iterator}}

\indexlibrary{\idxcode{insert_iterator}}%
\begin{codeblock}
namespace std {
  template <class Container>
  class insert_iterator @\changed{:}{\{}@
    @\removed{public iterator<output_iterator_tag,void,void,void,void> \}}@
  protected:
    Container* container;
    typename Container::iterator iter;

  public:
    @\changed{typedef Container}{using}@ container_type@\added{ = Container}@;
    @\added{using difference_type = ptrdiff_t;}@
    @\added{using iterator_category = output_iterator_tag;}@
    @\added{insert_iterator() = default;}@
    insert_iterator(Container& x, typename Container::iterator i);
    insert_iterator<Container>&
      operator=(const typename Container::value_type& value);
    insert_iterator<Container>&
      operator=(typename Container::value_type&& value);

    insert_iterator<Container>& operator*();
    insert_iterator<Container>& operator++();
    insert_iterator<Container>& operator++(int);
  };

  template <class Container>
    insert_iterator<Container> inserter(Container& x, typename Container::iterator i);
}
\end{codeblock}

\rSec3[insert.iter.ops]{\tcode{insert_iterator} operations}

\rSec4[insert.iter.cons]{\tcode{insert_iterator} constructor}

\indexlibrary{\idxcode{insert_iterator}!\idxcode{insert_iterator}}%
\begin{addedblock}
\begin{itemdecl}
insert_iterator() = default;
\end{itemdecl}

\begin{itemdescr}
\pnum
\effects
Default-initializes
\tcode{container} and \tcode{iter}. If \tcode{Container::iterator} is a literal
type, then this constructor shall be a trivial constructor.
\end{itemdescr}
\end{addedblock}

\indexlibrary{\idxcode{insert_iterator}!constructor}%

\begin{itemdecl}
insert_iterator(Container& x, typename Container::iterator i);
\end{itemdecl}

\begin{itemdescr}
\pnum
\effects
Initializes
\tcode{container}
with \tcode{std::addressof(x)} and
\tcode{iter}
with \tcode{i}.
\end{itemdescr}

\rSec4[insert.iter.op=]{\tcode{insert_iterator::operator=}}

\indexlibrary{\idxcode{operator=}!\idxcode{insert_iterator}}%
\begin{itemdecl}
insert_iterator<Container>&
  operator=(const typename Container::value_type& value);
\end{itemdecl}

\begin{itemdescr}
\pnum
\effects
\begin{codeblock}
iter = container->insert(iter, value);
++iter;
\end{codeblock}

\pnum
\returns
\tcode{*this}.
\end{itemdescr}

\indexlibrary{\idxcode{operator=}!\idxcode{insert_iterator}}%
\begin{itemdecl}
insert_iterator<Container>&
  operator=(typename Container::value_type&& value);
\end{itemdecl}

\begin{itemdescr}
\pnum
\effects
\begin{codeblock}
iter = container->insert(iter, std::move(value));
++iter;
\end{codeblock}

\pnum
\returns
\tcode{*this}.
\end{itemdescr}

\rSec4[insert.iter.op*]{\tcode{insert_iterator::operator*}}

\indexlibrary{\idxcode{operator*}!\idxcode{insert_iterator}}%
\begin{itemdecl}
insert_iterator<Container>& operator*();
\end{itemdecl}

\begin{itemdescr}
\pnum
\returns
\tcode{*this}.
\end{itemdescr}

\rSec4[insert.iter.op++]{\tcode{insert_iterator::operator++}}

\indexlibrary{\idxcode{operator++}!\idxcode{insert_iterator}}%
\begin{itemdecl}
insert_iterator<Container>& operator++();
insert_iterator<Container>& operator++(int);
\end{itemdecl}

\begin{itemdescr}
\pnum
\returns
\tcode{*this}.
\end{itemdescr}

\rSec4[inserter]{\tcode{inserter}}

\indexlibrary{\idxcode{inserter}}%
\begin{itemdecl}
template <class Container>
  insert_iterator<Container> inserter(Container& x, typename Container::iterator i);
\end{itemdecl}

\begin{itemdescr}
\pnum
\returns
\tcode{insert_iterator<Container>(x, i)}.
\end{itemdescr}

\rSec2[move.iterators]{Move iterators}

\pnum
Class template \tcode{move_iterator} is an iterator adaptor
with the same behavior as the underlying iterator except that its
indirection operator implicitly converts the value returned by the
underlying iterator's indirection operator to an rvalue reference.
Some generic algorithms can be called with move iterators to replace
copying with moving. \ednote{Pretty sure this is untrue now given how
the algorithms that do copying are constrained with IndirectlyCopyable.}

\pnum
\enterexample

\begin{codeblock}
list<string> s;
// populate the list \tcode{s}
vector<string> v1(s.begin(), s.end());          // copies strings into \tcode{v1}
vector<string> v2(make_move_iterator(s.begin()),
                  make_move_iterator(s.end())); // moves strings into \tcode{v2}
\end{codeblock}

\exitexample

\rSec3[move.iterator]{Class template \tcode{move_iterator}}

\indexlibrary{\idxcode{move_iterator}}%
\begin{codeblock}
namespace std {
  template <@\changed{class Iterator}{WeakInputIterator I}@>
    @\added{requires Same<ReferenceType<I>, ValueType<I>\&>}@
  class move_iterator {
  public:
\end{codeblock}\begin{removedblock}\begin{codeblock}
    typedef Iterator                                              iterator_type;
    typedef typename iterator_traits<Iterator>::difference_type   difference_type;
    typedef Iterator                                              pointer;
    typedef typename iterator_traits<Iterator>::value_type        value_type;
    typedef typename iterator_traits<Iterator>::iterator_category iterator_category;
    typedef value_type&&                                          reference;
\end{codeblock}\end{removedblock}\begin{addedblock}\begin{codeblock}
    using iterator_type = I;
    using difference_type = DifferenceType<I>;
    using value_type = ValueType<I>;
    using iterator_category = IteratorCategory<I>;
    using reference = ValueType<I>&&;
\end{codeblock}\end{addedblock}\begin{codeblock}

    move_iterator()@\added{ = default}@;
    explicit move_iterator(@\changed{Iterator}{I}@ i);
    template <@\changed{class}{WeakInputIterator}@ U>
      @\added{requires Convertible<U, I>}@
    move_iterator(const move_iterator<U>& u);
    template <@\changed{class}{WeakInputIterator}@ U>
      @\added{requires Convertible<U, I>}@
    move_iterator& operator=(const move_iterator<U>& u);

    iterator_type base() const;
    reference operator*() const;
    @\removed{pointer operator->() const;}@

    move_iterator& operator++();
    move_iterator operator++(int);
    move_iterator& operator--()@\removed{;}@
      @\added{requires BidirectionalIterator<I>;}@
    @\added{requires BidirectionalIterator<I>}@
      move_iterator operator--(int)@\removed{;}@

    move_iterator operator+(difference_type n) const@\removed{;}@
      @\added{requires RandomAccessIterator<I>;}@
    move_iterator& operator+=(difference_type n)@\removed{;}@
      @\added{requires RandomAccessIterator<I>;}@
    move_iterator operator-(difference_type n) const@\removed{;}@
      @\added{requires RandomAccessIterator<I>;}@
    move_iterator& operator-=(difference_type n)@\removed{;}@
      @\added{requires RandomAccessIterator<I>;}@
    @\unspec@ operator[](difference_type n) const@\removed{;}@
      @\added{requires RandomAccessIterator<I>;}@

  private:
    @\changed{Iterator}{I}@ current;   // \expos
  };

  template <@\changed{class Iterator1}{InputIterator I1}@, @\changed{class Iterator2}{InputIterator I2}@>
      @\added{requires EqualityComparable<I1, I2>()}@
    bool operator==(
      const move_iterator<@\changed{Iterator1}{I1}@>& x, const move_iterator<@\changed{Iterator2}{I2}@>& y);
  template <@\changed{class Iterator1}{InputIterator I1}@, @\changed{class Iterator2}{InputIterator I2}@>
      @\added{requires EqualityComparable<I1, I2>()}@
    bool operator!=(
      const move_iterator<@\changed{Iterator1}{I1}@>& x, const move_iterator<@\changed{Iterator2}{I2}@>& y);
  template <@\changed{class Iterator1}{RandomAccessIterator I1}@, @\changed{class Iterator2}{RandomAccessIterator I2}@>
      @\added{requires TotallyOrdered<I1, I2>()}@
    bool operator<(
      const move_iterator<@\changed{Iterator1}{I1}@>& x, const move_iterator<@\changed{Iterator2}{I2}@>& y);
  template <@\changed{class Iterator1}{RandomAccessIterator I1}@, @\changed{class Iterator2}{RandomAccessIterator I2}@>
      @\added{requires TotallyOrdered<I1, I2>()}@
    bool operator<=(
      const move_iterator<@\changed{Iterator1}{I1}@>& x, const move_iterator<@\changed{Iterator2}{I2}@>& y);
  template <@\changed{class Iterator1}{RandomAccessIterator I1}@, @\changed{class Iterator2}{RandomAccessIterator I2}@>
      @\added{requires TotallyOrdered<I1, I2>()}@
    bool operator>(
      const move_iterator<@\changed{Iterator1}{I1}@>& x, const move_iterator<@\changed{Iterator2}{I2}@>& y);
  template <@\changed{class Iterator1}{RandomAccessIterator I1}@, @\changed{class Iterator2}{RandomAccessIterator I2}@>
      @\added{requires TotallyOrdered<I1, I2>()}@
    bool operator>=(
      const move_iterator<@\changed{Iterator1}{I1}@>& x, const move_iterator<@\changed{Iterator2}{I2}@>& y);

  template <@\changed{class Iterator1}{WeakInputIterator I1}@, @\changed{class Iterator2}{WeakInputIterator I2}@>
      @\added{requires SizedIteratorRange<I2, I1>}@
    @\changed{auto}{DifferenceType<I2>}@ operator-(
      const move_iterator<@\changed{Iterator1}{I1}@>& x,
      const move_iterator<@\changed{Iterator2}{I2}@>& y)@\removed{ ->decltype(y.base() - x.base())}@;
  template <@\changed{class Iterator}{RandomAccessIterator I}@>
    move_iterator<@\changed{Iterator}{I}@>
      operator+(
        @\changed{typename move_iterator<Iterator>::difference_type}{DifferenceType<I>}@ n,
        const move_iterator<@\changed{Iterator}{I}@>& x);
  template <@\changed{class Iterator}{WeakInputIterator I}@>
    move_iterator<@\changed{Iterator}{I}@> make_move_iterator(@\changed{Iterator}{I}@ i);
}
\end{codeblock}

\begin{removedblock}
\rSec3[move.iter.requirements]{\tcode{move_iterator} requirements}

\pnum
The template parameter \tcode{Iterator} shall meet
the requirements for an Input Iterator~(\ref{input.iterators}).
Additionally, if any of the bidirectional or random access traversal
functions are instantiated, the template parameter shall meet the
requirements for a Bidirectional Iterator~(\ref{bidirectional.iterators})
or a Random Access Iterator~(\ref{random.access.iterators}), respectively.
\end{removedblock}

\rSec3[move.iter.ops]{\tcode{move_iterator} operations}

\rSec4[move.iter.op.const]{\tcode{move_iterator} constructors}

\indexlibrary{\idxcode{move_iterator}!\idxcode{move_iterator}}%
\begin{itemdecl}
move_iterator()@\added{ = default}@;
\end{itemdecl}

\begin{itemdescr}
\pnum
\effects Constructs a \tcode{move_iterator}, \changed{value}{default}
initializing \tcode{current}. Iterator operations applied to the resulting
iterator have defined behavior if and only if the corresponding operations are defined
on a \changed{value}{default}-initialized iterator of type \tcode{\changed{Iterator}{I}}.
\end{itemdescr}


\indexlibrary{\idxcode{move_iterator}!constructor}%
\begin{itemdecl}
explicit move_iterator(@\changed{Iterator}{I}@ i);
\end{itemdecl}

\begin{itemdescr}
\pnum
\effects Constructs a \tcode{move_iterator}, initializing
\tcode{current} with \tcode{i}.
\end{itemdescr}


\indexlibrary{\idxcode{move_iterator}!constructor}%
\begin{itemdecl}
template <@\changed{class}{WeakInputIterator}@ U>
  @\added{requires Convertible<U, I>}@
move_iterator(const move_iterator<U>& u);
\end{itemdecl}

\begin{itemdescr}
\pnum
\effects Constructs a \tcode{move_iterator}, initializing
\tcode{current} with \tcode{u.base()}.

\begin{removedblock}
\pnum
\requires \tcode{U} shall be convertible to
\tcode{Iterator}.
\end{removedblock}
\end{itemdescr}

\rSec4[move.iter.op=]{\tcode{move_iterator::operator=}}

\indexlibrary{\idxcode{operator=}!\idxcode{move_iterator}}%
\indexlibrary{\idxcode{move_iterator}!\idxcode{operator=}}%
\begin{itemdecl}
template <@\changed{class}{WeakInputIterator}@ U>
  @\added{requires Convertible<U, I>}@
move_iterator& operator=(const move_iterator<U>& u);
\end{itemdecl}

\begin{itemdescr}
\pnum
\effects Assigns \tcode{u.base()} to
\tcode{current}.

\begin{removedblock}
\pnum
\requires \tcode{U} shall be convertible to
\tcode{Iterator}.
\end{removedblock}
\end{itemdescr}

\rSec4[move.iter.op.conv]{\tcode{move_iterator} conversion}

\indexlibrary{\idxcode{base}!\idxcode{move_iterator}}%
\indexlibrary{\idxcode{move_iterator}!\idxcode{base}}%
\begin{itemdecl}
@\changed{Iterator}{I} base() const;
\end{itemdecl}

\begin{itemdescr}
\pnum
\returns \tcode{current}.
\end{itemdescr}

\rSec4[move.iter.op.star]{\tcode{move_iterator::operator*}}

\indexlibrary{\idxcode{operator*}!\idxcode{move_iterator}}%
\indexlibrary{\idxcode{move_iterator}!\idxcode{operator*}}%
\begin{itemdecl}
reference operator*() const;
\end{itemdecl}

\begin{itemdescr}
\pnum
\returns \tcode{std::move(*current)}.
\end{itemdescr}

\begin{removedblock}
\rSec4[move.iter.op.ref]{\tcode{move_iterator::operator->}}

\indexlibrary{\idxcode{operator->}!\idxcode{move_iterator}}%
\indexlibrary{\idxcode{move_iterator}!\idxcode{operator->}}%
\begin{itemdecl}
pointer operator->() const;
\end{itemdecl}

\begin{itemdescr}
\pnum
\returns \tcode{current}.
\end{itemdescr}
\end{removedblock}

\rSec4[move.iter.op.incr]{\tcode{move_iterator::operator++}}

\indexlibrary{\idxcode{operator++}!\idxcode{move_iterator}}%
\indexlibrary{\idxcode{move_iterator}!\idxcode{operator++}}%
\begin{itemdecl}
move_iterator& operator++();
\end{itemdecl}

\begin{itemdescr}
\pnum
\effects \tcode{++current}.

\pnum
\returns \tcode{*this}.
\end{itemdescr}

\indexlibrary{\idxcode{operator++}!\idxcode{move_iterator}}%
\indexlibrary{\idxcode{move_iterator}!\idxcode{operator++}}%
\begin{itemdecl}
move_iterator operator++(int);
\end{itemdecl}

\begin{itemdescr}
\pnum
\effects
\begin{codeblock}
move_iterator tmp = *this;
++current;
return tmp;
\end{codeblock}
\end{itemdescr}

\rSec4[move.iter.op.decr]{\tcode{move_iterator::operator-{-}}}

\indexlibrary{\idxcode{operator\dcr}!\idxcode{move_iterator}}%
\indexlibrary{\idxcode{move_iterator}!\idxcode{operator\dcr}}%
\begin{itemdecl}
move_iterator& operator--()@\removed{;}@
  @\added{requires BidirectionalIterator<I>;}@
\end{itemdecl}

\begin{itemdescr}
\pnum
\effects \tcode{\dcr{}current}.

\pnum
\returns \tcode{*this}.
\end{itemdescr}

\indexlibrary{\idxcode{operator\dcr}!\idxcode{move_iterator}}%
\indexlibrary{\idxcode{move_iterator}!\idxcode{operator\dcr}}%
\begin{itemdecl}
move_iterator operator--(int)@\removed{;}@
  @\added{requires BidirectionalIterator<I>;}@
\end{itemdecl}

\begin{itemdescr}
\pnum
\effects
\begin{codeblock}
move_iterator tmp = *this;
--current;
return tmp;
\end{codeblock}
\end{itemdescr}

\rSec4[move.iter.op.+]{\tcode{move_iterator::operator+}}

\indexlibrary{\idxcode{operator+}!\idxcode{move_iterator}}%
\indexlibrary{\idxcode{move_iterator}!\idxcode{operator+}}%
\begin{itemdecl}
move_iterator operator+(difference_type n) const@\removed{;}@
  @\added{requires RandomAccessIterator<I>;}@
\end{itemdecl}

\begin{itemdescr}
\pnum
\returns \tcode{move_iterator(current + n)}.
\end{itemdescr}

\rSec4[move.iter.op.+=]{\tcode{move_iterator::operator+=}}

\indexlibrary{\idxcode{operator+=}!\idxcode{move_iterator}}%
\indexlibrary{\idxcode{move_iterator}!\idxcode{operator+=}}%
\begin{itemdecl}
move_iterator& operator+=(difference_type n)@\removed{;}@
  @\added{requires RandomAccessIterator<I>;}@
\end{itemdecl}

\begin{itemdescr}
\pnum
\effects \tcode{current += n}.

\pnum
\returns \tcode{*this}.
\end{itemdescr}

\rSec4[move.iter.op.-]{\tcode{move_iterator::operator-}}

\indexlibrary{\idxcode{operator-}!\idxcode{move_iterator}}%
\indexlibrary{\idxcode{move_iterator}!\idxcode{operator-}}%
\begin{itemdecl}
move_iterator operator-(difference_type n) const@\removed{;}@
  @\added{requires RandomAccessIterator<I>;}@
\end{itemdecl}

\begin{itemdescr}
\pnum
\returns \tcode{move_iterator(current - n)}.
\end{itemdescr}

\rSec4[move.iter.op.-=]{\tcode{move_iterator::operator-=}}

\indexlibrary{\idxcode{operator-=}!\idxcode{move_iterator}}%
\indexlibrary{\idxcode{move_iterator}!\idxcode{operator-=}}%
\begin{itemdecl}
move_iterator& operator-=(difference_type n)@\removed{;}@
  @\added{requires RandomAccessIterator<I>;}@
\end{itemdecl}

\begin{itemdescr}
\pnum
\effects \tcode{current -= n}.

\pnum
\returns \tcode{*this}.
\end{itemdescr}

\rSec4[move.iter.op.index]{\tcode{move_iterator::operator[]}}

\indexlibrary{\idxcode{operator[]}!\idxcode{move_iterator}}%
\indexlibrary{\idxcode{move_iterator}!\idxcode{operator[]}}%
\begin{itemdecl}
@\unspec@ operator[](difference_type n) const@\removed{;}@
  @\added{requires RandomAccessIterator<I>;}@
\end{itemdecl}

\begin{itemdescr}
\pnum
\returns \tcode{std::move(current[n])}.
\end{itemdescr}

\rSec4[move.iter.op.comp]{\tcode{move_iterator} comparisons}

\indexlibrary{\idxcode{operator==}!\idxcode{move_iterator}}%
\indexlibrary{\idxcode{move_iterator}!\idxcode{operator==}}%
\begin{itemdecl}
template <@\changed{class Iterator1}{InputIterator I1}@, @\changed{class Iterator2}{InputIterator I2}@>
    @\added{requires EqualityComparable<I1, I2>()}@
  bool operator==(
    const move_iterator<@\changed{Iterator1}{I1}@>& x, const move_iterator<@\changed{Iterator2}{I2}@>& y);
\end{itemdecl}

\begin{itemdescr}
\pnum
\returns \tcode{x.base() == y.base()}.
\end{itemdescr}

\indexlibrary{\idxcode{operator"!=}!\idxcode{move_iterator}}%
\indexlibrary{\idxcode{move_iterator}!\idxcode{operator"!=}}%
\begin{itemdecl}
template <@\changed{class Iterator1}{InputIterator I1}@, @\changed{class Iterator2}{InputIterator I2}@>
    @\added{requires EqualityComparable<I1, I2>()}@
  bool operator!=(
    const move_iterator<@\changed{Iterator1}{I1}@>& x, const move_iterator<@\changed{Iterator2}{I2}@>& y);
\end{itemdecl}

\begin{itemdescr}
\pnum
\returns \tcode{!(x == y)}.
\end{itemdescr}

\indexlibrary{\idxcode{operator<}!\idxcode{move_iterator}}%
\indexlibrary{\idxcode{move_iterator}!\idxcode{operator<}}%
\begin{itemdecl}
template <@\changed{class Iterator1}{RandomAccessIterator I1}@, @\changed{class Iterator2}{RandomAccessIterator I2}@>
    @\added{requires TotallyOrdered<I1, I2>()}@
  bool operator<(
    const move_iterator<@\changed{Iterator1}{I1}@>& x, const move_iterator<@\changed{Iterator2}{I2}@>& y);
\end{itemdecl}

\begin{itemdescr}
\pnum
\returns \tcode{x.base() < y.base()}.
\end{itemdescr}

\indexlibrary{\idxcode{operator<=}!\idxcode{move_iterator}}%
\indexlibrary{\idxcode{move_iterator}!\idxcode{operator<=}}%
\begin{itemdecl}
template <@\changed{class Iterator1}{RandomAccessIterator I1}@, @\changed{class Iterator2}{RandomAccessIterator I2}@>
    @\added{requires TotallyOrdered<I1, I2>()}@
  bool operator<=(
    const move_iterator<@\changed{Iterator1}{I1}@>& x, const move_iterator<@\changed{Iterator2}{I2}@>& y);
\end{itemdecl}

\begin{itemdescr}
\pnum
\returns \tcode{!(y < x)}.
\end{itemdescr}

\indexlibrary{\idxcode{operator>}!\idxcode{move_iterator}}%
\indexlibrary{\idxcode{move_iterator}!\idxcode{operator>}}%
\begin{itemdecl}
template <@\changed{class Iterator1}{RandomAccessIterator I1}@, @\changed{class Iterator2}{RandomAccessIterator I2}@>
    @\added{requires TotallyOrdered<I1, I2>()}@
  bool operator>(
    const move_iterator<@\changed{Iterator1}{I1}@>& x, const move_iterator<@\changed{Iterator2}{I2}@>& y);
\end{itemdecl}

\begin{itemdescr}
\pnum
\returns \tcode{y < x}.
\end{itemdescr}

\indexlibrary{\idxcode{operator>=}!\idxcode{move_iterator}}%
\indexlibrary{\idxcode{move_iterator}!\idxcode{operator>=}}%
\begin{itemdecl}
template <@\changed{class Iterator1}{RandomAccessIterator I1}@, @\changed{class Iterator2}{RandomAccessIterator I2}@>
    @\added{requires TotallyOrdered<I1, I2>()}@
  bool operator>=(
    const move_iterator<@\changed{Iterator1}{I1}@>& x, const move_iterator<@\changed{Iterator2}{I2}@>& y);
\end{itemdecl}

\begin{itemdescr}
\pnum
\returns \tcode{!(x < y)}.
\end{itemdescr}

\rSec4[move.iter.nonmember]{\tcode{move_iterator} non-member functions}

\indexlibrary{\idxcode{operator-}!\idxcode{move_iterator}}%
\indexlibrary{\idxcode{move_iterator}!\idxcode{operator-}}%
\begin{itemdecl}
template <@\changed{class Iterator1}{WeakInputIterator I1}@, @\changed{class Iterator2}{WeakInputIterator I2}@>
    @\added{requires SizedIteratorRange<I2, I1>}@
  @\changed{auto}{DifferenceType<I2>}@ operator-(
    const move_iterator<@\changed{Iterator1}{I1}@>& x,
    const move_iterator<@\changed{Iterator2}{I2}@>& y)@\removed{ ->decltype(y.base() - x.base())}@;
\end{itemdecl}

\begin{itemdescr}
\pnum
\returns \tcode{x.base() - y.base()}.
\end{itemdescr}

\indexlibrary{\idxcode{operator+}!\idxcode{move_iterator}}%
\indexlibrary{\idxcode{move_iterator}!\idxcode{operator+}}%
\begin{itemdecl}
template <@\changed{class Iterator}{RandomAccessIterator I}@>
  move_iterator<@\changed{Iterator}{I}@>
    operator+(
      @\changed{typename move_iterator<Iterator>::difference_type}{DifferenceType<I>}@ n,
      const move_iterator<@\changed{Iterator}{I}@>& x);
\end{itemdecl}

\begin{itemdescr}
\pnum
\returns \tcode{x + n}.
\end{itemdescr}

\indexlibrary{\idxcode{make_move_iterator}}%
\begin{itemdecl}
template <@\changed{class Iterator}{WeakInputIterator I}@>
  move_iterator<@\changed{Iterator}{I}@> make_move_iterator(@\changed{Iterator}{I}@ i);
\end{itemdecl}

\begin{itemdescr}
\pnum
\returns \tcode{move_iterator<\changed{Iterator}{I}>(i)}.
\end{itemdescr}

\begin{addedblock}

\rSec2[common.iterators]{Common iterators}

\pnum
Class template \tcode{common_iterator} is an iterator/sentinel adaptor that is
capable of representing a non-bounded range of elements (where the types of the
iterator and sentinel differ) as a bounded range (where they are the same). It
does this by holding either an iterator or a sentinel, and implementing the
equality comparison operators appropriately.

\pnum
\enternote The \tcode{common_iterator} type is useful for interfacing with legacy
code that expects the begin and end of a range to have the same type, and for
use in \tcode{common_type} specializations that are required to make
iterator/sentinel pairs model the \tcode{EqualityComparable} concept.\exitnote

\pnum
\enterexample
\begin{codeblock}
template<class ForwardIterator>
void fun(ForwardIterator begin, ForwardIterator end);

list<int> s;
// populate the list \tcode{s}
using CI =
  common_iterator<counted_iterator<list<int>::iterator>,
                  counted_sentinel>;
// call \tcode{fun} on a range of 10 ints
fun(CI(make_counted_iterator(s.begin(), 10)),
    CI(counted_sentinel()));
\end{codeblock}
\exitexample

\rSec3[common.iterator]{Class template \tcode{common_iterator}}

\indexlibrary{\idxcode{counted_iterator}}%
\begin{codeblock}
namespace std {
  // \expos
  template<typename A, typename B>
  concept bool WeaklyEqualityComparable =
    EqualityComparable<A>() && EqualityComparable<B>() &&
    requires(A a, B b) {
      {a==b} -> bool;
      {a!=b} -> bool;
      {b==a} -> bool;
      {b!=a} -> bool;
    };
  // \expos
  template<Iterator I, Regular S>
  concept bool WeakSentinel =
    WeaklyEqualityComparable<I, S>;

  template <InputIterator I, WeakSentinel<I> S>
    requires !Same<I, S>
  class common_iterator {
  public:
    using difference_type = DifferenceType<I>;
    using value_type = ValueType<I>;
    using iterator_category =
      conditional_t<ForwardIterator<I>,
                    std::forward_iterator_tag,
                    std::input_iterator_tag>;
    using reference = ReferenceType<I>;

    common_iterator();
    common_iterator(I i);
    common_iterator(S s);
    template <InputIterator U, WeakSentinel<U> V>
      requires Convertible<U, I> && Convertible<V, S>
    common_iterator(const common_iterator<U, V>& u);
    template <InputIterator U, WeakSentinel<U> V>
      requires Convertible<U, I> && Convertible<V, S>
    common_iterator& operator=(const common_iterator<U, V>& u);

    ~common_iterator();

    reference operator*() const;

    common_iterator& operator++();
    common_iterator operator++(int);

  private:
    bool is_sentinel; // \expos
    I iter;           // \expos
    S sent;           // \expos
  };

  template <InputIterator I1, WeakSentinel<I1> S1,
            InputIterator I2, WeakSentinel<I2> S2>
    requires EqualityComparable<I1, I2>() && WeaklyEqualityComparable<I1, S2> &&
      WeaklyEqualityComparable<I2, S1>
  bool operator==(
    const common_iterator<I1, S1>& x, const common_iterator<I2, S2>& y);
  template <InputIterator I1, WeakSentinel<I1> S1,
            InputIterator I2, WeakSentinel<I2> S2>
    requires EqualityComparable<I1, I2>() && WeaklyEqualityComparable<I1, S2> &&
      WeaklyEqualityComparable<I2, S1>
  bool operator!=(
    const common_iterator<I1, S1>& x, const common_iterator<I2, S2>& y);

  template <InputIterator I1, WeakSentinel<I1> S1,
            InputIterator I2, WeakSentinel<I2> S2>
    requires SizedIteratorRange<I1, I1> && SizedIteratorRange<I2, I2> &&
      requires (I1 a, I2 b) { {a-b}->DifferenceType<I2>; {b-a}->DifferenceType<I2>; }
      requires (I1 i, S2 s) { {i-s}->DifferenceType<I2>; {s-i}->DifferenceType<I2>; }
      requires (I2 i, S1 s) { {i-s}->DifferenceType<I2>; {s-i}->DifferenceType<I2>; }
  DifferenceType<I2> operator-(
    const common_iterator<I1, S1>& x, const common_iterator<I2, S2>& y);
}
\end{codeblock}

\pnum
\enternote The use of the expository \tcode{WeaklyEqualityComparable} and
\tcode{WeakSentinel} concepts is avoid the self-referential requirements that
would happen if parameters \tcode{I} and \tcode{S} use \tcode{common_iterator<I, S>}
as their common type.\exitnote

\pnum
\enternote The ad hoc constraints on \tcode{common_iterator}'s \tcode{operator-}
exist for the same reason.\exitnote

\pnum
\enternote It is unspecified whether \tcode{common_iterator}'s members
\tcode{iter} and \tcode{sent} have distinct addresses or not.\exitnote

\rSec3[common.iter.ops]{\tcode{common_iterator} operations}

\rSec4[common.iter.op.const]{\tcode{common_iterator} constructors}

\indexlibrary{\idxcode{common_iterator}!\idxcode{common_iterator}}%
\begin{itemdecl}
common_iterator();
\end{itemdecl}

\begin{itemdescr}
\pnum
\effects Constructs a \tcode{common_iterator}, value-initializing \tcode{is_sentinel}
and \tcode{iter}. It is unspecified whether any initialization is performed for
\tcode{sent}. Iterator operations applied to the resulting iterator have defined
behavior if and only if the corresponding operations are defined on a
value-initialized iterator of type \tcode{I}.
\end{itemdescr}

\indexlibrary{\idxcode{common_iterator}!constructor}%
\begin{itemdecl}
common_iterator(I i);
\end{itemdecl}

\begin{itemdescr}
\pnum
\effects Constructs a \tcode{common_iterator}, initializing
\tcode{is_sentinel} with \tcode{false} and \tcode{iter} with \tcode{i}. It is
unspecified whether any initialization is performed for \tcode{sent}.
\end{itemdescr}

\indexlibrary{\idxcode{common_iterator}!constructor}%
\begin{itemdecl}
common_iterator(S s);
\end{itemdecl}

\begin{itemdescr}
\pnum
\effects Constructs a \tcode{common_iterator}, initializing
\tcode{is_sentinel} with \tcode{true} and \tcode{sent} with \tcode{s}. It is
unspecified whether any initialization is performed for \tcode{iter}.
\end{itemdescr}

\indexlibrary{\idxcode{common_iterator}!constructor}%
\begin{itemdecl}
template <InputIterator U, WeakSentinel<U> V>
  requires Convertible<U, I> && Convertible<V, S>
common_iterator(const common_iterator<U, V>& u);
\end{itemdecl}

\begin{itemdescr}
\pnum
\effects Constructs a \tcode{common_iterator}, initializing
\tcode{is_sentinel} with \tcode{u.is_sentinel}.
\begin{itemize}
\item If \tcode{u.is_sentinel} is true, \tcode{sent} is initialized with \tcode{u.sent}.
It is unspecified whether any initialization is performed for \tcode{iter}.
\item If \tcode{u.is_sentinel} is false, \tcode{iter} is initialized with \tcode{u.iter}.
It is unspecified whether any initialization is performed for \tcode{sent}.
\end{itemize}
\end{itemdescr}

\rSec4[common.iter.op=]{\tcode{common_iterator::operator=}}

\indexlibrary{\idxcode{operator=}!\idxcode{common_iterator}}%
\indexlibrary{\idxcode{common_iterator}!\idxcode{operator=}}%
\begin{itemdecl}
template <InputIterator U, WeakSentinel<U> V>
  requires Convertible<U, I> && Convertible<V, S>
common_iterator& operator=(const common_iterator<U, V>& u);
\end{itemdecl}

\begin{itemdescr}
\pnum
\effects Assigns \tcode{u.is_sentinel} to \tcode{is_sentinel}.
\begin{itemize}
\item If \tcode{u.is_sentinel} is true, assigns \tcode{u.sent} to \tcode{sent}.
It is unspecified whether any operation is performed on \tcode{iter}.
\item If \tcode{u.is_sentinel} is false, assigns \tcode{u.iter} to \tcode{iter}.
It is unspecified whether any operation is performed on \tcode{sent}.
\end{itemize}

\pnum
\returns \tcode{*this}
\end{itemdescr}

\indexlibrary{\idxcode{common_iterator}!destructor}%
\begin{itemdecl}
~common_iterator();
\end{itemdecl}

\begin{itemdescr}
\pnum
\effects
Runs the destructor(s) for any members that are currently initialized.
\end{itemdescr}

\rSec4[common.iter.op.star]{\tcode{common_iterator::operator*}}

\indexlibrary{\idxcode{operator*}!\idxcode{common_iterator}}%
\indexlibrary{\idxcode{common_iterator}!\idxcode{operator*}}%
\begin{itemdecl}
reference operator*() const;
\end{itemdecl}

\begin{itemdescr}
\pnum
\requires \tcode{!is_sentinel}

\pnum
\returns \tcode{*iter}.
\end{itemdescr}

\rSec4[common.iter.op.incr]{\tcode{common_iterator::operator++}}

\indexlibrary{\idxcode{operator++}!\idxcode{common_iterator}}%
\indexlibrary{\idxcode{common_iterator}!\idxcode{operator++}}%
\begin{itemdecl}
common_iterator& operator++();
\end{itemdecl}

\begin{itemdescr}
\pnum
\requires \tcode{!is_sentinel}

\pnum
\effects \tcode{++iter}.

\pnum
\returns \tcode{*this}.
\end{itemdescr}

\indexlibrary{\idxcode{operator++}!\idxcode{common_iterator}}%
\indexlibrary{\idxcode{common_iterator}!\idxcode{operator++}}%
\begin{itemdecl}
common_iterator operator++(int);
\end{itemdecl}

\begin{itemdescr}
\pnum
\requires \tcode{!is_sentinel}

\pnum
\effects
\begin{codeblock}
common_iterator tmp = *this;
++iter;
return tmp;
\end{codeblock}
\end{itemdescr}

\rSec4[common.iter.op.comp]{\tcode{common_iterator} comparisons}

\indexlibrary{\idxcode{operator==}!\idxcode{common_iterator}}%
\indexlibrary{\idxcode{common_iterator}!\idxcode{operator==}}%
\begin{itemdecl}
template <InputIterator I1, WeakSentinel<I1> S1,
          InputIterator I2, WeakSentinel<I2> S2>
  requires EqualityComparable<I1, I2>() && WeaklyEqualityComparable<I1, S2> &&
    WeaklyEqualityComparable<I2, S1>
bool operator==(
  const common_iterator<I1, S1>& x, const common_iterator<I2, S2>& y);
\end{itemdecl}

\begin{itemdescr}
\pnum
\returns
\begin{codeblock}
x.is_sentinel ?
    (y.is_sentinel || y.iter == x.sent) :
    (y.is_sentinel ?
        x.iter == y.sent :
        x.iter == y.iter;
\end{codeblock}
\end{itemdescr}

\indexlibrary{\idxcode{operator"!=}!\idxcode{common_iterator}}%
\indexlibrary{\idxcode{common_iterator}!\idxcode{operator"!=}}%
\begin{itemdecl}
template <InputIterator I1, WeakSentinel<I1> S1,
          InputIterator I2, WeakSentinel<I2> S2>
  requires EqualityComparable<I1, I2>() && WeaklyEqualityComparable<I1, S2> &&
    WeaklyEqualityComparable<I2, S1>
bool operator!=(
  const common_iterator<I1, S1>& x, const common_iterator<I2, S2>& y);
\end{itemdecl}

\begin{itemdescr}
\pnum
\returns \tcode{!(x == y)}.
\end{itemdescr}

\indexlibrary{\idxcode{operator-}!\idxcode{common_iterator}}%
\indexlibrary{\idxcode{common_iterator}!\idxcode{operator-}}%
\begin{itemdecl}
template <InputIterator I1, WeakSentinel<I1> S1,
          InputIterator I2, WeakSentinel<I2> S2>
  requires SizedIteratorRange<I1, I1> && SizedIteratorRange<I2, I2> &&
    requires (I1 a, I2 b) { {a-b}->DifferenceType<I2>; {b-a}->DifferenceType<I2>; }
    requires (I1 i, S2 s) { {i-s}->DifferenceType<I2>; {s-i}->DifferenceType<I2>; }
    requires (I2 i, S1 s) { {i-s}->DifferenceType<I2>; {s-i}->DifferenceType<I2>; }
DifferenceType<I2> operator-(
  const common_iterator<I1, S1>& x, const common_iterator<I2, S2>& y);
\end{itemdecl}

\begin{itemdescr}
\pnum
\returns
\begin{codeblock}
x.is_sentinel ?
    (y.is_sentinel ? 0 : x.sent - y.iter) :
    (y.is_sentinel ?
         x.iter - y.sent :
         x.iter - y.iter;
\end{codeblock}
\end{itemdescr}

\rSec2[counted.iterators]{Counted iterators and sentinels}

\pnum
Class template \tcode{counted_iterator} is an iterator adaptor
with the same behavior as the underlying iterator except that it
keeps track of its distance from its starting position. It can be
used together with class \tcode{counted_sentinel} in calls to generic
algorithms to operate on a range of $N$ elements starting at a given
position without needing to know the end position \textit{a priori}.

\pnum
\enterexample

\begin{codeblock}
list<string> s;
// populate the list \tcode{s}
vector<string> v(make_counted_iterator(s.begin(), 10),
                 counted_sentinel()); // copies 10 strings into \tcode{v}
\end{codeblock}

\exitexample

\rSec3[counted.iterator]{Class template \tcode{counted_iterator}}

\indexlibrary{\idxcode{counted_iterator}}%
\begin{codeblock}
namespace std {
  template <WeakInputIterator I>
  class counted_iterator {
  public:
    using iterator_type = I;
    using difference_type = DifferenceType<I>;
    using value_type = ValueType<I>;
    using iterator_category =
      conditional_t<ForwardIterator<I>,
                    IteratorCategory<I>,
                    std::input_iterator_tag>;
    using reference = ReferenceType<I>;

    counted_iterator() = default;
    counted_iterator(I x, DifferenceType<I> n);
    template <WeakInputIterator U>
      requires Convertible<U, I>
    counted_iterator(const counted_iterator<U>& u);
    template <WeakInputIterator U>
      requires Convertible<U, I>
    counted_iterator& operator=(const counted_iterator<U>& u);

    I base() const;
    DifferenceType<I> count() const;
    reference operator*() const;

    counted_iterator& operator++();
    counted_iterator operator++(int);
    counted_iterator& operator--()
      requires BidirectionalIterator<I>;
    counted_iterator operator--(int)
      requires BidirectionalIterator<I>;


    counted_iterator  operator+ (difference_type n) const
      requires RandomAccessIterator<I>;
    counted_iterator& operator+=(difference_type n)
      requires RandomAccessIterator<I>;
    counted_iterator  operator- (difference_type n) const
      requires RandomAccessIterator<I>;
    counted_iterator& operator-=(difference_type n)
      requires RandomAccessIterator<I>;
    @\unspec@ operator[](difference_type n) const
      requires RandomAccessIterator<I>;
  protected:
    I current;
    DifferenceType<I> cnt;
  };

  template <WeakInputIterator I1, WeakInputIterator I2>
    bool operator==(
      const counted_iterator<I1>& x, const counted_iterator<I2>& y);
  template <WeakInputIterator I>
    bool operator==(
      const counted_iterator<I>& x, counted_sentinel y);
  template <WeakInputIterator I>
    bool operator==(
      counted_sentinel x, const counted_iterator<I>& y);
  bool operator==(counted_sentinel x, counted_sentinel y);
  template <WeakInputIterator I1, WeakInputIterator I2>
    bool operator!=(
      const counted_iterator<I1>& x, const counted_iterator<I2>& y);
  template <WeakInputIterator I>
    bool operator!=(
      const counted_iterator<I>& x, counted_sentinel y);
  template <WeakInputIterator I>
    bool operator!=(
      counted_sentinel x, const counted_iterator<I>& y);
  bool operator!=(counted_sentinel x, counted_sentinel y);

  template <RandomAccessIterator I1, RandomAccessIterator I2>
      requires TotallyOrdered<I1, I2>()
    bool operator<(
      const counted_iterator<I1>& x, const counted_iterator<I2>& y);
  template <RandomAccessIterator I>
    bool operator<(
      const counted_iterator<I>& x, counted_sentinel y);
  template <RandomAccessIterator I>
    bool operator<(
      counted_sentinel x, const counted_iterator<I>& y);
  bool operator<(counted_sentinel x, counted_sentinel y);
  template <RandomAccessIterator I1, RandomAccessIterator I2>
      requires TotallyOrdered<I1, I2>()
    bool operator<=(
      const counted_iterator<I1>& x, const counted_iterator<I2>& y);
  template <RandomAccessIterator I>
    bool operator<=(
      const counted_iterator<I>& x, counted_sentinel y);
  template <RandomAccessIterator I>
    bool operator<=(
      counted_sentinel x, const counted_iterator<I>& y);
  bool operator<=(counted_sentinel x, counted_sentinel y);
  template <RandomAccessIterator I1, RandomAccessIterator I2>
      requires TotallyOrdered<I1, I2>()
    bool operator>(
      const counted_iterator<I1>& x, const counted_iterator<I2>& y);
  template <RandomAccessIterator I>
    bool operator>(
      const counted_iterator<I>& x, counted_sentinel y);
  template <RandomAccessIterator I>
    bool operator>(
      counted_sentinel x, const counted_iterator<I>& y);
  bool operator>(counted_sentinel x, counted_sentinel y);
  template <RandomAccessIterator I1, RandomAccessIterator I2>
      requires TotallyOrdered<I1, I2>()
    bool operator>=(
      const counted_iterator<I1>& x, const counted_iterator<I2>& y);
  template <RandomAccessIterator I>
    bool operator>=(
      const counted_iterator<I>& x, counted_sentinel y);
  template <RandomAccessIterator I>
    bool operator>=(
      counted_sentinel x, const counted_iterator<I>& y);
  bool operator>=(counted_sentinel x, counted_sentinel y);

  template <WeakInputIterator I1, WeakInputIterator I2>
    DifferenceType<I2> operator-(
      const counted_iterator<I1>& x, const counted_iterator<I2>& y);
  template <WeakInputIterator I>
    DifferenceType<I> operator-(
      const counted_iterator<I>& x, counted_sentinel y);
  template <WeakInputIterator I>
    DifferenceType<I> operator-(
      counted_sentinel x, const counted_iterator<I>& y);
  ptrdiff_t operator-(counted_sentinel x, counted_sentinel y);
  template <RandomAccessIterator I>
    counted_iterator<I>
      operator+(DifferenceType<I> n, const counted_iterator<I>& x);
  template <WeakInputIterator I>
    counted_iterator<I> make_counted_iterator(I i, DifferenceType<I> n);

  template <WeakInputIterator I>
    void advance(counted_iterator<I>& i, DifferenceType<I> n);
}
\end{codeblock}

\rSec3[counted.iter.ops]{\tcode{counted_iterator} operations}

\rSec4[counted.iter.op.const]{\tcode{counted_iterator} constructors}

\indexlibrary{\idxcode{counted_iterator}!\idxcode{counted_iterator}}%
\begin{itemdecl}
counted_iterator() = default;
\end{itemdecl}

\begin{itemdescr}
\pnum
\effects Constructs a \tcode{counted_iterator}, default
initializing \tcode{current} and \tcode{cnt}. Iterator operations applied to the
resulting iterator have defined behavior if and only if the corresponding operations
are defined on a default-initialized iterator of type \tcode{I}.
\end{itemdescr}

\indexlibrary{\idxcode{counted_iterator}!constructor}%
\begin{itemdecl}
counted_iterator(I i, DifferenceType<I> n);
\end{itemdecl}

\begin{itemdescr}
\pnum
\requires \tcode{n >= 0}

\pnum
\effects Constructs a \tcode{counted_iterator}, initializing
\tcode{current} with \tcode{i} and \tcode{cnt} with \tcode{n}.
\end{itemdescr}

\indexlibrary{\idxcode{counted_iterator}!constructor}%
\begin{itemdecl}
template <WeakInputIterator U>
  requires Convertible<U, I>
count_iterator(const counted_iterator<U>& u);
\end{itemdecl}

\begin{itemdescr}
\pnum
\effects Constructs a \tcode{counted_iterator}, initializing
\tcode{current} with \tcode{u.base()} and \tcode{cnt} with \tcode{u.count()}.
\end{itemdescr}

\rSec4[counted.iter.op=]{\tcode{counted_iterator::operator=}}

\indexlibrary{\idxcode{operator=}!\idxcode{counted_iterator}}%
\indexlibrary{\idxcode{counted_iterator}!\idxcode{operator=}}%
\begin{itemdecl}
template <WeakInputIterator U>
  requires Convertible<U, I>
counted_iterator& operator=(const counted_iterator<U>& u);
\end{itemdecl}

\begin{itemdescr}
\pnum
\effects Assigns \tcode{u.base()} to
\tcode{current} and \tcode{u.count()} to \tcode{cnt}.

\end{itemdescr}

\rSec4[counted.iter.op.conv]{\tcode{counted_iterator} conversion}

\indexlibrary{\idxcode{base}!\idxcode{counted_iterator}}%
\indexlibrary{\idxcode{counted_iterator}!\idxcode{base}}%
\begin{itemdecl}
I base() const;
\end{itemdecl}

\begin{itemdescr}
\pnum
\returns \tcode{current}.
\end{itemdescr}

\rSec4[counted.iter.op.cnt]{\tcode{counted_iterator} count}

\indexlibrary{\idxcode{count}!\idxcode{counted_iterator}}%
\indexlibrary{\idxcode{counted_iterator}!\idxcode{count}}%
\begin{itemdecl}
DifferenceType<I> count() const;
\end{itemdecl}

\begin{itemdescr}
\pnum
\returns \tcode{cnt}.
\end{itemdescr}

\rSec4[counted.iter.op.star]{\tcode{count_iterator::operator*}}

\indexlibrary{\idxcode{operator*}!\idxcode{counted_iterator}}%
\indexlibrary{\idxcode{counted_iterator}!\idxcode{operator*}}%
\begin{itemdecl}
reference operator*() const;
\end{itemdecl}

\begin{itemdescr}
\pnum
\returns \tcode{*current}.
\end{itemdescr}

\rSec4[counted.iter.op.incr]{\tcode{counted_iterator::operator++}}

\indexlibrary{\idxcode{operator++}!\idxcode{counted_iterator}}%
\indexlibrary{\idxcode{counted_iterator}!\idxcode{operator++}}%
\begin{itemdecl}
counted_iterator& operator++();
\end{itemdecl}

\begin{itemdescr}
\pnum
\requires \tcode{cnt > 0}

\pnum
\effects
\begin{codeblock}
++current;
@\dcr@cnt;
\end{codeblock}

\pnum
\returns \tcode{*this}.
\end{itemdescr}

\indexlibrary{\idxcode{operator++}!\idxcode{counted_iterator}}%
\indexlibrary{\idxcode{counted_iterator}!\idxcode{operator++}}%
\begin{itemdecl}
counted_iterator operator++(int);
\end{itemdecl}

\begin{itemdescr}
\pnum
\requires \tcode{cnt > 0}

\pnum
\effects
\begin{codeblock}
counted_iterator tmp = *this;
++current;
@\dcr@cnt;
return tmp;
\end{codeblock}
\end{itemdescr}

\rSec4[counted.iter.op.decr]{\tcode{counted_iterator::operator-{-}}}

\indexlibrary{\idxcode{operator\dcr}!\idxcode{counted_iterator}}%
\indexlibrary{\idxcode{counted_iterator}!\idxcode{operator\dcr}}%
\begin{itemdecl}
  counted_iterator& operator--();
    requires BidirectionalIterator<I>
\end{itemdecl}

\begin{itemdescr}
\pnum
\effects
\begin{codeblock}
--current;
++cnt;
\end{codeblock}

\pnum
\returns \tcode{*this}.
\end{itemdescr}

\indexlibrary{\idxcode{operator\dcr}!\idxcode{counted_iterator}}%
\indexlibrary{\idxcode{counted_iterator}!\idxcode{operator\dcr}}%
\begin{itemdecl}
  counted_iterator operator--(int)
    requires BidirectionalIterator<I>;
\end{itemdecl}

\begin{itemdescr}
\pnum
\effects
\begin{codeblock}
counted_iterator tmp = *this;
--current;
++cnt;
return tmp;
\end{codeblock}
\end{itemdescr}

\rSec4[counted.iter.op.+]{\tcode{counted_iterator::operator+}}

\indexlibrary{\idxcode{operator+}!\idxcode{counted_iterator}}%
\indexlibrary{\idxcode{counted_iterator}!\idxcode{operator+}}%
\begin{itemdecl}
  counted_iterator operator+(difference_type n) const
    requires RandomAccessIterator<I>;
\end{itemdecl}

\begin{itemdescr}
\pnum
\requires \tcode{n <= cnt}

\pnum
\returns \tcode{counted_iterator(current + n, cnt - n)}.
\end{itemdescr}

\rSec4[counted.iter.op.+=]{\tcode{counted_iterator::operator+=}}

\indexlibrary{\idxcode{operator+=}!\idxcode{counted_iterator}}%
\indexlibrary{\idxcode{counted_iterator}!\idxcode{operator+=}}%
\begin{itemdecl}
  counted_iterator& operator+=(difference_type n)
    requires RandomAccessIterator<I>;
\end{itemdecl}

\begin{itemdescr}
\pnum
\requires \tcode{n <= cnt}

\pnum
\effects
\begin{codeblock}
current += n;
cnt -= n;
\end{codeblock}

\pnum
\returns \tcode{*this}.
\end{itemdescr}

\rSec4[counted.iter.op.-]{\tcode{counted_iterator::operator-}}

\indexlibrary{\idxcode{operator-}!\idxcode{counted_iterator}}%
\indexlibrary{\idxcode{counted_iterator}!\idxcode{operator-}}%
\begin{itemdecl}
  counted_iterator operator-(difference_type n) const
    requires RandomAccessIterator<I>;
\end{itemdecl}

\begin{itemdescr}
\pnum
\requires \tcode{-n <= cnt}

\pnum
\returns \tcode{counted_iterator(current - n, cnt + n)}.
\end{itemdescr}

\rSec4[counted.iter.op.-=]{\tcode{counted_iterator::operator-=}}

\indexlibrary{\idxcode{operator-=}!\idxcode{counted_iterator}}%
\indexlibrary{\idxcode{counted_iterator}!\idxcode{operator-=}}%
\begin{itemdecl}
  counted_iterator& operator-=(difference_type n)
    requires RandomAccessIterator<I>;
\end{itemdecl}

\begin{itemdescr}
\pnum
\requires \tcode{-n <= cnt}

\pnum
\effects
\begin{codeblock}
current -= n;
cnt += n;
\end{codeblock}

\pnum
\returns \tcode{*this}.
\end{itemdescr}

\rSec4[counted.iter.op.index]{\tcode{counted_iterator::operator[]}}

\indexlibrary{\idxcode{operator[]}!\idxcode{counted_iterator}}%
\indexlibrary{\idxcode{counted_iterator}!\idxcode{operator[]}}%
\begin{itemdecl}
  @\unspec@ operator[](difference_type n) const
    requires RandomAccessIterator<I>;
\end{itemdecl}

\begin{itemdescr}
\pnum
\requires \tcode{n <= cnt}

\pnum
\returns \tcode{current[n]}.
\end{itemdescr}

\rSec4[counted.iter.op.comp]{\tcode{counted_iterator} comparisons}

\indexlibrary{\idxcode{operator==}!\idxcode{counted_iterator}}%
\indexlibrary{\idxcode{counted_iterator}!\idxcode{operator==}}%
\begin{itemdecl}
template <WeakInputIterator I1, WeakInputIterator I2>
  bool operator==(
    const counted_iterator<I1>& x, const counted_iterator<I2>& y);
\end{itemdecl}

\begin{itemdescr}
\pnum
\returns \tcode{x.base() == y.base()} if \tcode{EqualityComparable<I1, I2>()};
  otherwise, \tcode{x.count() == y.count()}.
\end{itemdescr}

\begin{itemdecl}
template <WeakInputIterator I>
  bool operator==(
    const counted_iterator<I>& x, counted_sentinel y);
\end{itemdecl}

\begin{itemdescr}
\pnum
\returns \tcode{x.count() == 0}.
\end{itemdescr}

\begin{itemdecl}
template <WeakInputIterator I>
  bool operator==(
    counted_sentinel x, const counted_iterator<I>& y);
\end{itemdecl}

\begin{itemdescr}
\pnum
\returns \tcode{y.count() == 0}.
\end{itemdescr}

\begin{itemdecl}
bool operator==(counted_sentinel x, counted_sentinel y);
\end{itemdecl}

\begin{itemdescr}
\pnum
\returns \tcode{true}.
\end{itemdescr}

\indexlibrary{\idxcode{operator"!=}!\idxcode{counted_iterator}}%
\indexlibrary{\idxcode{counted_iterator}!\idxcode{operator"!=}}%
\begin{itemdecl}
template <WeakInputIterator I1, WeakInputIterator I2>
  bool operator!=(
    const counted_iterator<I1>& x, const counted_iterator<I2>& y);
template <WeakInputIterator I>
  bool operator!=(
    const counted_iterator<I>& x, counted_sentinel y);
template <WeakInputIterator I>
  bool operator!=(
    counted_sentinel x, const counted_iterator<I>& y);
bool operator!=(counted_sentinel x, counted_sentinel y);
\end{itemdecl}

\begin{itemdescr}
\pnum
\returns \tcode{!(x == y)}.
\end{itemdescr}

\indexlibrary{\idxcode{operator<}!\idxcode{counted_iterator}}%
\indexlibrary{\idxcode{counted_iterator}!\idxcode{operator<}}%
\begin{itemdecl}
template <RandomAccessIterator I1, RandomAccessIterator I2>
    requires TotallyOrdered<I1, I2>()
  bool operator<(
    const counted_iterator<I1>& x, const counted_iterator<I2>& y);
\end{itemdecl}

\begin{itemdescr}
\pnum
\returns \tcode{x.base() < y.base()}.
\end{itemdescr}

\begin{itemdecl}
template <RandomAccessIterator I>
  bool operator<(
    const counted_iterator<I>& x, counted_sentinel y);
\end{itemdecl}

\begin{itemdescr}
\pnum
\returns \tcode{x.count() != 0}.
\end{itemdescr}

\begin{itemdecl}
template <RandomAccessIterator I>
  bool operator<(
    counted_sentinel x, const counted_iterator<I>& y);
\end{itemdecl}

\begin{itemdescr}
\pnum
\returns \tcode{false}.
\end{itemdescr}

\begin{itemdecl}
bool operator<(counted_sentinel x, counted_sentinel y);
\end{itemdecl}

\begin{itemdescr}
\pnum
\returns \tcode{false}.
\end{itemdescr}

\indexlibrary{\idxcode{operator<=}!\idxcode{counted_iterator}}%
\indexlibrary{\idxcode{counted_iterator}!\idxcode{operator<=}}%
\begin{itemdecl}
template <RandomAccessIterator I1, RandomAccessIterator I2>
    requires TotallyOrdered<I1, I2>()
  bool operator<=(
    const counted_iterator<I1>& x, const counted_iterator<I2>& y);
template <RandomAccessIterator I>
  bool operator<=(
    const counted_iterator<I>& x, counted_sentinel y);
template <RandomAccessIterator I>
  bool operator<=(
    counted_sentinel x, const counted_iterator<I>& y);
bool operator<=(counted_sentinel x, counted_sentinel y);
\end{itemdecl}

\begin{itemdescr}
\pnum
\returns \tcode{!(y < x)}.
\end{itemdescr}

\indexlibrary{\idxcode{operator>}!\idxcode{counted_iterator}}%
\indexlibrary{\idxcode{counted_iterator}!\idxcode{operator>}}%
\begin{itemdecl}
template <RandomAccessIterator I1, RandomAccessIterator I2>
    requires TotallyOrdered<I1, I2>()
  bool operator>(
    const counted_iterator<I1>& x, const counted_iterator<I2>& y);
template <RandomAccessIterator I>
  bool operator>(
    const counted_iterator<I>& x, counted_sentinel y);
template <RandomAccessIterator I>
  bool operator>(
    counted_sentinel x, const counted_iterator<I>& y);
bool operator>(counted_sentinel x, counted_sentinel y);
\end{itemdecl}

\begin{itemdescr}
\pnum
\returns \tcode{y < x}.
\end{itemdescr}

\indexlibrary{\idxcode{operator>=}!\idxcode{counted_iterator}}%
\indexlibrary{\idxcode{counted_iterator}!\idxcode{operator>=}}%
\begin{itemdecl}
template <RandomAccessIterator I1, RandomAccessIterator I2>
    requires TotallyOrdered<I1, I2>()
  bool operator>=(
    const counted_iterator<I1>& x, const counted_iterator<I2>& y);
template <RandomAccessIterator I>
  bool operator>=(
    const counted_iterator<I>& x, counted_sentinel y);
template <RandomAccessIterator I>
  bool operator>=(
    counted_sentinel x, const counted_iterator<I>& y);
bool operator>=(counted_sentinel x, counted_sentinel y);
\end{itemdecl}

\begin{itemdescr}
\pnum
\returns \tcode{!(x < y)}.
\end{itemdescr}

\rSec4[counted.iter.nonmember]{\tcode{counted_iterator} non-member functions}

\indexlibrary{\idxcode{operator-}!\idxcode{counted_iterator}}%
\indexlibrary{\idxcode{counted_iterator}!\idxcode{operator-}}%
\begin{itemdecl}
template <WeakInputIterator I1, WeakInputIterator I2>
  DifferenceType<I2> operator-(
    const counted_iterator<I1>& x, const counted_iterator<I2>& y);
\end{itemdecl}

\begin{itemdescr}
\pnum
\returns \tcode{x.base() - y.base()} if \tcode{SizedIteratorRange<I2, I1>};
otherwise, \tcode{y.count() - x.count()}.
\end{itemdescr}

\begin{itemdecl}
template <WeakInputIterator I>
  DifferenceType<I> operator-(
    const counted_iterator<I>& x, counted_sentinel y);
\end{itemdecl}

\begin{itemdescr}
\pnum
\returns \tcode{-x.count()}.
\end{itemdescr}

\begin{itemdecl}
template <WeakInputIterator I>
  DifferenceType<I> operator-(
    counted_sentinel x, const counted_iterator<I>& y);
\end{itemdecl}

\begin{itemdescr}
\pnum
\returns \tcode{y.count()}.
\end{itemdescr}

\begin{itemdecl}
ptrdiff_t operator-(counted_sentinel x, counted_sentinel y);
\end{itemdecl}

\begin{itemdescr}
\pnum
\returns \tcode{0}.
\end{itemdescr}

\indexlibrary{\idxcode{operator+}!\idxcode{counted_iterator}}%
\indexlibrary{\idxcode{counted_iterator}!\idxcode{operator+}}%
\begin{itemdecl}
template <RandomAccessIterator I>
  counted_iterator<I>
    operator+(DifferenceType<I> n, const counted_iterator<I>& x);
\end{itemdecl}

\begin{itemdescr}
\pnum
\requires \tcode{n <= x.count()}.

\pnum
\returns \tcode{x + n}.
\end{itemdescr}

\indexlibrary{\idxcode{make_counted_iterator}}%
\begin{itemdecl}
template <WeakInputIterator I>
  counted_iterator<I> make_counted_iterator(I i, DifferenceType<I> n);
\end{itemdecl}

\begin{itemdescr}
\pnum
\requires \tcode{n >= 0}.

\pnum
\returns \tcode{counted_iterator<I>(i, n)}.
\end{itemdescr}

\indexlibrary{\idxcode{advance}}%
\begin{itemdecl}
template <WeakInputIterator I>
  void advance(counted_iterator<I>& i, DifferenceType<I> n);
\end{itemdecl}

\begin{itemdescr}
\pnum
\requires \tcode{n <= i.count()}.

\pnum
\effects
\begin{codeblock}
i = make_counted_iterator(next(i.base(), n), i.count() - n);
\end{codeblock}
\end{itemdescr}

\rSec3[counted.sentinel]{Counted sentinel}

\pnum
Class \tcode{counted_sentinel} is an empty type used to represent the end of a counted
range. It is used together with class template
\tcode{counted_iterator}(~\ref{counted.iterator}) to denote a range of elements that
starts at a known position and includes the subsequent $N$ elements.

\indexlibrary{\idxcode{counted_sentinel}}%
\begin{itemdecl}
namespace std {
  class counted_sentinel { };
}
\end{itemdecl}

\rSec3[counted.traits.specializations]{Specializations of \tcode{common_type}}

\indexlibrary{\idxcode{common_type}}%
\begin{itemdecl}
namespace std {
  template<WeakInputIterator I>
  struct common_type<counted_iterator<I>, counted_sentinel> {
    using type = common_iterator<counted_iterator<I>, counted_sentinel>;
  };
  template<WeakInputIterator I>
  struct common_type<counted_sentinel, counted_iterator<I>> {
    using type = common_iterator<counted_iterator<I>, counted_sentinel>;
  };
}
\end{itemdecl}

\begin{itemdescr}
\pnum
\enternote By specializing \tcode{common_type} this way, \tcode{counted_iterator}
and \tcode{counted_sentinel} can satisfy the \tcode{Common} requirement of the
\tcode{EqualityComparable} concept.\exitnote
\end{itemdescr}

\rSec2[unreachable.sentinels]{Unreachable sentinel}

\rSec3[unreachable.sentinel]{Class \tcode{unreachable} sentinel}

\pnum
\indexlibrary{\idxcode{unreachable}}%
Class \tcode{unreachable} is a sentinel type that can be used with any
\tcode{Iterator} to denote an infinite range. Comparing an iterator for equality with
an object of type \tcode{unreachable} always returns \tcode{false}.

\enterexample
\begin{codeblock}
char* p;
// set \tcode{p} to point to a character buffer containing newlines
char* nl = find(p, unreachable(), '@\textbackslash@n');
\end{codeblock}

Provided a newline character really exists in the buffer, the use of \tcode{unreachable}
above potentially make the call to \tcode{find} more efficient since the loop test against
the sentinel does not require a conditional branch.
\exitexample

\begin{codeblock}
namespace std {
  class unreachable { };
  template <Iterator I>
    constexpr bool operator==(I const &, unreachable) noexcept;
  template <Iterator I>
    constexpr bool operator==(unreachable, I const &) noexcept;
  constexpr bool operator==(unreachable, unreachable) noexcept;
  template <Iterator I>
    constexpr bool operator!=(I const &, unreachable) noexcept;
  template <Iterator I>
    constexpr bool operator!=(unreachable, I const &) noexcept;
  constexpr bool operator!=(unreachable, unreachable) noexcept;
}
\end{codeblock}

\rSec3[unreachable.sentinel.ops]{\tcode{unreachable} operations}

\rSec4[unreachable.sentinel.op==]{\tcode{operator==}}

\indexlibrary{\idxcode{operator==}!\idxcode{unreachable}}%
\indexlibrary{\idxcode{unreachable}!\idxcode{operator==}}%
\begin{itemdecl}
template <Iterator I>
  constexpr bool operator==(I const &, unreachable) noexcept;
template <Iterator I>
  constexpr bool operator==(unreachable, I const &) noexcept;
\end{itemdecl}

\begin{itemdescr}
\pnum
\returns \tcode{false}.
\end{itemdescr}

\begin{itemdecl}
constexpr bool operator==(unreachable, unreachable) noexcept;
\end{itemdecl}

\begin{itemdescr}
\pnum
\returns \tcode{true}.
\end{itemdescr}

\rSec4[unreachable.sentinel.op!=]{\tcode{operator!=}}

\indexlibrary{\idxcode{operator"!=}!\idxcode{unreachable}}%
\indexlibrary{\idxcode{unreachable}!\idxcode{operator"!=}}%
\begin{itemdecl}
template <Iterator I>
  constexpr bool operator!=(I const & x, unreachable y) noexcept;
template <Iterator I>
  constexpr bool operator!=(unreachable x, I const & y) noexcept;
constexpr bool operator!=(unreachable x, unreachable y) noexcept;
\end{itemdecl}

\begin{itemdescr}
\pnum
\returns
\tcode{!(x == y)}
\end{itemdescr}

\rSec3[unreachable.traits.specializations]{Specializations of \tcode{common_type}}

\indexlibrary{\idxcode{common_type}}%
\begin{itemdecl}
namespace std {
  template<Iterator I>
  struct common_type<I, unreachable> {
    using type = common_iterator<I, unreachable>;
  };
  template<Iterator I>
  struct common_type<unreachable, I> {
    using type = common_iterator<I, unreachable>;
  };
}
\end{itemdecl}

\begin{itemdescr}
\pnum
\enternote By specializing \tcode{common_type} this way, any iterator and
\tcode{unreachable} can satisfy the \tcode{Common} requirement of the
\tcode{EqualityComparable} concept.\exitnote
\end{itemdescr}

\end{addedblock}

\rSec1[stream.iterators]{Stream iterators}

\pnum
To make it possible for algorithmic templates to work directly with input/output streams, appropriate
iterator-like
class templates
are provided.

\enterexample
\begin{codeblock}
partial_sum(istream_iterator<double, char>(cin),
  istream_iterator<double, char>(),
  ostream_iterator<double, char>(cout, "@\textbackslash@n"));
\end{codeblock}

reads a file containing floating point numbers from
\tcode{cin},
and prints the partial sums onto
\tcode{cout}.
\exitexample

\rSec2[istream.iterator]{Class template \tcode{istream_iterator}}

\pnum
\indexlibrary{\idxcode{istream_iterator}}%
The class template
\tcode{istream_iterator}
is an input iterator~(\ref{input.iterators}) that
reads (using
\tcode{operator\shr})
successive elements from the input stream for which it was constructed.
After it is constructed, and every time
\tcode{++}
is used, the iterator reads and stores a value of
\tcode{T}.
If the iterator fails to read and store a value of \tcode{T}
(\tcode{fail()}
on the stream returns
\tcode{true}),
the iterator becomes equal to the
\term{end-of-stream}
iterator value.
The constructor with no arguments
\tcode{istream_iterator()}
always constructs
an end-of-stream input iterator object, which is the only legitimate iterator to be used
for the end condition.
The result of
\tcode{operator*}
on an end-of-stream iterator is not defined.
For any other iterator value a
\tcode{const T\&}
is returned.
\removed{The result of
\tcode{operator->}
on an end-of-stream iterator is not defined.
For any other iterator value a
\tcode{const T*}
is returned.}
The behavior of a program that applies \tcode{operator++()} to an end-of-stream
iterator is undefined.
It is impossible to store things into istream iterators.

\pnum
Two end-of-stream iterators are always equal.
An end-of-stream iterator is not
equal to a non-end-of-stream iterator.
Two non-end-of-stream iterators are equal when they are constructed from the same stream.

\begin{codeblock}
namespace std {
  template <class T, class charT = char, class traits = char_traits<charT>,
      class Distance = ptrdiff_t>
  class istream_iterator@\removed{:}@
    @\removed{public iterator<input_iterator_tag, T, Distance, const T*, const T\&>}@ {
  public:
    @\added{typedef input_iterator_tag iterator_category;}@
    @\added{typedef Distance difference_type;}@
    @\added{typedef T value_type;}@
    @\added{typedef const T\& reference;}@
    typedef charT char_type;
    typedef traits traits_type;
    typedef basic_istream<charT,traits> istream_type;
    @\seebelow@ istream_iterator();
    istream_iterator(istream_type& s);
    istream_iterator(const istream_iterator& x) = default;
   ~istream_iterator() = default;

    const T& operator*() const;
    @\removed{const T* operator->() const;}@
    istream_iterator<T,charT,traits,Distance>& operator++();
    istream_iterator<T,charT,traits,Distance>  operator++(int);
  private:
    basic_istream<charT,traits>* in_stream; // \expos
    T value;                                // \expos
  };

  template <class T, class charT, class traits, class Distance>
    bool operator==(const istream_iterator<T,charT,traits,Distance>& x,
            const istream_iterator<T,charT,traits,Distance>& y);
  template <class T, class charT, class traits, class Distance>
    bool operator!=(const istream_iterator<T,charT,traits,Distance>& x,
            const istream_iterator<T,charT,traits,Distance>& y);
}
\end{codeblock}

\rSec3[istream.iterator.cons]{\tcode{istream_iterator} constructors and destructor}


\indexlibrary{\idxcode{istream_iterator}!constructor}%
\begin{itemdecl}
@\seebelow@ istream_iterator();
\end{itemdecl}

\begin{itemdescr}
\pnum
\effects
Constructs the end-of-stream iterator. If \tcode{T} is a literal type, then this
constructor shall be a \tcode{constexpr} constructor.

\pnum
\postcondition \tcode{in_stream == 0}.
\end{itemdescr}


\indexlibrary{\idxcode{istream_iterator}!constructor}%
\begin{itemdecl}
istream_iterator(istream_type& s);
\end{itemdecl}

\begin{itemdescr}
\pnum
\effects
Initializes \textit{in_stream} with \tcode{\&s}. \textit{value} may be initialized during
construction or the first time it is referenced.

\pnum
\postcondition \tcode{in_stream == \&s}.
\end{itemdescr}

\indexlibrary{\idxcode{istream_iterator}!constructor}%
\begin{itemdecl}
istream_iterator(const istream_iterator& x) = default;
\end{itemdecl}

\begin{itemdescr}
\pnum
\effects
Constructs a copy of \tcode{x}. If \tcode{T} is a literal type, then this constructor shall be a trivial copy constructor.

\pnum
\postcondition \tcode{in_stream == x.in_stream}.
\end{itemdescr}

\indexlibrary{\idxcode{istream_iterator}!destructor}%
\begin{itemdecl}
~istream_iterator() = default;
\end{itemdecl}

\begin{itemdescr}
\pnum
\effects
The iterator is destroyed. If \tcode{T} is a literal type, then this destructor shall be a trivial destructor.
\end{itemdescr}

\rSec3[istream.iterator.ops]{\tcode{istream_iterator} operations}

\indexlibrary{\idxcode{operator*}!\idxcode{istream_iterator}}%
\indexlibrary{\idxcode{istream_iterator}!\idxcode{operator*}}%
\begin{itemdecl}
const T& operator*() const;
\end{itemdecl}

\begin{itemdescr}
\pnum
\returns
\textit{value}.
\end{itemdescr}

\begin{removedblock}
\indexlibrary{\idxcode{operator->}!\idxcode{istream_iterator}}%
\indexlibrary{\idxcode{istream_iterator}!\idxcode{operator->}}%
\begin{itemdecl}
const T* operator->() const;
\end{itemdecl}

\begin{itemdescr}
\pnum
\returns
\tcode{\&(operator*())}.
\end{itemdescr}
\end{removedblock}

\indexlibrary{\idxcode{operator++}!\idxcode{istream_iterator}}%
\indexlibrary{\idxcode{istream_iterator}!\idxcode{operator++}}%
\begin{itemdecl}
istream_iterator<T,charT,traits,Distance>& operator++();
\end{itemdecl}

\begin{itemdescr}
\pnum
\requires \tcode{in_stream != 0}.

\pnum
\effects
\tcode{*in_stream \shr value}.

\pnum
\returns
\tcode{*this}.
\end{itemdescr}

\indexlibrary{\idxcode{operator++}!\idxcode{istream_iterator}}%
\indexlibrary{\idxcode{istream_iterator}!\idxcode{operator++}}%
\begin{itemdecl}
istream_iterator<T,charT,traits,Distance> operator++(int);
\end{itemdecl}

\begin{itemdescr}
\pnum
\requires \tcode{in_stream != 0}.

\pnum
\effects
\begin{codeblock}
istream_iterator<T,charT,traits,Distance> tmp = *this;
*in_stream >> value;
return (tmp);
\end{codeblock}
\end{itemdescr}

\indexlibrary{\idxcode{operator==}!\idxcode{istream_iterator}}%
\indexlibrary{\idxcode{istream_iterator}!\idxcode{operator==}}%
\begin{itemdecl}
template <class T, class charT, class traits, class Distance>
  bool operator==(const istream_iterator<T,charT,traits,Distance> &x,
                  const istream_iterator<T,charT,traits,Distance> &y);
\end{itemdecl}

\begin{itemdescr}
\pnum
\returns
\tcode{x.in_stream == y.in_stream}.%
\indexlibrary{\idxcode{istream_iterator}!\idxcode{operator==}}
\end{itemdescr}

\indexlibrary{\idxcode{operator"!=}!\idxcode{istream_iterator}}%
\indexlibrary{\idxcode{istream_iterator}!\idxcode{operator"!=}}%
\begin{itemdecl}
template <class T, class charT, class traits, class Distance>
  bool operator!=(const istream_iterator<T,charT,traits,Distance> &x,
                  const istream_iterator<T,charT,traits,Distance> &y);
\end{itemdecl}

\indexlibrary{\idxcode{istream_iterator}!\idxcode{operator"!=}}%
\begin{itemdescr}
\pnum
\returns
\tcode{!(x == y)}
\end{itemdescr}

\rSec2[ostream.iterator]{Class template \tcode{ostream_iterator}}

\pnum
\indexlibrary{\idxcode{ostream_iterator}}%
\tcode{ostream_iterator}
writes (using
\tcode{operator\shl})
successive elements onto the output stream from which it was constructed.
If it was constructed with
\tcode{charT*}
as a constructor argument, this string, called a
\term{delimiter string},
is written to the stream after every
\tcode{T}
is written.
It is not possible to get a value out of the output iterator.
Its only use is as an output iterator in situations like

\begin{codeblock}
while (first != last)
  *result++ = *first++;
\end{codeblock}

\pnum
\tcode{ostream_iterator}
is defined as:

\begin{codeblock}
namespace std {
  template <class T, class charT = char, class traits = char_traits<charT> >
  class ostream_iterator@\removed{:}@
    @\removed{public iterator<output_iterator_tag, void, void, void, void>}@ {
  public:
    @\added{typedef output_iterator_tag iterator_category;}@
    @\added{typedef ptrdiff_t difference_type;}@
    typedef charT char_type;
    typedef traits traits_type;
    typedef basic_ostream<charT,traits> ostream_type;
    @\added{constexpr ostream_iterator() noexcept;}@
    ostream_iterator(ostream_type& s);
    ostream_iterator(ostream_type& s, const charT* delimiter);
    ostream_iterator(const ostream_iterator<T,charT,traits>& x);
   ~ostream_iterator();
    ostream_iterator<T,charT,traits>& operator=(const T& value);

    ostream_iterator<T,charT,traits>& operator*();
    ostream_iterator<T,charT,traits>& operator++();
    ostream_iterator<T,charT,traits>& operator++(int);
  private:
    basic_ostream<charT,traits>* out_stream;  // \expos
    const charT* delim;                       // \expos
  };
}
\end{codeblock}

\rSec3[ostream.iterator.cons.des]{\tcode{ostream_iterator} constructors and destructor}

\begin{addedblock}
\indexlibrary{\idxcode{ostream_iterator}!constructor}%
\begin{itemdecl}
constexpr ostream_iterator() noexcept;
\end{itemdecl}

\begin{itemdescr}
\pnum
\effects
Initializes \textit{out_stream} and \tcode{delim} with null.
\end{itemdescr}
\end{addedblock}

\indexlibrary{\idxcode{ostream_iterator}!constructor}%
\begin{itemdecl}
ostream_iterator(ostream_type& s);
\end{itemdecl}

\begin{itemdescr}
\pnum
\effects
Initializes \textit{out_stream} with \tcode{\&s} and \textit{delim} with null.
\end{itemdescr}


\indexlibrary{\idxcode{ostream_iterator}!constructor}%
\begin{itemdecl}
ostream_iterator(ostream_type& s, const charT* delimiter);
\end{itemdecl}

\begin{itemdescr}
\pnum
\effects
Initializes \textit{out_stream} with \tcode{\&s} and \textit{delim} with \tcode{delimiter}.
\end{itemdescr}


\indexlibrary{\idxcode{ostream_iterator}!constructor}%
\begin{itemdecl}
ostream_iterator(const ostream_iterator& x);
\end{itemdecl}

\begin{itemdescr}
\pnum
\effects
Constructs a copy of \tcode{x}.
\end{itemdescr}

\indexlibrary{\idxcode{ostream_iterator}!destructor}%
\begin{itemdecl}
~ostream_iterator();
\end{itemdecl}

\begin{itemdescr}
\pnum
\effects
The iterator is destroyed.
\end{itemdescr}

\rSec3[ostream.iterator.ops]{\tcode{ostream_iterator} operations}

\indexlibrary{\idxcode{operator=}!\idxcode{ostream_iterator}}%
\indexlibrary{\idxcode{ostream_iterator}!\idxcode{operator=}}%
\begin{itemdecl}
ostream_iterator& operator=(const T& value);
\end{itemdecl}

\begin{itemdescr}
\pnum
\effects
\begin{codeblock}
*@\textit{out_stream}@ << value;
if(delim != 0)
  *@\textit{out_stream}@ << @\textit{delim}@;
return (*this);
\end{codeblock}

\begin{addedblock}
\pnum
\requires \tcode{out_stream!= 0}.
\end{addedblock}
\end{itemdescr}

\indexlibrary{\idxcode{operator*}!\idxcode{ostream_iterator}}%
\indexlibrary{\idxcode{ostream_iterator}!\idxcode{operator*}}%
\begin{itemdecl}
ostream_iterator& operator*();
\end{itemdecl}

\begin{itemdescr}
\pnum
\returns
\tcode{*this}.
\end{itemdescr}

\indexlibrary{\idxcode{operator++}!\idxcode{ostream_iterator}}%
\indexlibrary{\idxcode{ostream_iterator}!\idxcode{operator++}}%
\begin{itemdecl}
ostream_iterator& operator++();
ostream_iterator& operator++(int);
\end{itemdecl}

\begin{itemdescr}
\pnum
\returns
\tcode{*this}.
\end{itemdescr}

\rSec2[istreambuf.iterator]{Class template \tcode{istreambuf_iterator}}

\pnum
The
class template
\tcode{istreambuf_iterator}
defines an input iterator~(\ref{input.iterators}) that
reads successive
\textit{characters}
from the streambuf for which it was constructed.
\tcode{operator*}
provides access to the current input character, if any.
\removed{\enternote \tcode{operator->} may return a proxy. \exitnote}
Each time
\tcode{operator++}
is evaluated, the iterator advances to the next input character.
If the end of stream is reached (\tcode{streambuf_type::sgetc()} returns
\tcode{traits::eof()}),
the iterator becomes equal to the
\term{end-of-stream}
iterator value.
The default constructor
\tcode{istreambuf_iterator()}
and the constructor
\tcode{istreambuf_iterator(0)}
both construct an end-of-stream iterator object suitable for use
as an end-of-range.
All specializations of \tcode{istreambuf_iterator} shall have a trivial copy
constructor, a \tcode{constexpr} default constructor, and a trivial destructor.

\pnum
The result of
\tcode{operator*()}
on an end-of-stream iterator is undefined.
\indextext{undefined behavior}%
For any other iterator value a
\tcode{char_type}
value is returned.
It is impossible to assign a character via an input iterator.

\indexlibrary{\idxcode{istreambuf_iterator}}%
\begin{codeblock}
namespace std {
  template<class charT, class traits = char_traits<charT> >
  class istreambuf_iterator
     @\removed{: public iterator<input_iterator_tag, charT,}@
                       @\removed{typename traits::off_type, \unspec, charT>}@ {
  public:
    @\added{typedef input_iterator_tag            iterator_category;}@
    @\added{typedef charT                         value_type;}@
    @\added{typedef typename traits::off_type     difference_type;}@
    @\added{typedef charT                         reference;}@
    typedef charT                         char_type;
    typedef traits                        traits_type;
    typedef typename traits::int_type     int_type;
    typedef basic_streambuf<charT,traits> streambuf_type;
    typedef basic_istream<charT,traits>   istream_type;

    class proxy;                          // \expos

    constexpr istreambuf_iterator() noexcept;
    istreambuf_iterator(const istreambuf_iterator&) noexcept = default;
    ~istreambuf_iterator() = default;
    istreambuf_iterator(istream_type& s) noexcept;
    istreambuf_iterator(streambuf_type* s) noexcept;
    istreambuf_iterator(const proxy& p) noexcept;
    charT operator*() const;
    @\removed{pointer operator->() const;}@
    istreambuf_iterator<charT,traits>& operator++();
    proxy operator++(int);
    bool equal(const istreambuf_iterator& b) const;
  private:
    streambuf_type* sbuf_;                // \expos
  };

  template <class charT, class traits>
    bool operator==(const istreambuf_iterator<charT,traits>& a,
            const istreambuf_iterator<charT,traits>& b);
  template <class charT, class traits>
    bool operator!=(const istreambuf_iterator<charT,traits>& a,
            const istreambuf_iterator<charT,traits>& b);
}
\end{codeblock}

\rSec3[istreambuf.iterator::proxy]{Class template \tcode{istreambuf_iterator::proxy}}

\indexlibrary{\idxcode{proxy}!\idxcode{istreambuf_iterator}}%
\begin{codeblock}
namespace std {
  template <class charT, class traits = char_traits<charT> >
  class istreambuf_iterator<charT, traits>::proxy { // \expos
    charT keep_;
    basic_streambuf<charT,traits>* sbuf_;
    proxy(charT c, basic_streambuf<charT,traits>* sbuf)
      : keep_(c), sbuf_(sbuf) { }
  public:
    charT operator*() { return keep_; }
  };
}
\end{codeblock}

\pnum
Class
\tcode{istreambuf_iterator<charT,traits>::proxy}
is for exposition only.
An implementation is permitted to provide equivalent functionality without
providing a class with this name.
Class
\tcode{istreambuf_iterator<charT, traits>\colcol{}proxy}
provides a temporary
placeholder as the return value of the post-increment operator
(\tcode{operator++}).
It keeps the character pointed to by the previous value
of the iterator for some possible future access to get the character.

\rSec3[istreambuf.iterator.cons]{\tcode{istreambuf_iterator} constructors}


\indexlibrary{\idxcode{istreambuf_iterator}!constructor}%
\begin{itemdecl}
constexpr istreambuf_iterator() noexcept;
\end{itemdecl}

\begin{itemdescr}
\pnum
\effects
Constructs the end-of-stream iterator.
\end{itemdescr}


\indexlibrary{\idxcode{istreambuf_iterator}!constructor}%
\begin{itemdecl}
istreambuf_iterator(basic_istream<charT,traits>& s) noexcept;
istreambuf_iterator(basic_streambuf<charT,traits>* s) noexcept;
\end{itemdecl}

\begin{itemdescr}
\pnum
\effects
Constructs an
\tcode{istreambuf_iterator<>}
that uses the
\tcode{basic_streambuf<>}
object
\tcode{*(s.rdbuf())},
or
\tcode{*s},
respectively.
Constructs an end-of-stream iterator if
\tcode{s.rdbuf()}
is null.
\end{itemdescr}


\indexlibrary{\idxcode{istreambuf_iterator}!constructor}%
\begin{itemdecl}
istreambuf_iterator(const proxy& p) noexcept;
\end{itemdecl}

\begin{itemdescr}
\pnum
\effects
Constructs a
\tcode{istreambuf_iterator<>}
that uses the
\tcode{basic_streambuf<>}
object pointed to by the
\tcode{proxy}
object's constructor argument \tcode{p}.
\end{itemdescr}

\rSec3[istreambuf.iterator::op*]{\tcode{istreambuf_iterator::operator*}}

\indexlibrary{\idxcode{operator*}!\idxcode{istreambuf_iterator}}%
\begin{itemdecl}
charT operator*() const
\end{itemdecl}

\begin{itemdescr}
\pnum
\returns
The character obtained via the
\tcode{streambuf}
member
\tcode{sbuf_->sgetc()}.
\end{itemdescr}

\rSec3[istreambuf.iterator::op++]{\tcode{istreambuf_iterator::operator++}}

\indexlibrary{\idxcode{operator++}!\idxcode{istreambuf_iterator}}%
\begin{itemdecl}
istreambuf_iterator<charT,traits>&
    istreambuf_iterator<charT,traits>::operator++();
\end{itemdecl}

\begin{itemdescr}
\pnum
\effects
\tcode{sbuf_->sbumpc()}.

\pnum
\returns
\tcode{*this}.
\end{itemdescr}

\indexlibrary{\idxcode{operator++}!\idxcode{istreambuf_iterator}}%
\indexlibrary{\idxcode{istreambuf_iterator}!\idxcode{operator++}}%
\begin{itemdecl}
proxy istreambuf_iterator<charT,traits>::operator++(int);
\end{itemdecl}

\begin{itemdescr}
\pnum
\returns
\tcode{proxy(sbuf_->sbumpc(), sbuf_)}.
\end{itemdescr}

\rSec3[istreambuf.iterator::equal]{\tcode{istreambuf_iterator::equal}}

\indexlibrary{\idxcode{equal}!\idxcode{istreambuf_iterator}}%
\begin{itemdecl}
bool equal(const istreambuf_iterator<charT,traits>& b) const;
\end{itemdecl}

\begin{itemdescr}
\pnum
\returns
\tcode{true}
if and only if both iterators are at end-of-stream,
or neither is at end-of-stream, regardless of what
\tcode{streambuf}
object they use.
\end{itemdescr}

\rSec3[istreambuf.iterator::op==]{\tcode{operator==}}

\indexlibrary{\idxcode{operator==}!\idxcode{istreambuf_iterator}}%
\begin{itemdecl}
template <class charT, class traits>
  bool operator==(const istreambuf_iterator<charT,traits>& a,
                  const istreambuf_iterator<charT,traits>& b);
\end{itemdecl}

\begin{itemdescr}
\pnum
\returns
\tcode{a.equal(b)}.
\end{itemdescr}

\rSec3[istreambuf.iterator::op!=]{\tcode{operator!=}}

\indexlibrary{\idxcode{operator"!=}!\idxcode{istreambuf_iterator}}%
\begin{itemdecl}
template <class charT, class traits>
  bool operator!=(const istreambuf_iterator<charT,traits>& a,
                  const istreambuf_iterator<charT,traits>& b);
\end{itemdecl}

\begin{itemdescr}
\pnum
\returns
\tcode{!a.equal(b)}.
\end{itemdescr}

\rSec2[ostreambuf.iterator]{Class template \tcode{ostreambuf_iterator}}

\indexlibrary{\idxcode{ostreambuf_iterator}}%
\begin{codeblock}
namespace std {
  template <class charT, class traits = char_traits<charT> >
  class ostreambuf_iterator @\removed{:}@
    @\removed{public iterator<output_iterator_tag, void, void, void, void>}@ {
  public:
    @\added{typedef output_iterator_tag           iterator_category;}@
    @\added{typedef ptrdiff_t                     difference_type;}@
    typedef charT                         char_type;
    typedef traits                        traits_type;
    typedef basic_streambuf<charT,traits> streambuf_type;
    typedef basic_ostream<charT,traits>   ostream_type;

  public:
    @\added{constexpr ostreambuf_iterator() noexcept;}@
    ostreambuf_iterator(ostream_type& s) noexcept;
    ostreambuf_iterator(streambuf_type* s) noexcept;
    ostreambuf_iterator& operator=(charT c);

    ostreambuf_iterator& operator*();
    ostreambuf_iterator& operator++();
    ostreambuf_iterator& operator++(int);
    bool failed() const noexcept;

  private:
    streambuf_type* sbuf_;                // \expos
  };
}
\end{codeblock}

\pnum
The
class template
\tcode{ostreambuf_iterator}
writes successive
\textit{characters}
onto the output stream from which it was constructed.
It is not possible to get a character value out of the output iterator.

\rSec3[ostreambuf.iter.cons]{\tcode{ostreambuf_iterator} constructors}

\begin{addedblock}
\indexlibrary{\idxcode{ostreambuf_iterator}!constructor}%
\begin{itemdecl}
constexpr ostreambuf_iterator() noexcept;
\end{itemdecl}

\begin{itemdescr}
\pnum
\effects
Initializes \tcode{sbuf_} with null.
\end{itemdescr}
\end{addedblock}

\indexlibrary{\idxcode{ostreambuf_iterator}!constructor}%
\begin{itemdecl}
ostreambuf_iterator(ostream_type& s) noexcept;
\end{itemdecl}

\begin{itemdescr}
\pnum
\requires
\tcode{s.rdbuf()}
shall not null pointer.
\end{itemdescr}

\begin{itemdescr}
\pnum
\effects
Initializes \tcode{sbuf_} with \tcode{s.rdbuf()}.
\end{itemdescr}


\indexlibrary{\idxcode{ostreambuf_iterator}!constructor}%
\begin{itemdecl}
ostreambuf_iterator(streambuf_type* s) noexcept;
\end{itemdecl}

\begin{itemdescr}
\pnum
\requires
\tcode{s}
shall not be a null pointer.

\pnum
\effects
Initializes \tcode{sbuf_} with \tcode{s}.
\end{itemdescr}

\rSec3[ostreambuf.iter.ops]{\tcode{ostreambuf_iterator} operations}

\indexlibrary{\idxcode{operator=}!\idxcode{ostreambuf_iterator}}%
\begin{itemdecl}
ostreambuf_iterator<charT,traits>&
  operator=(charT c);
\end{itemdecl}

\begin{itemdescr}
\pnum
\effects
If
\tcode{failed()}
yields
\tcode{false},
calls
\tcode{sbuf_->sputc(c)};
otherwise has no effect.

\begin{addedblock}
\pnum
\requires \tcode{sbuf_ != 0}.
\end{addedblock}

\pnum
\returns
\tcode{*this}.
\end{itemdescr}

\indexlibrary{\idxcode{operator*}!\idxcode{ostreambuf_iterator}}%
\begin{itemdecl}
ostreambuf_iterator<charT,traits>& operator*();
\end{itemdecl}

\begin{itemdescr}
\pnum
\returns
\tcode{*this}.
\end{itemdescr}

\indexlibrary{\idxcode{operator++}!\idxcode{ostreambuf_iterator}}%
\begin{itemdecl}
ostreambuf_iterator<charT,traits>& operator++();
ostreambuf_iterator<charT,traits>& operator++(int);
\end{itemdecl}

\begin{itemdescr}
\pnum
\returns
\tcode{*this}.
\end{itemdescr}

\indexlibrary{\idxcode{failed}!\idxcode{ostreambuf_iterator}}%
\begin{itemdecl}
bool failed() const noexcept;
\end{itemdecl}

\begin{itemdescr}
\pnum
\returns
\tcode{true}
if in any prior use of member
\tcode{operator=},
the call to
\tcode{sbuf_->sputc()}
returned
\tcode{traits::eof()};
or
\tcode{false}
otherwise.

\begin{addedblock}
\pnum
\requires \tcode{sbuf_ != 0}.
\end{addedblock}
\end{itemdescr}

\begin{addedblock}

\rSec1[iterables]{Iterable concepts}

\rSec2[iterables.general]{General}

\pnum
This subclause describes components for dealing with ranges of elements.

\pnum
The following subclauses describe
iterable and range requirements, and
components for
iterable primitives,
predefined ranges,
and stream ranges,
as summarized in Table~\ref{tab:iterables.lib.summary}.

\begin{libsumtab}{Iterables library summary}{tab:iterables.lib.summary}
  \ref{iterables.requirements} & Requirements       &                           \\ \rowsep
  \ref{iterable.primitives} & Iterable primitives   &   \tcode{<iterator>}      \\
  \ref{predef.range} & Predefined ranges            &                           \\
  \ref{stream.ranges} & Stream ranges               &                           \\
\end{libsumtab}

\rSec2[iterables.requirements]{Iterable requirements}

\rSec3[iterables.requirements.general]{In general}

\pnum
Iterables are an abstraction of containers that allow a C++ program to
operate on elements of data structures uniformly. It their simplest form, an
iterable object is one on which one can call \tcode{begin} and
\tcode{end} to get an iterator~(\ref{iterator.iterators}) and a
sentinel~(\ref{sentinel.iterators}) or an iterator. To be able to construct
template algorithms and range adaptors that work correctly and efficiently on
different types of sequences, the library formalizes not just the interfaces but
also the semantics and complexity assumptions of iterables.

\pnum
This International Standard defines three fundamental categories of iterables
based on the syntax and semantics supported by each: \techterm{iterable},
\techterm{sized iterable} and \techterm{range}, as shown in
Table~\ref{tab:iterables.relations}.

\begin{floattable}{Relations among iterable categories}{tab:iterables.relations}
  {lll}
  \topline
  \textbf{Sized Iterable} &            &                   \\
                          & $\searrow$ &                   \\
                          &            & \textbf{Iterable} \\
                          & $\nearrow$ &                   \\
  \textbf{Range}          &            &                   \\
\end{floattable}

\pnum
The \tcode{Iterable} concept requires only that \tcode{begin} and \tcode{end}
return an iterator and a sentinel. \enternote An iterator is a valid sentinel.
\exitnote The \tcode{SizedIterable} concept refines \tcode{Iterable} but adds
the requirement that the number of elements in the iterable can be determined
in constant time with the \tcode{size} function. The \tcode{Range} concept describes
requirements on an iterable type with constant-time copy, WeakInputIterator and assignment
operators.

\pnum
In addition to the three fundamental iterable categories, this standard defines
a number of convenience refinements of \tcode{Iterable} that group together requirements
that appear often in the concepts, algorithms, and range views. \techterm{Bounded iterables}
are iterables for which \tcode{begin} and \tcode{end} return objects of the
same type. \techterm{Random access iterables} are iterables for which
\tcode{begin} returns a model of
\tcode{RandomAccessIterator}~(\ref{random.access.iterators}). The iterable
categories \techterm{bidirectional iterable}, \techterm{forward iterable},
\techterm{input iterable} and \techterm{output iterable} are defined similarly.
\enternote There is no \techterm{weak input iterable} or
\techterm{weak output iterable} because of the \tcode{EqualityComparable}
requirement on iterators and sentinels. \exitnote \ednote{Rethink that because
a weak input iterable would not require (strongly) incrementable iterators.}

\rSec3[iterable.iterables]{Iterables}

\pnum
The \tcode{Iterable} concept defines the requirements of a type that allows
iteration over its elements by providing a \tcode{begin} iterator and an
\tcode{end} iterator or sentinel.

\begin{codeblock}
template <class T>
concept bool Iterable =
  requires(T t) {
    typename IteratorType<T>;
    typename SentinelType<T>;
    { begin(t) } -> IteratorType<T>;
    { end(t) } -> SentinelType<T>;
    requires IteratorRange<IteratorType, SentinelType>;
  };
\end{codeblock}

\tcode{begin} and \tcode{end} are required to be amortized constant time.
\enternote Most algorithms requiring this concept simply forward to an
Iterator-based algorithm by calling \tcode{begin} and \tcode{end}. \exitnote

\rSec3[sized.iterables]{Sized iterables}

\pnum
The \tcode{SizedIterable} concept describes the requirements of an Iterable
type that knows its size in constant time with the \tcode{size} function.

\begin{codeblock}
// For exposition only:
template <Iterator T>
concept bool SizedIterableLike_ =
  requires(T t) {
    typename SizeType<T>;
    { size(t) } -> SizeType<T>;
    requires Integral<SizeType<T>>;
  };

template <SizedIterableLike_ T>
concept bool SizedIterable =
  is_sized_iterable<T>::value;
\end{codeblock}

\pnum
Any \tcode{Iterable} object \tcode{o} for which \tcode{size(o)} compiles and
returns an \tcode{Integral} type is a \tcode{SizedIterable} by default. The
\tcode{is_sized_iterable} trait allows users to override the default in the
case of accidental conformance.

\enternote
A possible implementation for \tcode{is_sized_iterable} is given below:

\begin{codeblock}
// For exposition only:
template<typename R>
struct is_sized_iterable_impl_
  : std::integral_constant< bool, SizedIterableLike_<R> >
{};

template<typename R>
struct is_sized_iterable
  : conditional<
        is_same<R, remove_const_t<remove_reference_t<R>>>::value,
        is_sized_iterable_impl_<R>,
        is_sized_iterable<remove_const_t<remove_reference_t<R>>>
    >::type
{};
\end{codeblock}
\ednote{The handling of top-level reference here is inconsistent with the other
type traits.}
\exitnote

\rSec3[range.iterables]{Ranges}

\pnum
The \tcode{Range} concept describes the requirements of an Iterable type that
has constant time copy, move and assignment operators; that is, the cost of
these operations is not proportional to the number of elements in the Range.

\pnum
\enterexample
Examples of Ranges are:

\begin{itemize}
\item An Iterable type that wraps a pair of iterators.

\item An Iterable type that hold its elements by \tcode{shared_ptr}
and shares ownership with all its copies.

\item An Iterable type that generates its elements on demand.
\end{itemize}

A container~(\cxxref{containers}) is not a Range since copying the
container copies the elements, which cannot be done in constant time.
\exitexample

\begin{codeblock}
template <Iterable T>
concept bool Range =
  Semiregular<T> && is_range<T>::value;
\end{codeblock}

\pnum
Since the difference between Iterable and Range is largely semantic, the
two are differentiated with the help of the \tcode{is_range} trait. Users may
specialize the \tcode{is_range} trait. By default, \tcode{is_range} uses the
following heuristics to determine whether an Iterable type \tcode{T} is a Range:

\begin{itemize}
\item If \tcode{T} derives from \tcode{range_base}, \tcode{is_range<T>::value}
is true.
\item If a top-level const changes \tcode{T}'s \tcode{IteratorType}'s
\tcode{ReferenceType} type, \tcode{is_range<T>::value}
is false. \enternote Deep constness implies element ownership, whereas shallow
constness implies reference semantics. \exitnote
\end{itemize}

\pnum
\enternote
Below is a possible implementation of the \tcode{is_range} trait.

\begin{codeblock}
struct range_base
{};

// For exposition only:
template <Iterable T>
concept bool ContainerLike_ =
  !Same<decltype(*begin(declval<T &>())),
        decltype(*begin(declval<T const &>()))>;

// For exposition only:
template<typename T>
struct is_range_impl_
  : std::integral_constant<
      bool,
      Iterable<T> && (!ContainerLike_<T> || Derived<T, range_base>)
    >
{};

template<typename T, typename Enable = void>
struct is_range
  : conditional<
      is_same<T, remove_const_t<remove_reference_t<T>>>::value,
      is_range_impl_<T>,
      is_range<remove_const_t<remove_reference_t<T>>>
    >::type
{};
\end{codeblock}
\ednote{The handling of top-level reference here is inconsistent with the other
type traits.}
\exitnote

\rSec3[bounded.iterables]{Bounded iterables}

\pnum
The \tcode{BoundedIterable} concept describes requirements of an Iterable type
for which \tcode{begin} and \tcode{end} return objects of the same type.
\enternote The standard containers~(\cxxref{containers}) are models of
\tcode{BoundedIterable}. \exitnote

\begin{codeblock}
template <Iterable T>
concept bool BoundedIterable =
  Same<IteratorType<T>, SentinelType<T>>;
\end{codeblock}

\rSec3[input.iterables]{Input iterables}

\pnum
The \tcode{InputIterable} concept describes requirements of an Iterable type
for which \tcode{begin} returns a model of
\tcode{InputIterator}~(\ref{input.iterators}).

\begin{codeblock}
template <Iterable T>
concept bool InputIterable =
  InputIterator<IteratorType<T>>;
\end{codeblock}

\rSec3[forward.iterables]{Forward iterables}

\pnum
The \tcode{ForwardIterable} concept describes requirements of an
InputIterable type for which \tcode{begin} returns a model of
\tcode{ForwardIterator}~(\ref{forward.iterators}).

\begin{codeblock}
template <InputIterable T>
concept bool ForwardIterable =
  ForwardIterator<IteratorType<T>>;
\end{codeblock}

\rSec3[bidirectional.iterables]{Bidirectional iterables}

\pnum
The \tcode{BidirectionalIterable} concept describes requirements of a
ForwardIterable type for which \tcode{begin} returns a model of
\tcode{BidirectionalIterator}~(\ref{bidirectional.iterators}).

\begin{codeblock}
template <ForwardIterable T>
concept bool BidirectionalIterable =
  BidirectionalIterator<IteratorType<T>>;
\end{codeblock}

\rSec3[random.access.iterables]{Random access iterables}

\pnum
The \tcode{RandomAccessIterable} concept describes requirements of a
BidirectionalIterable type for which \tcode{begin} returns a model of
\tcode{RandomAccessIterator}~(\ref{random.access.iterators}).

\begin{codeblock}
template <BidirectionalIterable T>
concept bool RandomAccessIterable =
  RandomAccessIterator<IteratorType<T>>;
\end{codeblock}

\end{addedblock}

\rSec1[iterator.range]{range access}

\pnum
In addition to being available via inclusion of the \tcode{<iterator>} header,
the function templates in \ref{iterator.range} are available when any of the following
headers are included: \tcode{<array>}, \tcode{<deque>}, \tcode{<forward_list>},
\tcode{<list>}, \tcode{<map>}, \tcode{<regex>}, \tcode{<set>}, \tcode{<string>},
\tcode{<unordered_map>}, \tcode{<unordered_set>}, and \tcode{<vector>}.

\indexlibrary{\idxcode{begin(C\&)}}%
\begin{itemdecl}
template <class C> auto begin(C& c) -> decltype(c.begin());
template <class C> auto begin(const C& c) -> decltype(c.begin());
\end{itemdecl}

\begin{itemdescr}
\pnum
\returns \tcode{c.begin()}.
\end{itemdescr}

\indexlibrary{\idxcode{end(C\&)}}%
\begin{itemdecl}
template <class C> auto end(C& c) -> decltype(c.end());
template <class C> auto end(const C& c) -> decltype(c.end());
\end{itemdecl}

\begin{itemdescr}
\pnum
\returns \tcode{c.end()}.
\end{itemdescr}

\indexlibrary{\idxcode{begin(T (\&)[N])}}%
\begin{itemdecl}
template <class T, size_t N> constexpr T* begin(T (&array)[N]) noexcept;
\end{itemdecl}

\begin{itemdescr}
\pnum
\returns \tcode{array}.
\end{itemdescr}

\indexlibrary{\idxcode{end(T (\&)[N])}}%
\begin{itemdecl}
template <class T, size_t N> constexpr T* end(T (&array)[N]) noexcept;
\end{itemdecl}

\begin{itemdescr}
\pnum
\returns \tcode{array + N}.
\end{itemdescr}

\indexlibrary{\idxcode{cbegin(const C\&)}}%
\begin{itemdecl}
template <class C> constexpr auto cbegin(const C& c) noexcept(noexcept(std::begin(c)))
  -> decltype(std::begin(c));
\end{itemdecl}
\begin{itemdescr}
\pnum \returns \tcode{std::begin(c)}.
\end{itemdescr}

\indexlibrary{\idxcode{cend(const C\&)}}%
\begin{itemdecl}
template <class C> constexpr auto cend(const C& c) noexcept(noexcept(std::end(c)))
  -> decltype(std::end(c));
\end{itemdecl}
\begin{itemdescr}
\pnum \returns \tcode{std::end(c)}.
\end{itemdescr}

\indexlibrary{\idxcode{rbegin(C\&)}}%
\begin{itemdecl}
template <class C> auto rbegin(C& c) -> decltype(c.rbegin());
template <class C> auto rbegin(const C& c) -> decltype(c.rbegin());
\end{itemdecl}
\begin{itemdescr}
\pnum \returns \tcode{c.rbegin()}.
\end{itemdescr}

\indexlibrary{\idxcode{rend(const C\&)}}%
\begin{itemdecl}
template <class C> auto rend(C& c) -> decltype(c.rend());
template <class C> auto rend(const C& c) -> decltype(c.rend());
\end{itemdecl}
\begin{itemdescr}
\pnum \returns \tcode{c.rend()}.
\end{itemdescr}

\indexlibrary{\idxcode{rbegin(T (\&array)[N])}}%
\begin{itemdecl}
template <class T, size_t N> reverse_iterator<T*> rbegin(T (&array)[N]);
\end{itemdecl}
\begin{itemdescr}
\pnum \returns \tcode{reverse_iterator<T*>(array + N)}.
\end{itemdescr}

\indexlibrary{\idxcode{rend(T (\&array)[N])}}%
\begin{itemdecl}
template <class T, size_t N> reverse_iterator<T*> rend(T (&array)[N]);
\end{itemdecl}
\begin{itemdescr}
\pnum \returns \tcode{reverse_iterator<T*>(array)}.
\end{itemdescr}

\indexlibrary{\idxcode{rbegin(initializer_list<E>)}}%
\begin{itemdecl}
template <class E> reverse_iterator<const E*> rbegin(initializer_list<E> il);
\end{itemdecl}
\begin{itemdescr}
\pnum \returns \tcode{reverse_iterator<const E*>(il.end())}.
\end{itemdescr}

\indexlibrary{\idxcode{rend(initializer_list<E>)}}%
\begin{itemdecl}
template <class E> reverse_iterator<const E*> rend(initializer_list<E> il);
\end{itemdecl}
\begin{itemdescr}
\pnum \returns \tcode{reverse_iterator<const E*>(il.begin())}.
\end{itemdescr}

\indexlibrary{\idxcode{crbegin(const C\& c)}}%
\begin{itemdecl}
template <class C> auto crbegin(const C& c) -> decltype(std::rbegin(c));
\end{itemdecl}
\begin{itemdescr}
\pnum \returns \tcode{std::rbegin(c)}.
\end{itemdescr}

\indexlibrary{\idxcode{crend(const C\& c)}}%
\begin{itemdecl}
template <class C> auto crend(const C& c) -> decltype(std::rend(c));
\end{itemdecl}
\begin{itemdescr}
\pnum \returns \tcode{std::rend(c)}.
\end{itemdescr}
