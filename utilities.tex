%!TEX root = std.tex
\setcounter{chapter}{19}
\rSec0[utilities]{General utilities library}

\setcounter{section}{1}
\rSec1[utility]{Utility components}

\ednote{Change the \tcode{<utility>} synopsis as follows:}

\begin{codeblock}
  // \ref{utility.swap}, swap:
  template<class T>
    @\newtxt{requires MoveConstructible<T> \&\& MoveAssignable<T> \&\& Destructible<T>}@
  void swap(T& a, T& b) noexcept(@\seebelow@);
  template <class T, size_t N>
    @\newtxt{requires requires (std::remove_all_extents_t<T>\& t) \{ swap(t, t); \}}@
  void swap(T (&a)[N], T (&b)[N]) noexcept(noexcept(swap(*a, *b)));}

  // \ref{utility.exchange}, exchange:
  template <class T, class U=T>
    @\newtxt{requires MoveConstructible<T> \&\& Assignable<T, U> \&\& Destructible<T>}@
  T exchange(T& obj, U&& new_val);
\end{codeblock}

\ednote{To the end of the \tcode{<utility>} synopsis, add the following:}

{\color{newclr}
\begin{codeblock}
  // \ref{taggedtup.tagged}, struct with named accessors
  template <class Base, class... Types> struct tagged;

  // \ref{tagged.astuple}, tuple-like access to tagged
  template <class Base, class... Tags> struct tuple_size<tagged<Base, Tags...>>;
  template <size_t N, class Base, class... Tags> struct tuple_element<N, tagged<Base, Tags...>>;

  // \ref{tagged.pair}, tagged pairs
  template <class T1, class T2> using tagged_pair = @\seebelow@;
\end{codeblock}
}

\setcounter{subsection}{1}
\rSec2[utility.swap]{swap}

\indexlibrary{\idxcode{swap}}%
\begin{itemdecl}
template<class T>
  @\newtxt{requires MoveConstructible<T> \&\& MoveAssignable<T> \&\& Destructible<T>}@
void swap(T& a, T& b) noexcept(@\seebelow@);
\end{itemdecl}

\begin{itemdescr}
\pnum
\remark The expression inside \tcode{noexcept} is equivalent to:

\begin{codeblock}
is_nothrow_move_constructible<T>::value &&
is_nothrow_move_assignable<T>::value
\end{codeblock}

\oldtxt{
\pnum
\requires
Type
\tcode{T}
shall be
\tcode{MoveConstructible} (Table~\ref{moveconstructible})
and
\tcode{MoveAssignable} (Table~\ref{moveassignable}).
}

\pnum
\effects
Exchanges values stored in two locations.
\end{itemdescr}

\indexlibrary{\idxcode{swap}}%
\begin{itemdecl}
template <class T, size_t N>
  @\newtxt{requires requires (std::remove_all_extents_t<T>\& t) \{ swap(t, t); \}}@
void swap(T (&a)[N], T (&b)[N]) noexcept(noexcept(swap(*a, *b)));}
\end{itemdecl}

\begin{itemdescr}
\oldtxt{
\pnum
\requires
\tcode{a[i]} shall be swappable with~(\ref{swappable.requirements}) \tcode{b[i]}
for all \tcode{i} in the range \range{0}{N}.
}

\pnum
\effects \tcode{swap_ranges(a, a + N, b)}
\end{itemdescr}

\rSec2[utility.exchange]{exchange}

\begin{itemdecl}
template <class T, class U=T>
  @\newtxt{requires MoveConstructible<T> \&\& Assignable<T, U> \&\& Destructible<T>}@
T exchange(T& obj, U&& new_val);
\end{itemdecl}

\begin{itemdescr}
\pnum
\effects
Equivalent to:

\begin{codeblock}
T old_val = std::move(obj);
obj = std::forward<U>(new_val);
return old_val;
\end{codeblock}
\end{itemdescr}

\setcounter{subsection}{5}

{\color{newclr}
\rSec2[taggedtup]{Tagged tuple-like types}

\rSec3[taggedtup.general]{In general}

\pnum The library provides a template for augmenting a tuple-like type with named element accessor
member functions. The library also provides several templates that provide access to \tcode{tagged}
objects as if they were \tcode{tuple} objects (see~\cxxref{tuple.elem}).

\ednote{This type exists so that the algorithms can return pair- and tuple-like objects with named
accessors, as requested by LEWG. Rather than create a bunch of one-off class types with no relation
to pair and tuple, I opted instead to create a general utility. I'll note that this functionality
can be merged into \tcode{pair} and \tcode{tuple} directly with minimal breakage, but I opted for
now to keep it separate.}

\rSec3[taggedtup.tagged]{Class template \tcode{tagged}}

Class template \tcode{tagged} augments a tuple-like class type by giving it named accessors. It is
used to define the alias templates \tcode{tagged_pair}~(\ref{pairs.tagged}) and
\tcode{tagged_tuple}~(\ref{tuple.tagged}).

\pnum In the class synopsis below, let $i$ be in the range
\range{0}{sizeof...(Tags)} in order and $T_i$ be the $i^{th}$ type in \tcode{Tags}, where indexing
is zero-based.

\indexlibrary{\idxcode{tagged}}%
\begin{codeblock}
// defined in header <utility>

namespace std {
  template <class Base, class... Tags>
  struct tagged :
    Base, @\textit{TAGGET}@(tagged<Base, Tags...>, @$T_i$@, @$i$@)... { // \seebelow
    using Base::Base;
    template <class U>
      requires Assignable<Base, U>
    tagged& operator=(U&& u);
  };
}
\end{codeblock}

\pnum The size of the \tcode{Tags} parameter pack shall be less than or equal to
\tcode{tuple_size<Base>::value}.

\pnum A \techterm{tagged getter} is an empty trivial class type that has a named member function that
returns a reference to a member of a tuple-like object that is assumed to be derived from the getter
class. The tuple-like type of a tagged getter is called its \techterm{DerivedCharacteristic}.
The index of the tuple element returned from the getter's member functions is called its
\techterm{ElementIndex}. The name of the getter's member function is called its
\techterm{ElementName}

\pnum A tagged getter class with DerivedCharacteristic \tcode{\textit{D}}, ElementIndex
\tcode{\textit{N}}, and ElementName \tcode{\textit{name}} shall provide the following interface:

\begin{codeblock}
struct @\xname{\textit{TAGGED_GETTER}}@ {
  constexpr decltype(auto) @$name$@() &       { return get<@$N$@>(static_cast<@$D$@&>(*this)); }
  constexpr decltype(auto) @$name$@() &&      { return get<@$N$@>(static_cast<@$D$@&&>(*this)); }
  constexpr decltype(auto) @$name$@() const & { return get<@$N$@>(static_cast<const @$D$@&>(*this)); }
};
\end{codeblock}

\pnum A \techterm{tag specifier} is a type that facilitates a mapping from a tuple-like type and an
element index into a \textit{tagged getter} that gives named access to the element at that index.
The types in the \tcode{Tags} parameter pack shall be tag specifiers. The tag specifiers in the
\tcode{Tags} parameter pack shall be unique. \enternote The mapping mechanism from tag specifier to
tagged getter is unspecified.\exitnote

\pnum Let \tcode{\textit{TAGGET}(D, T, $N$)} name a tagged getter type that gives named
access to the $N$-th element of the tuple-like type \tcode{D}.

\pnum It shall not be possible to delete an instance of class template \tcode{tagged} through a
pointer to any base other than \tcode{Base}.

\pnum A \techterm{tagged type} is a unary function type whose return type is a tag specifier. Let
\tcode{\textit{TAGSPEC}(F)} name the tag specifier of the tagged type \tcode{F}, and let
\tcode{\textit{TAGELEM}(F)} name the argument type of the tagged type \tcode{F}.

\indexlibrary{\idxcode{operator=}!\idxcode{tagged}}
\indexlibrary{\idxcode{tagged}!\idxcode{operator=}}
\begin{itemdecl}
template <class U>
  requires Assignable<Base, U>
tagged& operator=(U&& u);
\end{itemdecl}

\begin{itemdescr}
\pnum
\effects Assigns \tcode{forward<U>(u)} to \tcode{static_cast<Base\&>(*this)}.

\pnum
\returns \tcode{*this}.
\end{itemdescr}

\rSec3[tagged.astuple]{Tuple-like access to \tcode{tagged}}

\indexlibrary{\idxcode{tuple_size}}%
\begin{itemdecl}
template <class Base, class... Tags>
struct tuple_size<tagged<Base, Tags...>>
  : tuple_size<Base> { };
\end{itemdecl}

\indexlibrary{\idxcode{tuple_element}}%
\begin{itemdecl}
template <class Base, class... Tags>
struct tuple_element<N, tagged<Base, Tags...>>
  : tuple_element<N, Base> { };
\end{itemdecl}
}

\setcounter{section}{2}

\rSec1[pairs]{Pairs}

\setcounter{subsection}{1}

\rSec2[pairs.pair]{Class template \tcode{pair}}

\ednote{To the bottom of the \tcode{pair} synopsis, add the following:}

{\color{newclr}
\begin{codeblock}
namespace std {
  // ...
  template <class T1, class T2>
  using tagged_pair = tagged<pair<@\textit{TAGELEM}@(T1), @\textit{TAGELEM}@(T2)>,
                             @\textit{TAGSPEC}@(T1), @\textit{TAGSPEC}@(T2)>;
}
\end{codeblock}
}

\ednote{After [pair.piecewise], add the following sub-section:}

\setcounter{subsection}{5}

{\color{newclr}
\rSec2[pairs.tagged]{Alias template \tcode{tagged_pair}}

\begin{codeblock}
template <class T1, class T2>
using tagged_pair = tagged<pair<@\textit{TAGELEM}@(T1), @\textit{TAGELEM}@(T2)>,
                           @\textit{TAGSPEC}@(T1), @\textit{TAGSPEC}@(T2)>;
\end{codeblock}

\pnum Types \tcode{T1}  and \tcode{T2} shall be tagged types~(\ref{taggedtup.tagged}).

\pnum \enterexample
\begin{codeblock}
// See \ref{alg.tagspec}:
tagged_pair<tag::min(int), tag::max(int)> p{0, 1};
assert(&p.min() == &p.first);
assert(&p.max() == &p.second);
\end{codeblock}
\exitexample


\rSec3[pairs.tagged.creation]{Tagged pair creation functions}

\indexlibrary{\idxcode{make_pair}}%
\begin{itemdecl}
template <class Tag1, class Tag2, class T1, class T2>
  constexpr tagged_pair<Tag1(V1), Tag2(V2)> make_tagged_pair(T1&& x, T2&& y);
\end{itemdecl}

\begin{itemdescr}
\pnum
\returns \tcode{tagged_pair<Tag1(V1), Tag2(V2)>(std::forward<T1>(x), std::forward<T2>(y))};

where \tcode{V1} and \tcode{V2} are determined as follows: Let \tcode{Ui} be
\tcode{decay_t<Ti>} for each \tcode{Ti}. Then each \tcode{Vi} is \tcode{X\&}
if \tcode{Ui} equals \tcode{reference_wrapper<X>}, otherwise \tcode{Vi} is
\tcode{Ui}.

\pnum
\enterexample
In place of:

\begin{codeblock}
  return tagged_pair<tag::min(int), tag::max(double)>(5, 3.1415926);   // explicit types
\end{codeblock}

a \Cpp program may contain:

\begin{codeblock}
  return make_tagged_pair<tag::min, tag::max>(5, 3.1415926);           // types are deduced
\end{codeblock}
\exitexample
\end{itemdescr}
}

\rSec1[tuple]{Tuples}

\rSec2[tuple.general]{In general}

\ednote{To the bottom of the \tcode{<tuple>} synopsis, add the following:}

{\color{newclr}
\begin{codeblock}
namespace std {
  // ...
  template <class... Types>
  using tagged_tuple = tagged<tuple<@\textit{TAGELEM}@(Types)...>,
                              @\textit{TAGSPEC}@(Types)...>;
}
\end{codeblock}
}

\ednote{After [tuple.tuple], add the following sub-section:}

\setcounter{subsection}{2}

{\color{newclr}
\rSec2[tuple.tagged]{Alias template \tcode{tagged_tuple}}

\begin{codeblock}
template <class... Types>
using tagged_tuple = tagged<tuple<@\textit{TAGELEM}@(Types)...>,
                            @\textit{TAGSPEC}@(Types)...>;
\end{codeblock}

\pnum The types in parameter pack \tcode{Types} shall be tagged types~(\ref{taggedtup.tagged}).

\pnum \enterexample
\begin{codeblock}
// See \ref{alg.tagspec}:
tagged_tuple<tag::in(char*), tag::out(char*)> t{0, 0};
assert(&t.in() == &get<0>(t));
assert(&t.out() == &get<1>(t));
\end{codeblock}
\exitexample

\rSec3[tuple.tagged.creation]{Tagged tuple creation functions}

\pnum
In the function descriptions that follow, let $i$ be in the range \range{0}{sizeof...(Types)}
in order and let $T_i$ be the $i^{th}$ type in a template parameter pack named \tcode{Types},
where indexing is zero-based.

\indexlibrary{\idxcode{make_tagged_tuple}}%
\indexlibrary{\idxcode{tagged_tuple}!\idxcode{make_tagged_tuple}}%
\begin{itemdecl}
template<class...Tags, class... Types>
  constexpr tagged_tuple<Tags(@\textit{VTypes}@)...>> make_tagged_tuple(Types&&... t);
\end{itemdecl}

\begin{itemdescr} \pnum Let \tcode{$U_i$} be \tcode{decay_t<$T_i$>} for each
$T_i$ in \tcode{Types}. Then each $V_i$ in \tcode{VTypes} is
\tcode{X\&} if $U_i$ equals \tcode{reference_wrapper<X>}, otherwise
$V_i$ is $U_i$.

\pnum
\returns \tcode{tagged_tuple<Tags(VTypes)...>(std::forward<Types>(t)...)}.

\pnum
\enterexample

\begin{codeblock}
int i; float j;
make_tagged_tuple<tag::in1, tag::in2, tag::out>(1, ref(i), cref(j))
\end{codeblock}

creates a tagged tuple of type

\begin{codeblock}
tagged_tuple<tag::in1(int), tag::in2(int&), tag::out(const float&)>
\end{codeblock}
\exitexample
\end{itemdescr}
}

\setcounter{section}{8}
\rSec1[function.objects]{Function objects}

\ednote{To \tcode{<functional>} header synopsis, add \tcode{identity} function object.}

\begin{quote}
\setcounter{Paras}{1}
\pnum
\synopsis{Header \tcode{<functional>} synopsis}

\begin{codeblock}
  @\added{struct identity;}@
\end{codeblock}
\end{quote}

\ednote{Add constraints to the comparison function objects.}

\begin{quote}
\setcounter{subsection}{5}
\rSec2[comparisons]{Comparisons}

\pnum
The library provides basic function object classes for all of the comparison
operators in the language~(\cxxref{expr.rel}, \cxxref{expr.eq}).

\indexlibrary{\idxcode{equal_to}}%
\begin{itemdecl}
template <class T = void>
  @\added{requires EqualityComparable<T>() || Same<T, void>}@
struct equal_to {
  constexpr bool operator()(const T& x, const T& y) const;
  typedef T first_argument_type;
  typedef T second_argument_type;
  typedef bool result_type;
};
\end{itemdecl}

\begin{itemdescr}
\pnum
\tcode{operator()} returns \tcode{x == y}.
\end{itemdescr}

\indexlibrary{\idxcode{not_equal_to}}%
\begin{itemdecl}
template <class T = void>
  @\added{requires EqualityComparable<T>() || Same<T, void>}@
struct not_equal_to {
  constexpr bool operator()(const T& x, const T& y) const;
  typedef T first_argument_type;
  typedef T second_argument_type;
  typedef bool result_type;
};
\end{itemdecl}

\begin{itemdescr}
\pnum
\tcode{operator()} returns \tcode{x != y}.
\end{itemdescr}

\indexlibrary{\idxcode{greater}}%
\begin{itemdecl}
template <class T = void>
  @\added{requires TotallyOrdered<T>() || Same<T, void>}@
struct greater {
  constexpr bool operator()(const T& x, const T& y) const;
  typedef T first_argument_type;
  typedef T second_argument_type;
  typedef bool result_type;
};
\end{itemdecl}

\begin{itemdescr}
\pnum
\tcode{operator()} returns \tcode{x > y}.
\end{itemdescr}

\indexlibrary{\idxcode{less}}%
\begin{itemdecl}
template <class T = void>
  @\added{requires TotallyOrdered<T>() || Same<T, void>}@
struct less {
  constexpr bool operator()(const T& x, const T& y) const;
  typedef T first_argument_type;
  typedef T second_argument_type;
  typedef bool result_type;
};
\end{itemdecl}

\begin{itemdescr}
\pnum
\tcode{operator()} returns \tcode{x < y}.
\end{itemdescr}

\indexlibrary{\idxcode{greater_equal}}%
\begin{itemdecl}
template <class T = void>
  @\added{requires TotallyOrdered<T>() || Same<T, void>}@
struct greater_equal {
  constexpr bool operator()(const T& x, const T& y) const;
  typedef T first_argument_type;
  typedef T second_argument_type;
  typedef bool result_type;
};
\end{itemdecl}

\begin{itemdescr}
\pnum
\tcode{operator()} returns \tcode{x >= y}.
\end{itemdescr}

\indexlibrary{\idxcode{less_equal}}%
\begin{itemdecl}
template <class T = void>
  @\added{requires TotallyOrdered<T>() || Same<T, void>}@
struct less_equal {
  constexpr bool operator()(const T& x, const T& y) const;
  typedef T first_argument_type;
  typedef T second_argument_type;
  typedef bool result_type;
};
\end{itemdecl}

\begin{itemdescr}
\pnum
\tcode{operator()} returns \tcode{x <= y}.
\end{itemdescr}

\indexlibrary{\idxcode{equal_to<>}}%
\begin{itemdecl}
template <> struct equal_to<void> {
  template <class T, class U>
    @\added{requires EqualityComparable<T, U>()}@
  constexpr @\oldtxt{bool}\newtxt{auto}@ operator()(T&& t, U&& u) const
    @\newtxt{-> decltype(std::forward<T>(t) == std::forward<U>(u))}@;

  typedef @\unspec@ is_transparent;
};
\end{itemdecl}

\begin{itemdescr}
\pnum
\tcode{operator()} returns \tcode{std::forward<T>(t) == std::forward<U>(u)}.
\end{itemdescr}

\indexlibrary{\idxcode{not_equal_to<>}}%
\begin{itemdecl}
template <> struct not_equal_to<void> {
  template <class T, class U>
    @\added{requires EqualityComparable<T, U>()}@
  constexpr @\oldtxt{bool}\newtxt{auto}@ operator()(T&& t, U&& u) const
    @\newtxt{-> decltype(std::forward<T>(t) != std::forward<U>(u))}@;

  typedef @\unspec@ is_transparent;
};
\end{itemdecl}

\begin{itemdescr}
\pnum
\tcode{operator()} returns \tcode{std::forward<T>(t) != std::forward<U>(u)}.
\end{itemdescr}

\indexlibrary{\idxcode{greater<>}}%
\begin{itemdecl}
template <> struct greater<void> {
  template <class T, class U>
    @\added{requires TotallyOrdered<T, U>()}@
  constexpr @\oldtxt{bool}\newtxt{auto}@ operator()(T&& t, U&& u) const
    @\newtxt{-> decltype(std::forward<T>(t) > std::forward<U>(u))}@;

  typedef @\unspec@ is_transparent;
};
\end{itemdecl}

\begin{itemdescr}
\pnum
\tcode{operator()} returns \tcode{std::forward<T>(t) > std::forward<U>(u)}.
\end{itemdescr}

\indexlibrary{\idxcode{less<>}}%
\begin{itemdecl}
template <> struct less<void> {
  template <class T, class U>
    @\added{requires TotallyOrdered<T, U>()}@
  constexpr @\oldtxt{bool}\newtxt{auto}@ operator()(T&& t, U&& u) const
    @\newtxt{-> decltype(std::forward<T>(t) < std::forward<U>(u))}@;

  typedef @\unspec@ is_transparent;
};
\end{itemdecl}

\begin{itemdescr}
\pnum
\tcode{operator()} returns \tcode{std::forward<T>(t) < std::forward<U>(u)}.
\end{itemdescr}

\indexlibrary{\idxcode{greater_equal<>}}%
\begin{itemdecl}
template <> struct greater_equal<void> {
  template <class T, class U>
    @\added{requires TotallyOrdered<T, U>()}@
  constexpr @\oldtxt{bool}\newtxt{auto}@ operator()(T&& t, U&& u) const
    @\newtxt{-> decltype(std::forward<T>(t) >= std::forward<U>(u))}@;

  typedef @\unspec@ is_transparent;
};
\end{itemdecl}

\begin{itemdescr}
\pnum
\tcode{operator()} returns \tcode{std::forward<T>(t) >= std::forward<U>(u)}.
\end{itemdescr}

\indexlibrary{\idxcode{less_equal<>}}%
\begin{itemdecl}
template <> struct less_equal<void> {
  template <class T, class U>
    @\added{requires TotallyOrdered<T, U>()}@
  constexpr @\oldtxt{bool}\newtxt{auto}@ operator()(T&& t, U&& u) const
    @\newtxt{-> decltype(std::forward<T>(t) <= std::forward<U>(u))}@;

  typedef @\unspec@ is_transparent;
};
\end{itemdecl}

\begin{itemdescr}
\pnum
\tcode{operator()} returns \tcode{std::forward<T>(t) <= std::forward<U>(u)}.
\end{itemdescr}

\pnum
For templates \tcode{greater}, \tcode{less}, \tcode{greater_equal}, and
\tcode{less_equal}, the specializations for any pointer type yield a total order,
even if the built-in operators \tcode{<}, \tcode{>}, \tcode{<=}, \tcode{>=}
do not.
\end{quote}

\ednote{After subsection 20.9.12 [unord.hash] add the following subsection:}

\setcounter{subsection}{12}
\begin{addedblock}
\rSec2[func.identity]{Class identity}

\indexlibrary{\idxcode{identity}}%
\begin{itemdecl}
struct identity {
  template <class T>
  constexpr T&& operator()(T&& t) const noexcept;
};
\end{itemdecl}

\begin{itemdescr}
\pnum
\tcode{operator()} returns \tcode{std::forward<T>(t)}.

\ednote{REVIEW: From Stephan T. Lavavej: "[This] \tcode{identity} functor, being
a non-template, clashes with any attempt to provide \tcode{identity<T>::type}." <Insert
bikeshed naming discussion here>.}
\end{itemdescr}
\end{addedblock}
