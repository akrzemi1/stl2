%!TEX root = std.tex
\setcounter{chapter}{19}
\rSec0[utilities]{General utilities library}

\setcounter{section}{8}
\rSec1[function.objects]{Function objects}

\ednote{To \tcode{<functional>} header synopsis, add \tcode{identity} function object.}

\begin{quote}
\setcounter{Paras}{1}
\pnum
\synopsis{Header \tcode{<functional>} synopsis}

\begin{codeblock}
  @\added{struct identity;}@
\end{codeblock}
\end{quote}

\ednote{Add constraints to the comparison function objects.}

\begin{quote}
\setcounter{subsection}{5}
\rSec2[comparisons]{Comparisons}

\pnum
The library provides basic function object classes for all of the comparison
operators in the language~(\cxxref{expr.rel}, \cxxref{expr.eq}).

\indexlibrary{\idxcode{equal_to}}%
\begin{itemdecl}
template <class T = void> struct equal_to {
  @\added{requires EqualityComparable<T>()}@
  constexpr bool operator()(const T& x, const T& y) const;
  typedef T first_argument_type;
  typedef T second_argument_type;
  typedef bool result_type;
};
\end{itemdecl}

\begin{itemdescr}
\pnum
\tcode{operator()} returns \tcode{x == y}.
\end{itemdescr}

\indexlibrary{\idxcode{not_equal_to}}%
\begin{itemdecl}
template <class T = void> struct not_equal_to {
  @\added{requires EqualityComparable<T>()}@
  constexpr bool operator()(const T& x, const T& y) const;
  typedef T first_argument_type;
  typedef T second_argument_type;
  typedef bool result_type;
};
\end{itemdecl}

\begin{itemdescr}
\pnum
\tcode{operator()} returns \tcode{x != y}.
\end{itemdescr}

\indexlibrary{\idxcode{greater}}%
\begin{itemdecl}
template <class T = void> struct greater {
  @\added{requires TotallyOrdered<T>()}@
  constexpr bool operator()(const T& x, const T& y) const;
  typedef T first_argument_type;
  typedef T second_argument_type;
  typedef bool result_type;
};
\end{itemdecl}

\begin{itemdescr}
\pnum
\tcode{operator()} returns \tcode{x > y}.
\end{itemdescr}

\indexlibrary{\idxcode{less}}%
\begin{itemdecl}
template <class T = void> struct less {
  @\added{requires TotallyOrdered<T>()}@
  constexpr bool operator()(const T& x, const T& y) const;
  typedef T first_argument_type;
  typedef T second_argument_type;
  typedef bool result_type;
};
\end{itemdecl}

\begin{itemdescr}
\pnum
\tcode{operator()} returns \tcode{x < y}.
\end{itemdescr}

\indexlibrary{\idxcode{greater_equal}}%
\begin{itemdecl}
template <class T = void> struct greater_equal {
  @\added{requires TotallyOrdered<T>()}@
  constexpr bool operator()(const T& x, const T& y) const;
  typedef T first_argument_type;
  typedef T second_argument_type;
  typedef bool result_type;
};
\end{itemdecl}

\begin{itemdescr}
\pnum
\tcode{operator()} returns \tcode{x >= y}.
\end{itemdescr}

\indexlibrary{\idxcode{less_equal}}%
\begin{itemdecl}
template <class T = void> struct less_equal {
  @\added{requires TotallyOrdered<T>()}@
  constexpr bool operator()(const T& x, const T& y) const;
  typedef T first_argument_type;
  typedef T second_argument_type;
  typedef bool result_type;
};
\end{itemdecl}

\begin{itemdescr}
\pnum
\tcode{operator()} returns \tcode{x <= y}.
\end{itemdescr}

\indexlibrary{\idxcode{equal_to<>}}%
\begin{itemdecl}
template <> struct equal_to<void> {
  template <class T, class U>
    @\added{requires EqualityComparable<T, U>()}@
  constexpr @\changed{auto}{bool}@ operator()(T&& t, U&& u) const
    @\removed{-> decltype(std::forward<T>(t) == std::forward<U>(u))}@;

  typedef @\unspec@ is_transparent;
};
\end{itemdecl}

\begin{itemdescr}
\pnum
\tcode{operator()} returns \tcode{std::forward<T>(t) == std::forward<U>(u)}.
\end{itemdescr}

\indexlibrary{\idxcode{not_equal_to<>}}%
\begin{itemdecl}
template <> struct not_equal_to<void> {
  template <class T, class U>
    @\added{requires EqualityComparable<T, U>()}@
  constexpr @\changed{auto}{bool}@ operator()(T&& t, U&& u) const
    @\removed{-> decltype(std::forward<T>(t) != std::forward<U>(u))}@;

  typedef @\unspec@ is_transparent;
};
\end{itemdecl}

\begin{itemdescr}
\pnum
\tcode{operator()} returns \tcode{std::forward<T>(t) != std::forward<U>(u)}.
\end{itemdescr}

\indexlibrary{\idxcode{greater<>}}%
\begin{itemdecl}
template <> struct greater<void> {
  template <class T, class U>
    @\added{requires TotallyOrdered<T, U>()}@
  constexpr @\changed{auto}{bool}@ operator()(T&& t, U&& u) const
    @\removed{-> decltype(std::forward<T>(t) > std::forward<U>(u))}@;

  typedef @\unspec@ is_transparent;
};
\end{itemdecl}

\begin{itemdescr}
\pnum
\tcode{operator()} returns \tcode{std::forward<T>(t) > std::forward<U>(u)}.
\end{itemdescr}

\indexlibrary{\idxcode{less<>}}%
\begin{itemdecl}
template <> struct less<void> {
  template <class T, class U>
    @\added{requires TotallyOrdered<T, U>()}@
  constexpr @\changed{auto}{bool}@ operator()(T&& t, U&& u) const
    @\removed{-> decltype(std::forward<T>(t) < std::forward<U>(u))}@;

  typedef @\unspec@ is_transparent;
};
\end{itemdecl}

\begin{itemdescr}
\pnum
\tcode{operator()} returns \tcode{std::forward<T>(t) < std::forward<U>(u)}.
\end{itemdescr}

\indexlibrary{\idxcode{greater_equal<>}}%
\begin{itemdecl}
template <> struct greater_equal<void> {
  template <class T, class U>
    @\added{requires TotallyOrdered<T, U>()}@
  constexpr @\changed{auto}{bool}@ operator()(T&& t, U&& u) const
    @\removed{-> decltype(std::forward<T>(t) >= std::forward<U>(u))}@;

  typedef @\unspec@ is_transparent;
};
\end{itemdecl}

\begin{itemdescr}
\pnum
\tcode{operator()} returns \tcode{std::forward<T>(t) >= std::forward<U>(u)}.
\end{itemdescr}

\indexlibrary{\idxcode{less_equal<>}}%
\begin{itemdecl}
template <> struct less_equal<void> {
  template <class T, class U>
    @\added{requires TotallyOrdered<T, U>()}@
  constexpr @\changed{auto}{bool}@ operator()(T&& t, U&& u) const
    @\removed{-> decltype(std::forward<T>(t) <= std::forward<U>(u))}@;

  typedef @\unspec@ is_transparent;
};
\end{itemdecl}

\begin{itemdescr}
\pnum
\tcode{operator()} returns \tcode{std::forward<T>(t) <= std::forward<U>(u)}.
\end{itemdescr}

\pnum
For templates \tcode{greater}, \tcode{less}, \tcode{greater_equal}, and
\tcode{less_equal}, the specializations for any pointer type yield a total order,
even if the built-in operators \tcode{<}, \tcode{>}, \tcode{<=}, \tcode{>=}
do not.
\end{quote}

\ednote{After subsection 20.9.12 [unord.hash] add the following subsection:}

\setcounter{subsection}{12}
\begin{addedblock}
\rSec2[func.identity]{Class identity}

\indexlibrary{\idxcode{identity}}%
\begin{itemdecl}
struct identity {
  template <class T>
  constexpr T&& operator()(T&& t) const noexcept;
};
\end{itemdecl}

\begin{itemdescr}
\pnum
\tcode{operator()} returns \tcode{std::forward<T>(t)}.
\end{itemdescr}
\end{addedblock}
