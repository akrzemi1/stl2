%!TEX root = std.tex
\setcounter{chapter}{19}
\rSec0[utilities]{General utilities library}

\setcounter{section}{1}
\rSec1[utility]{Utility components}

\ednote{Change the \tcode{<utility>} synopsis as follows:}

\begin{codeblock}
  // \ref{utility.swap}, swap:
  template<class T>
    @\newtxt{requires MoveConstructible<T> \&\& MoveAssignable<T>}@
  void swap(T& a, T& b) noexcept(@\seebelow@);
  template <class T, size_t N>
    @\newtxt{requires Swappable<T\&>}@
  void swap(T (&a)[N], T (&b)[N]) noexcept(noexcept(swap(*a, *b)));}

  // \ref{utility.exchange}, exchange:
  template <class T, class U=T>
    @\newtxt{requires MoveConstructible<T> \&\& Assignable<T, U>}@
  T exchange(T& obj, U&& new_val);
\end{codeblock}

\setcounter{subsection}{1}
\rSec2[utility.swap]{swap}

\indexlibrary{\idxcode{swap}}%
\begin{itemdecl}
template<class T>
  @\newtxt{requires MoveConstructible<T> \&\& MoveAssignable<T>}@
void swap(T& a, T& b) noexcept(@\seebelow@);
\end{itemdecl}

\begin{itemdescr}
\pnum
\remark The expression inside \tcode{noexcept} is equivalent to:

\begin{codeblock}
is_nothrow_move_constructible<T>::value &&
is_nothrow_move_assignable<T>::value
\end{codeblock}

\oldtxt{
\pnum
\requires
Type
\tcode{T}
shall be
\tcode{MoveConstructible} (Table~\ref{moveconstructible})
and
\tcode{MoveAssignable} (Table~\ref{moveassignable}).
}

\pnum
\effects
Exchanges values stored in two locations.
\end{itemdescr}

\indexlibrary{\idxcode{swap}}%
\begin{itemdecl}
template <class T, size_t N>
  @\newtxt{requires Swappable<T\&>}@
void swap(T (&a)[N], T (&b)[N]) noexcept(noexcept(swap(*a, *b)));}
\end{itemdecl}

\begin{itemdescr}
\oldtxt{
\pnum
\requires
\tcode{a[i]} shall be swappable with~(\ref{swappable.requirements}) \tcode{b[i]}
for all \tcode{i} in the range \range{0}{N}.
}

\pnum
\effects \tcode{swap_ranges(a, a + N, b)}
\end{itemdescr}

\rSec2[utility.exchange]{exchange}

\begin{itemdecl}
template <class T, class U=T>
  @\newtxt{requires MoveConstructible<T> \&\& Assignable<T, U>}@
T exchange(T& obj, U&& new_val);
\end{itemdecl}

\begin{itemdescr}
\pnum
\effects
Equivalent to:

\begin{codeblock}
T old_val = std::move(obj);
obj = std::forward<U>(new_val);
return old_val;
\end{codeblock}
\end{itemdescr}

\setcounter{section}{8}
\rSec1[function.objects]{Function objects}

\ednote{To \tcode{<functional>} header synopsis, add \tcode{identity} function object.}

\begin{quote}
\setcounter{Paras}{1}
\pnum
\synopsis{Header \tcode{<functional>} synopsis}

\begin{codeblock}
  @\added{struct identity;}@
\end{codeblock}
\end{quote}

\ednote{Add constraints to the comparison function objects.}

\begin{quote}
\setcounter{subsection}{5}
\rSec2[comparisons]{Comparisons}

\pnum
The library provides basic function object classes for all of the comparison
operators in the language~(\cxxref{expr.rel}, \cxxref{expr.eq}).

\indexlibrary{\idxcode{equal_to}}%
\begin{itemdecl}
template <class T = void>
  @\added{requires EqualityComparable<T>() || Same<T, void>}@
struct equal_to {
  constexpr bool operator()(const T& x, const T& y) const;
  typedef T first_argument_type;
  typedef T second_argument_type;
  typedef bool result_type;
};
\end{itemdecl}

\begin{itemdescr}
\pnum
\tcode{operator()} returns \tcode{x == y}.
\end{itemdescr}

\indexlibrary{\idxcode{not_equal_to}}%
\begin{itemdecl}
template <class T = void>
  @\added{requires EqualityComparable<T>() || Same<T, void>}@
struct not_equal_to {
  constexpr bool operator()(const T& x, const T& y) const;
  typedef T first_argument_type;
  typedef T second_argument_type;
  typedef bool result_type;
};
\end{itemdecl}

\begin{itemdescr}
\pnum
\tcode{operator()} returns \tcode{x != y}.
\end{itemdescr}

\indexlibrary{\idxcode{greater}}%
\begin{itemdecl}
template <class T = void>
  @\added{requires TotallyOrdered<T>() || Same<T, void>}@
struct greater {
  constexpr bool operator()(const T& x, const T& y) const;
  typedef T first_argument_type;
  typedef T second_argument_type;
  typedef bool result_type;
};
\end{itemdecl}

\begin{itemdescr}
\pnum
\tcode{operator()} returns \tcode{x > y}.
\end{itemdescr}

\indexlibrary{\idxcode{less}}%
\begin{itemdecl}
template <class T = void>
  @\added{requires TotallyOrdered<T>() || Same<T, void>}@
struct less {
  constexpr bool operator()(const T& x, const T& y) const;
  typedef T first_argument_type;
  typedef T second_argument_type;
  typedef bool result_type;
};
\end{itemdecl}

\begin{itemdescr}
\pnum
\tcode{operator()} returns \tcode{x < y}.
\end{itemdescr}

\indexlibrary{\idxcode{greater_equal}}%
\begin{itemdecl}
template <class T = void>
  @\added{requires TotallyOrdered<T>() || Same<T, void>}@
struct greater_equal {
  constexpr bool operator()(const T& x, const T& y) const;
  typedef T first_argument_type;
  typedef T second_argument_type;
  typedef bool result_type;
};
\end{itemdecl}

\begin{itemdescr}
\pnum
\tcode{operator()} returns \tcode{x >= y}.
\end{itemdescr}

\indexlibrary{\idxcode{less_equal}}%
\begin{itemdecl}
template <class T = void>
  @\added{requires TotallyOrdered<T>() || Same<T, void>}@
struct less_equal {
  constexpr bool operator()(const T& x, const T& y) const;
  typedef T first_argument_type;
  typedef T second_argument_type;
  typedef bool result_type;
};
\end{itemdecl}

\begin{itemdescr}
\pnum
\tcode{operator()} returns \tcode{x <= y}.
\end{itemdescr}

\indexlibrary{\idxcode{equal_to<>}}%
\begin{itemdecl}
template <> struct equal_to<void> {
  template <class T, class U>
    @\added{requires EqualityComparable<T, U>()}@
  constexpr @\oldtxt{bool}\newtxt{auto}@ operator()(T&& t, U&& u) const
    @\newtxt{-> decltype(std::forward<T>(t) == std::forward<U>(u))}@;

  typedef @\unspec@ is_transparent;
};
\end{itemdecl}

\begin{itemdescr}
\pnum
\tcode{operator()} returns \tcode{std::forward<T>(t) == std::forward<U>(u)}.
\end{itemdescr}

\indexlibrary{\idxcode{not_equal_to<>}}%
\begin{itemdecl}
template <> struct not_equal_to<void> {
  template <class T, class U>
    @\added{requires EqualityComparable<T, U>()}@
  constexpr @\oldtxt{bool}\newtxt{auto}@ operator()(T&& t, U&& u) const
    @\newtxt{-> decltype(std::forward<T>(t) != std::forward<U>(u))}@;

  typedef @\unspec@ is_transparent;
};
\end{itemdecl}

\begin{itemdescr}
\pnum
\tcode{operator()} returns \tcode{std::forward<T>(t) != std::forward<U>(u)}.
\end{itemdescr}

\indexlibrary{\idxcode{greater<>}}%
\begin{itemdecl}
template <> struct greater<void> {
  template <class T, class U>
    @\added{requires TotallyOrdered<T, U>()}@
  constexpr @\oldtxt{bool}\newtxt{auto}@ operator()(T&& t, U&& u) const
    @\newtxt{-> decltype(std::forward<T>(t) > std::forward<U>(u))}@;

  typedef @\unspec@ is_transparent;
};
\end{itemdecl}

\begin{itemdescr}
\pnum
\tcode{operator()} returns \tcode{std::forward<T>(t) > std::forward<U>(u)}.
\end{itemdescr}

\indexlibrary{\idxcode{less<>}}%
\begin{itemdecl}
template <> struct less<void> {
  template <class T, class U>
    @\added{requires TotallyOrdered<T, U>()}@
  constexpr @\oldtxt{bool}\newtxt{auto}@ operator()(T&& t, U&& u) const
    @\newtxt{-> decltype(std::forward<T>(t) < std::forward<U>(u))}@;

  typedef @\unspec@ is_transparent;
};
\end{itemdecl}

\begin{itemdescr}
\pnum
\tcode{operator()} returns \tcode{std::forward<T>(t) < std::forward<U>(u)}.
\end{itemdescr}

\indexlibrary{\idxcode{greater_equal<>}}%
\begin{itemdecl}
template <> struct greater_equal<void> {
  template <class T, class U>
    @\added{requires TotallyOrdered<T, U>()}@
  constexpr @\oldtxt{bool}\newtxt{auto}@ operator()(T&& t, U&& u) const
    @\newtxt{-> decltype(std::forward<T>(t) >= std::forward<U>(u))}@;

  typedef @\unspec@ is_transparent;
};
\end{itemdecl}

\begin{itemdescr}
\pnum
\tcode{operator()} returns \tcode{std::forward<T>(t) >= std::forward<U>(u)}.
\end{itemdescr}

\indexlibrary{\idxcode{less_equal<>}}%
\begin{itemdecl}
template <> struct less_equal<void> {
  template <class T, class U>
    @\added{requires TotallyOrdered<T, U>()}@
  constexpr @\oldtxt{bool}\newtxt{auto}@ operator()(T&& t, U&& u) const
    @\newtxt{-> decltype(std::forward<T>(t) <= std::forward<U>(u))}@;

  typedef @\unspec@ is_transparent;
};
\end{itemdecl}

\begin{itemdescr}
\pnum
\tcode{operator()} returns \tcode{std::forward<T>(t) <= std::forward<U>(u)}.
\end{itemdescr}

\pnum
For templates \tcode{greater}, \tcode{less}, \tcode{greater_equal}, and
\tcode{less_equal}, the specializations for any pointer type yield a total order,
even if the built-in operators \tcode{<}, \tcode{>}, \tcode{<=}, \tcode{>=}
do not.
\end{quote}

\ednote{After subsection 20.9.12 [unord.hash] add the following subsection:}

\setcounter{subsection}{12}
\begin{addedblock}
\rSec2[func.identity]{Class identity}

\indexlibrary{\idxcode{identity}}%
\begin{itemdecl}
struct identity {
  template <class T>
  constexpr T&& operator()(T&& t) const noexcept;
};
\end{itemdecl}

\begin{itemdescr}
\pnum
\tcode{operator()} returns \tcode{std::forward<T>(t)}.

\ednote{REVIEW: From Stephan T. Lavavej: "[This] \tcode{identity} functor, being
a non-template, clashes with any attempt to provide \tcode{identity<T>::type}." <Insert
bikeshed naming discussion here>.}
\end{itemdescr}
\end{addedblock}
