\rSec0[numerics]{Numerics library}

\setcounter{section}{4}
\rSec1[rand]{Random number generation}

\rSec2[rand.req]{Requirements}

%%%%%%%%%%%%%%%%%%%%%%%%%%%%%%%%%%%%%%%%%%%%%%%%%%%%%%%%%%%%%%%%%%%%%%
% Uniform Random Number Generator requirements:

\setcounter{subsubsection}{2}
\rSec3[rand.req.urng]{Uniform random number generator requirements}%
\indextext{uniform random number generator!requirements|(}%
\indextext{requirements!uniform random number generator|(}

\begin{addedblock}
\begin{codeblock}
namespace std {
  template <class G>
  concept bool UniformRandomNumberGenerator =
    requires(G g) {
      typename ResultType<G>;
      requires UnsignedIntegral<ResultType<G>>;
      { g() } -> Same<ResultType<G>>;
      { G::min() } -> Same<ResultType<G>>;
      { G::max() } -> Same<ResultType<G>>;
    };
}
\end{codeblock}
\end{addedblock}

\pnum
A \techterm{uniform random number generator}
\tcode{g} of type \tcode{G}
is a function object
returning unsigned integer values
such that each value
in the range of possible results
has (ideally) equal probability
of being returned.
\enternote
 The degree to which \tcode{g}'s results
 approximate the ideal
 is often determined statistically.
\exitnote

\ednote{Remove para 2 and 3 and Table 116 (Uniform random number generator requirements).}

\begin{addedblock}
\pnum
Let \tcode{g} be any object of type \tcode{G}. Then type \tcode{G} models
\tcode{UniformRandomNumberGenerator} if and only if

\begin{itemize}
\item Both \tcode{G::min()} and \tcode{G::max()} are constant expressions~(\cxxref{expr.const}).
\item \tcode{(G::min() < G::max()) != false}.
\item \tcode{(G::min() <= g() <= G::max()) != false}.
\item \tcode{g()} has amortized constant complexity.
\end{itemize}

\end{addedblock}