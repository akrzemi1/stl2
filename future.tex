%!TEX root = stl2-ts.tex
\infannex{future}{Future Work}

\pnum
This document brings ranges and concepts to a minimal useful subset of the Standard Library.
A proper and full implementation has many more moving parts. In addition, we can use the opportunity
this work presents to address long-standing shortcomings in the Standard Library. Some of these
future work items are described below.

\rSec1[future.experience]{Implementation Experience with Concepts}

\pnum
The ideas presented here have been vetted in the range-v3~(\cite{range-v3}) library, which has seen both heavy
development and heavy use over the past year and a half. However, this library is implemented in
C++11 with the help of a library-based emulation layer for Concepts Lite. Andrew Sutton's origin~(\cite{origin})
library implements many of these ideas using real concepts, but it's a subset of the work presented
here. A critically important piece of this work will be to fully implement this design in C++17 with
a compiler that supports Concepts.

\rSec1[future.rvalrngs]{Algorithms and Rvalue Ranges}

\pnum
As presented in this document, the vast majority of algorithms do not accept rvalue Iterables.
This shortcoming becomes more painful with the addition of range views~(\ref{future.views}).
The issue is one of safety. Those algorithms that return iterators into the input range, like
\tcode{find}, can return iterators that will immediately be made invalid once the temporary
Iterable they point into gets destroyed -- but not for \textit{all} Iterables. For an Iterable that
is little more than a pair of iterators, it is perfectly safe to use an iterator into the Iterable
after the Iterable has been destroyed. The iterator's validity is not tied to the Iterable's
lifetime at all.

\pnum
The range-v3 library~(\cite{range-v3}) on which this proposal is based has recently implemented an
experimental solution to this problem. The library provides a wrapper called \tcode{dangling}
that the algorithms can use to mark returned iterators that point into rvalue ranges. It works
as follows:

\begin{codeblock}
  template <class T>
  struct dangling {
    dangling() = default;
    dangling(T t) : t_(t) {}            // implicit
    T get_unsafe() const { return t_; } // explicit
  private:
    T t_;
  };

  template <Iterable Rng>
  using safe_iterator_t =
    std::conditional_t<
      std::is_lvalue_reference<Rng>::value,
      IteratorType<Rng>,
      dangling<IteratorType<Rng>>>;

  template <InputIterable Rng, class T, class Proj = identity>
    requires //...
    safe_iterator_t<Rng>
      find(Rng&& rng, const T& value, Proj proj = Proj{}) {
        return find(begin(rng), end(rng), value, std::move(proj));
      }
\end{codeblock}

\pnum
Of particular note is that a \tcode{dangling<Iterator>} is \textit{not} an iterator. In order to
obtain an iterator into an expired Iterable, the user must explicitly call a method called
\tcode{get_unsafe}. That makes it impossible for a user to accidentally use an iterator that may be
invalid.

\pnum
An extension of this idea would be to define \tcode{safe_iterator_t} in terms of a trait that can be
specialized on user-defined Iterable types. That way, Iterables that guarantee their iterators can
outlive the Iterable will never have their iterators wrapped in \tcode{dangling}.

\pnum
If time shows that the range-v3 solution is workable, a future paper can integrate the new
functionality into this document.

\rSec1[future.proxy]{Proxy Iterators}

\pnum
As early at 1998 when Herb Sutter published his ``When is a Container not a Container''~\cite{Sutter1998}
article, it was known that proxy iterators were a challenge that the current iterator concept
hierarchy could not meet. The problem stems from the fact that the ForwardIterator concept as
specified in the current standard requires the iterator's reference type to be a true reference, not
a proxy. The Palo Alto report lifts this restriction in its respecification of the iterator
concepts but doesn't actually solve the problem. The majority of algorithms, once you study the
constraints, do not accept many interesting proxy iterator types.

\pnum
The author of this document has researched a possible library-only solution to the problem and
implemented it in the range-v3 library. (The details of that solution are described in a series of
blog posts beginning with ``To Be or Not to Be (an Iterator),'' Niebler, 2015~\cite{niebler2015}.) This
document does not reflect the results of that research.

Whether and how to best support proxy iterators is left as future work.

\rSec1[future.iterator_range]{Iterator Range Type}

\pnum
This paper does not define a concrete type for storing an iterator and a sentinel that models the
\tcode{Range} concept. Such a type, like that presented in ``A minimal std::range<Iter>,''~(\ref{TODO})
by J. Yasskin would be an obvious addition. Algorithms like \tcode{equal_range} and
\tcode{rotate} could use a concrete range type instead of \tcode{pair} as their return type,
improving composability. It would also be useful to implement a \tcode{view::all} range view that
yields a lightweight range object that refers to all the elements of a container.

\pnum
A future paper will propose such a type.

http://www.open-std.org/jtc1/sc22/wg21/docs/papers/2012/n3350.html

\rSec1[future.views]{Range Views}

\pnum
The vast majority of the power of ranges comes from their composability. Views on existing Iterables
can combine in chains of operations that evaluate lazily, giving programmers a concise and efficient
way to express rich transformations on data. This paper is narrowly focused on the concepts and the
algorithms, leaving range views as critically important future work.

\rSec1[future.views]{Range Actions}

\pnum
If range views are composable, non-mutating, lazy algorithms over ranges, then range \tcode{actions}
are composable, (potentially) mutating, eager algorithms over ranges. Such actions would allow users
to send a container through a pipeline to sort it and remove the non-unique elements, for instance.
This is something the range views cannot do. Range actions sit along side the views in the
programmers toolbox.

\rSec1[future.facade]{Range Facade and Adaptor Utilities}

\pnum
Until it becomes trivial for users to create their own iterator types, the full potential of
iterators will remain unrealized. The range abstraction makes that achievable. With the right library
components, it should be possible for users to define a range with a minimal interface
(e.g., \tcode{current}, \tcode{done}, and \tcode{next} members), and have iterator types
automatically generated. Such a range \textit{facade} class template is left as future work.

\pnum
Another common need is to adapt an existing range. For instance, a lazy \tcode{transform} view
should be as simple as writing an adaptor that customizes the behavior of a range by passing each
element through a transformation function. The specification of a range \textit{adaptor} class
template is also left as future work.

\rSec1[future.infrng]{Infinite Ranges}

\pnum
It is not hard to define a type that represents an ``infinite'' range. Imagine a random number
generator that keeps generating new random numbers for every application of \tcode{operator++}
and \tcode{operator*} of its iterator. Indeed, the very existence of the \tcode{unreachable}
sentinel type -- which this document proposes -- encourages users to think about some ranges as
``infinite''. The sad reality is that infinite ranges are disallowed by the iterator concept
hierarchy as defined in this document.

\pnum
The problem with infinite ranges is the requirement that \tcode{WeaklyIncrementable} types have an
associated \tcode{DifferenceType}. The \tcode{DifferenceType} must be a built-in integral type, and
it must be large enough to hold the distance between any two valid iterators. This is implicit in
the fact that the \tcode{distance} iterator primitive returns objects of type
\tcode{DifferenceType}.

\pnum
Given the fact that there are no infinite precision built-in integral types, the presence of
\tcode{DifferenceType} in the concept requirements places a hard upper limit on how big ranges can
be. This is a severe limitation.

\pnum
Some observations:

\begin{itemize}
\item Not all possible ranges have finitely countable elements.
\item Even a range with finitely countable elements may not be countable within the precision of the
built-in integral types.
\item Not all algorithms require the ability to count increments. Algorithms like \tcode{distance}
and \tcode{count} require finitely countable iterators, but \tcode{find} and \tcode{accumulate} do
not.
\end{itemize}

\pnum
The above above observations suggest that there is a factorization of the existing concept hierarchy
that could solve the problem. Some \tcode{Incrementable}s are finitely countable with a built-in
integral \tcode{DifferenceType}, and some are not. The algorithms that require countability must
say so with an extra requirement.

\pnum
Obviously, some algorithms like \tcode{accumulate} should never be asked to operate on a truly
infinite sequence, even if they don't actually maintain counters internally. Such algorithms could
still be used with ``possibly'' infinite sequences. Note that the standard already defines such
possibly infinite sequences; for instance, there is no limit in principle to the number of elements
in a range delimited with \tcode{istream_iterator}s. It's not hard to imagine other useful possibly-
infinite range types: an infinite range that is delimited with a predicate or a sentinel value, for
instance. The algorithm requirements should merely enforce that there is no integer overflow
possible if the range happens to be larger than can be represented, not necessarily whether the
algorithm will terminate or not.

\rSec1[future.commontype]{Common Type}

\pnum
The all-important \tcode{Common} concept relies on the existence of a SFINAE-friendly
\tcode{common_type} trait, which did not make it into C++14. Solving the outstanding issues
(active issues: \#2450, \#2460, \#2465) (defects: \#2408) with \tcode{common_type} is left as future
work on which the correctness of this document depends.

\rSec1[future.numcont]{Numeric Algorithms and Containers}

\pnum
The numeric algorithms must also be constrained, and additional range-based overloads must be added.
Also, the containers should be made range-aware; that is, they should have constructors and insert
and append member functions that accept range arguments. These things can be done in a separate
proposal.


\rSec1[future.constraints]{Verbosity in Algorithm Constraints}

\pnum
The constraints of some of the algorithms are verbose. Some composite concepts exist that group
constraints together to increase uniformity, aid comprehensibility, and reduce verbosity for those
algorithms that use them. See \tcode{Sortable}, \tcode{Mergeable}, and \tcode{Permutable}, for
instance. There may be other useful clusters of concepts for other algorithm groups. Finding them
and using them to improve the algorithm constraints is an important work item.
