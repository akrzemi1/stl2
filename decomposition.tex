%!TEX root = stl2-ts.tex
\infannex{decomposition}{Decomposition Rationale}

\pnum
Concepts in \Cpp are formed from conjunctions and disjunctions of requirements; there is
no subtractive syntax to form a new concept by removing selected requirements from an existing
concept. If one needs a concept ``like \tcode{Semiregular} but without default construction''
it is necessary to either duplicate all of the desired requirements in the definition
of a new concept, or to refactor the desired requirements from the existing concept to
a new concept (\tcode{concept bool Foo = // ...}) and declare that the existing concept
refines the new one (\tcode{concept bool Semiregular = Foo \&\& // ...}). This process of
refactoring an existing concept into two is referred to as ``decomposition.''

\pnum
The current revision of this document includes concepts formed by decomposition from
\tcode{Semiregular}, which is itself a decomposition of \tcode{Regular} formed by relaxing the
requirement for equality comparison. This section explains why these decomposed concepts are
useful to the library specification.

\rSec1[decomposition.problem]{Problem Discussion}

\rSec2[decomposition.problem.feature]{Syntactic Feature Concepts}
\pnum
N4382 inherited from the C++ Standard several concepts that describe individual syntactic
features of types:
\begin{itemize}
\item \tcode{Destructible}
\item \tcode{Constructible}
\item \tcode{DefaultConstructible}
\item \tcode{MoveConstructible}
\item \tcode{CopyConstructible}
\item \tcode{Assignable}
\item \tcode{MoveAssignable}
\item \tcode{CopyAssignable}
\end{itemize}
These concepts are not useful in isolation, e.g. the capability to copy construct an object
isn't useful if that copy cannot then be destroyed. They are building blocks that are used
to compose useful constraints for object types piecemeal from a list of desired features.
This design is flexible, but awkward - not all combinations of features are sensible, and
the resulting constraints are verbose and error prone. Two examples of this problem in
a post-N4382 revision of this document are the constraints on \tcode{swap} :
\begin{codeblock}
template <class T>
  requires MoveConstructible<T> && MoveAssignable<T> // oops - forgot Destructible<T>
void swap(T&, T&);
\end{codeblock}
and the \tcode{Function} concept:
\begin{codeblock}
template <class F, class ...Args>
concept bool Function =
  Destructible<T> &&
  CopyConstructible<T> &&
  requires(F f, Args...args) {
    typename ResultType<F, Args...>;
    { &f } -> Same<F*>;
    { f.~F() } noexcept;   // oops, Destructible<T> already requires this
    { new F } -> Same<F*>; // oops, mistakenly requires default construction
    { delete new F };
    { f(forward<Args>(args)...) } -> Same<ResultType<F, Args...>{}>;
  };
\end{codeblock}

\pnum
This design violates one of the fundamental ideals of the Palo Alto Report:
\begin{quote}
The concepts used should express general ideas in the application domain (hence the name
``concepts'') rather than mere programming language artifacts. ... concepts should represent
fully formed abstractions.
\end{quote}
The syntactic feature concepts do not ``represent fully formed abstractions.''

\rSec2[decomposition.problem.object]{Object Concepts}

\pnum
Conversely, N4382 inherited two concepts from the Palo Alto Report that form a small
hierarchy of object types:
\begin{itemize}
\item \tcode{Semiregular}
\item \tcode{Regular}
\end{itemize}
These concepts stand on their own by including a set of features with attendant semantics
that describe useful families of object types. There is a gap in the design space below
\tcode{Semiregular} that N4382 fills by using the feature concepts together with subsets
of the requirements of \tcode{Regular} to describe other useful object types. The Palo
Alto Report acknowledges the existence of such useful relaxations of \tcode{Regular} in
its decomposition of \tcode{Semiregular} into \tcode{Movable} and \tcode{Copyable} in
Appendix D.1.

\pnum 
It seems that the solution to the design problems the feature concepts present is to
pre-compose the useful families of object types and insert them into the \tcode{Regular}
hierarchy, allowing us to eliminate the feature concepts from the design altogether. The
determination of which combinations are in fact useful is guided by a few few simple precepts
informed by usage in the standard library and the requirements of the \tcode{Regular} hierarchy:
\begin{itemize}
\item Addressability is fundamental. Objects whose addresses cannot be taken with \tcode{\&}
simply are not value-semantic objects. 
\item Destruction is prerequisite to construction. It must be forbidden to create an
object which cannot be destroyed. This is so fundamental a notion to C++ that the
standard library requires it implicitly.
%% Andrew: If we ever have to write Destructible as a constraint on [an] algorithm, we've done something wrong.
\item Deallocation is equivalent to destruction. The capability to destroy an object should
not depend on whether that object has automatic or dynamic storage duration.
\item Allocation is equivalent to construction. The author of a library component should
have the freedom to determine whether an object should be constructed in automatic storage,
as a member subobject, or on the heap to best meet the design requirements of that component.
\item Construction is prerequisite to assignment. If there's a way to transfer the value
of one object to another, it should be possible to transfer that value into a freshly
created object.
\item Move is prerequisite to copy. Object types that can be copied but for which moves are
ill-formed are pathological.
\end{itemize}

\rSec1[decomposition.concepts]{Decomposed Hierarchy}

\pnum
This section presents the decomposed concepts and describes how they form a hierarchy that
embodies the design precepts presented above. Our starting point is the decomposition of
\tcode{Semiregular} presented in the Palo Alto Report's Appendix D (presented here in N4377
concept syntax, with axioms elided for brevity):

\begin{codeblock}
template <class T>
concept bool Movable() {
  return requires (T a, T b) {
    // requirements for move construction
    T{move(a)};
    // requirements for move assignment
    { a = move(b) } -> Same<T&>;
    // requirements for move (de-) allocation
    { new T{move(a)} } -> Same<T*>;
    delete new T{move(a)};

    // general object requirements that are
    // here because Movable is the base of the
    // concept hierarchy:
    // addressability
    { &t } -> Same<T*>;
    // destruction
    { t.~T() } noexcept;
  };
}

template <class T>
concept bool Copyable() {
  return Movable<T>() && // refines Movable with
    requires (T a, T b) {
      // copy construction
      T{a};
      // copy assignment
      { a = b } -> Same<T&>;
      // copy (de-) allocation
      { new T{a} } -> Same<T*>;
      delete new T{a};
    };
}

template <class T>
concept bool Semiregular() {
  return Copyable<T>() && // refines Copyable with:
    requires {
      // default construction
      T{};
      // default allocation
      { new T{} } -> Same<T*>;
      // array (de-) allocation
      delete[] new T[1];
    };
}
\end{codeblock}
The \tcode{Copyable} and \tcode{Movable} concepts are necessary to describe both object types
that are modeled by the standard library, and requirements on user types with which the standard
library interacts. Simply changing those requirements to \tcode{Semiregular} would cause too
much breakage of existing code, and result in inefficiencies when the old code is updated to meet
the requirements of the stricter concept. This would violate the general ``Don't pay for what you
don't use'' principle of \Cpp.

Looking beyond the algorithms and iterators that were the focus of N3351 - and N4382, for that
matter - the standard library requires concepts to describe:
\begin{itemize}
\item object types that are copy/move constructible but without assignment requirements
\item object types that are constructible from a specific set of argument types
\item object types that are only destructible (although the applicability of such a concept is
narrow, it provides a convenient basis for the other object concepts)
\end{itemize}
We introduce concepts named \tcode{CopyConstructible}, \tcode{MoveConstructible},
\tcode{Constructible}, \tcode{DefaultConstructible}, and \tcode{Destructible} to match these usages
in the library, formed by further decomposition of \tcode{Copyable} and \tcode{Movable}. Although
the Palo Alto Report warns against successive decomposition:
\begin{quote}
We recognize that these concepts could be further decomposed into smaller and smaller units.
For example, we could easily see factoring out shared ``object requirements'' for the
\tcode{Copyable}, \tcode{Movable}, and \tcode{Function} concepts. However, each subsequent
refactoring yields concepts with less coherent meaning (if any).
\end{quote}
the proverbial ship has already sailed; the standard library uses these concepts in its design
now. We simply provide convenient names by which to refer to them, and a coherent hierarchy
relating them to each other as decompositions of \tcode{Regular}. Further, we feel that a larger
body of concepts with fewer distinct requirements is more readable than a smaller body of concepts
individually composed of more and/or duplicated requirements.

\rSec2[decomposition.concepts.destructible]{Destructible (\ref{concepts.lib.object.destructible})}
(As with the presentation of the N3351 concepts, we omit the non-syntactic requirements for
brevity. Each concept is fully specified in [concepts.lib.object] (\ref{concepts.lib.object}).)

\begin{codeblock}
template <class T>
concept bool Destructible() {
  return requires (T t, T* p) {
    { t.@$\sim$@T() } noexcept;
    delete p;
    delete[] p;
    { &t } -> Same<T*>;
  };
}
\end{codeblock}

\tcode{Destructible} is modeled by object types that are, unsuprisingly, destructible.  Note
that \tcode{Destructible} only admits non-array object types since they are the only types for which
the syntax \tcode{t.$\sim$T()} is valid. It includes requirements for deallocation per the design
precept that ``destruction is equivalent to deallocation.'' Since \tcode{Destructible} is the basis
of the concept hierarchy, it includes requirements that are applicable to all value-semantic
object types - namely addressability.

\rSec2[decomposition.concepts.constructible]{Constructible (\ref{concepts.lib.object.constructible})}
\begin{codeblock}
template <class T, class ...Args>
concept bool Constructible() {
  return requires (Args&&...args) {
    T{forward<Args>(args)...};
  } &&
  (is_reference<T>::value ||
    (Destructible<T>() && requires (Args&&...args) {
      new T{forward<Args>(args)...};
    }));
}
\end{codeblock}

\tcode{Constructible} describes either an object type \tcode{T} that is constructible from argument types
\tcode{Args...}, or a reference type \tcode{T} that can be bound to \tcode{Args...}. For the object case,
it includes allocation requirements per the design precept that ``construction is equivalent to allocation''
and refines \tcode{Destructible} per ``destruction is prerequisite to construction.''

\rSec2[decomposition.concepts.defaultconstructible]{DefaultConstructible (\ref{concepts.lib.object.defaultconstructible})}
\begin{codeblock}
template <class T>
concept bool DefaultConstructible() {
  return Constructible<T>() &&
    requires (size_t n) {
      new T[n]{};
    };
}
\end{codeblock}

\tcode{DefaultConstructible} adds an array allocation requirement onto the requirements of
\tcode{Constructible<T>}. Since array initialization copy-initializes the array elements,
this effectively requires both that any user-defined overload of \tcode{new T[]} is
accessible, and that \tcode{T} not have an explicit default constructor.

\rSec2[decomposition.concepts.moveconstructible]{MoveConstructible (\ref{concepts.lib.object.moveconstructible})}
\begin{codeblock}
template <class T>
concept bool MoveConstructible() {
  return Constructible<T, remove_cv_t<T>&&>() &&
    requires (remove_cv_t<T> t) {
      new T[1]{move(t)};
    };
}
\end{codeblock}
\tcode{MoveConstructible} describes an object type \tcode{T} whose objects can be move constructed:
constructed from rvalues with the same value type (\tcode{remove_cv_t<T>}). It also requires array
allocation, effectively forbidding \tcode{T} to have an explicit move constructor.

\rSec2[decomposition.concepts.copyconstructible]{CopyConstructible (\ref{concepts.lib.object.copyconstructible})}
\begin{codeblock}
template <class T>
concept bool CopyConstructible() {
  return MoveConstructible<T>() &&
    Constructible<T, remove_cv_t<T>&>() &&
    Constructible<T, const remove_cv_t<T>&>() &&
    Constructible<T, const remove_cv_t<T>&&>() &&
    requires (remove_cv_t<T>& a, const remove_cv_t<T>& b) {
      new T[1]{a};
      new T[1]{b};
      new T[1]{move(b)};
    };
}
\end{codeblock}
\tcode{CopyConstructible} describes an object type \tcode{T} whose objects can be copy constructed:
constructed from (possibly \tcode{const}) lvalues or \tcode{const} rvalues with the same value type
(\tcode{remove_cv_t<T>}). The array allocation requirements forbid \tcode{T} to have explicit
copy constructor(s).

\rSec2[decomposition.concepts.movable]{Movable (\ref{concepts.lib.object.movable})}
\begin{codeblock}
template <class T>
concept bool Movable() {
  return MoveConstructible<T>() &&
    Assignable<T&, T&&>();
}
\end{codeblock}
\tcode{Movable} refines \tcode{MoveConstructible} with the requirement that assignment to an object
from an rvalue of the same object type transfers the value intact from the source to the destination
object. It is roughly equivalent to \tcode{Movable} from the Palo Alto Report.

\rSec2[decomposition.concepts.copyable]{Copyable (\ref{concepts.lib.object.copyable})}
\begin{codeblock}
template <class T>
concept bool Copyable() {
  return CopyConstructible<T>() &&
    Movable<T>() &&
    Assignable<T&, T&>() &&
    Assignable<T&, const T&>() &&
    Assignable<T&, const T&&>();
}
\end{codeblock}
\tcode{Copyable} refines both \tcode{CopyConstructible} and \tcode{Movable} with additional
requirements that assignment to an object from a possibly-\tcode{const} lvalue or \tcode{const}
rvalue of the same object type replicates the value of the source object in the destination object.
It is roughly equivalent to \tcode{Copyable} from the Palo Alto Report.

\rSec2[decomposition.concepts.semiregular]{Semiregular (\ref{concepts.lib.object.semiregular})}
\begin{codeblock}
template <class T>
concept bool Semiregular() {
  return Copyable<T>() &&
    DefaultConstructible<T>();
}
\end{codeblock}
Just as in the Palo Alto Report, \tcode{Semiregular} refines \tcode{Copyable}. This formulation,
however, indirectly incorporates the requirements for default construction, allocation, and array
deallocation by refining \tcode{DefaultConstructible<T>} instead of listing them explicitly.

\rSec2[decomposition.concepts.regular]{Regular (\ref{concepts.lib.object.regular})}
\begin{codeblock}
template <class T>
concept bool Regular() {
  return Semiregular<T>() &&
    EqualityComparable<T>();
}
\end{codeblock}
\tcode{Regular} is roughly equivalent to the formulation in the Palo Alto Report.

\rSec1[decomposition.usage]{Usage}

\pnum
The utility of these object concepts versus the syntactic feature concepts should be clear: each
includes a complement of required semantic behaviors along with its syntactic requirements resulting
in fully formed abstractions. Similarly to how the strong type system of \Cpp turns many programming
errors into type errors, the use of these stronger concepts makes it easier to avoid semantic errors.
Requirements formed using these concepts are both more concise and more strict than the equivalents
formed by composition of syntactic feature concepts.

\pnum
Since ``destruction is prerequisite to destruction,'' implicit requirements for destruction become
explicit and cannot be inadvertently ommitted. The declaration of \tcode{swap}, for example, becomes:
\begin{codeblock}
template <Movable T>
void swap(T&, T&);
\end{codeblock}
One of the design goals of STL2 is to explicitly specify such implicit requirements. The use of these
stronger concepts makes it possible to do so without adding verbiage to the specification.

\pnum
Pushing requirements down the hierarchy allows for reuse of requirements without replication. The
\tcode{Function} concept becomes:
\begin{codeblock}
template <class F, class ...Args>
concept bool Function() {
  return CopyConstructible<T>() &&
    requires (F f, Args&&...args) {
      typename ResultType<F, Args...>;
      { f(forward<Args>(args)...) } -> Same<ResultType<F, Args...>{}>;
    };
}
\end{codeblock}
which is significantly more concise than the definition from the Palo Alto Report despite
incorporating more requirements.
